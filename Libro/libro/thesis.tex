% file thesis.tex
% Archivo thesis.tex
% Documento maestro que incluye todos los paquetes necesarios para el documento
% principal.

% Documento obtenido por un sinfin de iteraciones de administradores del LDC
% Estructura actual hecha por:
% Jairo Lopez <jairo@ldc.usb.ve>
% Actualizado ligeramente por:
% Alexander Tough 

\documentclass[oneside,12pt,letterpaper]{report}
\tolerance=1000  
\hbadness=10000  
\raggedbottom

% Para escribir algoritmos
\usepackage{listings}
\usepackage{algpseudocode}
\usepackage{algorithmicx}
\usepackage{algorithm}

% Paquetes para manejar graficos
\usepackage{epsf}
\usepackage[pdftex]{graphicx}
\usepackage{epsfig}
% Simbolos matematicos
\usepackage{latexsym,amssymb}
% Paquetes para presentar una tesis decente.
\usepackage{setspace,cite} % Doble espacio para texto, espacio singular para
                           % los caption y pie de pagina
\usepackage[table]{xcolor}
\usepackage{tikz}
\usepackage[T1]{fontenc}

\usetikzlibrary{shapes.geometric,arrows}

\usetikzlibrary{arrows,shapes}
\usepackage{verbatim}

\usepackage{comment}

% Paquetes no utilizados para citas
%\usepackage{mcite} 
%\usepackage{draft} 

\usepackage{wrapfig}
\usepackage{alltt}

% Acentos 
\usepackage[spanish,activeacute,es-noquoting]{babel}

\usepackage[spanish]{translator}
\usepackage[utf8]{inputenc}
\usepackage{color, xcolor, colortbl}
\usepackage{multirow}
\usepackage{subfig}
\usepackage[OT1]{fontenc}
\usepackage{tocbibind}
\usepackage{anysize}
\usepackage{listings} 

% Para poder tener texto asiatico
%\usepackage{CJK}

% Opciones para los glosarios
\usepackage[style=altlist,toc,numberline,acronym]{glossaries}
\usepackage{glossaries}
\usepackage{url}
\usepackage{amsthm}
\usepackage{amsmath}
\usepackage{fancyhdr} % Necesario para los encabezados
\usepackage{fancyvrb}
\usepackage{makeidx} % En caso de necesitar indices.
\makeindex  % Necesitado para los indices
\usepackage{nomencl}

%para graficos
\usepackage{tikz}
\usepackage{pgf-pie}
% Definiciones para definicions, teoremas y lemas
\theoremstyle{definition} \newtheorem{definicion}{Definici\'{o}n}
\theoremstyle{plain} \newtheorem{teorema}{Teorema}
\theoremstyle{plain} \newtheorem{lema}{Lema}

% Para la creacion de los pdfs
\usepackage{hyperref}

% Para resolver el lio del Unicode para la informacion de los PDFs
% En pdftitle coloca el nombre de su proyecto de grado/pasantia.
% En pdfauthor coloca su nombre.
\hypersetup{
    pdftitle = {Desarrollo de un prototipo de robot humanoide que busque, encuentre y patee una pelota },
    pdfauthor={Jennifer Dos Reis y Juliana Leon},
    colorlinks,
    citecolor=black,
    filecolor=black,
    linkcolor=black,
    urlcolor=black,
    backref,
    pdftex
}

\definecolor{brown}{rgb}{0.7,0.2,0}
\definecolor{darkgreen}{rgb}{0,0.6,0.1}
\definecolor{darkgrey}{rgb}{0.4,0.4,0.4}
\definecolor{lightgrey}{rgb}{0.95,0.95,0.95}

\usepackage{listings}
\lstnewenvironment{code}{\lstset{basicstyle=\small}}{}

\lstset{escapeinside=~~}
\lstset{
   frame=single,
   framerule=1pt,
   showstringspaces=false,
   basicstyle=\footnotesize\ttfamily,
   keywordstyle=\textbf,
   backgroundcolor=\color{lightgrey}
}

\makeglossaries
\makenomenclature
% Incluye el glosario
\section{Glosario de t\'erminos}
\begin{tabular}{r p{16cm}l }
    Framework & \emph{Marco de trabajo}. Es un conjunto de técnicas, conceptos y estilos de trabajo que se establecen para resolver un problema particular y que sirve de referencia para solucionar problemas similares.\\

    TightVNC & Es un paquete de software que sirve para controlar la interfaz gráfica  de ordenadores remotos.\\

    AVR & Es una familia de microcontroladores de instrucciones reducidas de la compañía Atmel.\\

    IDE & \emph{Integrated development environment / Entorno de desarrollo integrado}. Es un programa diseñado para facilitar la programación en uno o varios lenguajes. Usualmente incluye herramientas de compilación, editor de textos y depurador.\\

    ROS & \emph{Robot Operating System / Sistema de operación para robots}. Es un framework que provee herramientas para ayudar a desarrolladores de aplicaciones para robots.\\

    Licencia BSD & \emph{Berkeley Software Distribution / distribución de software berkeley}. Es una licencia para software libre otorgada principalmente a sistemas BSD.\\
    
    XBEE & Es una familia de módulos de radio, con protocolo de comunicación inalámbrica basado en radio frecuencias.\\
    
    CSI & \emph{Camera Serial Interface / Interfaz serial para cámaras }. Es un estándar que define la interfaz de comunicación entre una cámara y un procesador. Es comúnmente utilizado en dispositivos móviles.\\
    
   MMAL & \emph{(Multi-Media Abstraction Layer / Capa de abstracción multimedia}: Es una librería que brinda una interfaz de bajo nivel para controlar dispositivos que se ejecutan en el núcleo de video de la Raspberry Pi, como el módulo de cámara.\\
   
   V4L & \emph{Video 4 Linux / video para linux}. Es una interfaz de programación de video para Linux. Algunos dispositivos soportados son cámaras web USB. \\
   
   RGB & \emph{Red Green Blue / rojo, verde, azul}: Es un modelo de color que se basa en la intensidad de los colores primarios de la luz (rojo, verde y azul).\\
   
   HSV & \emph{Hue, Saturation, Value / Matiz, Saturación, Valor}. Es un modelo de color que se basa en las cualidades de matiz, saturación y valor del color. \\
   
	HDMI & \emph{High-Definition Multimedia Interface/ interfaz multimedia de alta definición}. Es una interfaz para transferir datos de audio y video digital entre un dispositivo y un monitor, proyector, televisor digital o dispositivo de audio digital. \\
	
	SD & \emph{Secure Digital / Digital Seguro}. Es un formato de tarjetas de memoria de almacenamiento digital. Existen tarjetas SD con diferentes características en cuanto su clase, capacidad de almacenamiento y tamaño físico.
	
\end{tabular}

%VGA90 (): Es un modo de resolución gráfica para pantallas.. No se bien como es la broma…. >.<
 
 


  
 

  

 


% Para crear la hoja escaneada de las firmas
\usepackage[absolute]{textpos}

% Pone los nombres y las opciones para mostrar los codigos fuentes
\lstset{language=C, breaklines=true, frame=single, showstringspaces=false,
        showtabs=false, numbers=left, keywordstyle=\color{black},
        basicstyle=\footnotesize, captionpos=b }
\renewcommand{\lstlistingname}{C\'{o}digo fuente}
\renewcommand{\lstlistlistingname}{\'{I}ndice de c\'{o}digos fuentes}

\newcommand{\todo}{ TODO: }

% Dimensiones de la pagina
\setlength{\headheight}{15pt}
\marginsize{3cm}{2cm}{2cm}{2cm}
% Crea el glosario

%%%%%%%%%%%%%%%%%%%%%%%%%%%%%%%%%%%%%%%%%%%%%%%%%%%%%%%%%%%%%%%%%%%%%%%%%%%
%%%%%%%%%%%%%%%%      end of preamble and start of document     %%%%%%%%%%%
%%%%%%%%%%%%%%%%%%%%%%%%%%%%%%%%%%%%%%%%%%%%%%%%%%%%%%%%%%%%%%%%%%%%%%%%%%%
\begin{document}

% Pagina de titulo
% Pagina de titulo
\begin{titlepage}
\begin{center}

% Upper part (aqui ya esta incluido el logo de la USB).
\includegraphics[scale=0.5,type=png,ext=.png,read=.png]{imagenes/cebolla} \\

% Encabezado
\textsc {\large UNIVERSIDAD SIMÓN BOLÍVAR} \\
\textsc{\bfseries DECANATO DE ESTUDIOS PROFESIONALES\\
COORDINACI'ON DE INGENIER'IA DE LA COMPUTACI'ON}

\bigskip
\bigskip
\bigskip
\bigskip
\bigskip
\bigskip
\bigskip
\bigskip
\bigskip

% Title/Titulo
% Aqui ponga el nombre de su proyecto de grado/pasantia larga
\textsc{\bfseries Desarrollo de un prototipo robot humanoide que busque, encuentre y patee una pelota}

\bigskip
\bigskip
\bigskip
\bigskip
\bigskip

% Author and supervisor/Autor y tutor
\begin{minipage}{\textwidth}
\centering
Por: \\
Jennifer Dos Reis De Dobrega \\ Juliana Le\'on Quinteiro\\

\bigskip
\bigskip
\bigskip

Realizado con la asesoría de: \\
Ivette Carolina Mart\'inez 
\end{minipage}

\bigskip
\bigskip
\bigskip
\bigskip
\bigskip
\bigskip
\bigskip
\bigskip
\bigskip

% Bottom half
{PROYECTO DE GRADO \\ Presentado ante la Ilustre Universidad Simón Bolívar \\
como requisito parcial para optar al título de \\ Ingeniero en Computación} \\

\bigskip
\bigskip
\vfill

% Date/Fecha 
{\large \bfseries Sartenejas, 
%FECHA
Noviembre 2014}

\end{center}
\end{titlepage}



% Pagina de acta final (vacio)
%\input{acta.tex}
%\includegraphics[width=\textwidth, height=\textheight]{figures/acta.jpg}

\setcounter{secnumdepth}{3}
\setcounter{tocdepth}{4}

% Define encabezado numeros romanos y como se separan los captiulos y las
% secciones
\addtolength{\headheight}{3pt}
\pagenumbering{roman}
\pagestyle{fancyplain}

\renewcommand{\chaptermark}[1]{\markboth{\chaptername\ \thechapter:\,\ #1}{}}
\renewcommand{\sectionmark}[1]{\markright{\thesection\,\ #1}}

\onehalfspacing

\lhead{}
\chead{}
\rhead{}
\renewcommand{\headrulewidth}{0.0pt}
\lfoot{}
\cfoot{\fancyplain{}{\thepage}}
\rfoot{}



% Pagina de resumen
\setcounter{page}{4}
\begin{center}
	{\bf Resumen} \pdfbookmark[0]{Resumen}{resumen} % Sets a PDF bookmark for the dedication
\end{center}	

En este informe se presenta un humanoide de tamaño pequeño (38 cm de alto) construido con las piezas del kit Bioloid Premium \cite{robotics} del fabricante ROBOTIS \cite{robotics1}. Del kit se ha excluido la tarjeta CM-510 para sustituirla por la tarjeta controladora Arbotix, que será la que controle los 16 motores Dynamixel Ax-12+ (para mover al robot) y 2 servomotores analógicos (para mover la cámara). Además se ha agregado un mini computador Raspberry Pi, con su cámara \cite{raspberrycam}, para que el robot pueda detectar y seguir la pelota de forma autónoma. 

Todos estos componentes deben ser coordinados para que se logre cumplir la tarea de detectar, seguir y patear la pelota. Por ello se hace necesaria la comunicación entre la Arbotix y la Raspberry Pi. La herramienta empleada para ello es el framework ROS (Ros Operating System) \cite{ros}.

En la Raspberry Pi se usa el lenguaje c++ y se ejecuta un solo programa encargado de captar la imagen de la cámara, filtrar y procesar para encontrar la pelota, tomar la decisión de la acción a tomar y hacer la petición a la Arbotix para que de la orden a los motores de ejecutar el movimiento. Para captar la imagen de la cámara se ha utilizado la librería raspicam\_cv \cite{camara}. Para filtrar y procesar la imagen se ha usado las librerías OpenCv \cite{opencv}. 

La Arbotix, además de controlar los motores, se encarga de monitorizar que el robot se encuentre balanceado, para ello usa el sensor Gyro de Robotis \cite{gyro}. Si detecta un desbalance de un cierto tamaño puede saber si se ha caído y levantarse. 

% Pagina de dedicatoria (opcional)
%\pagebreak

\setcounter{page}{5}

\vspace*{8cm} 
\pdfbookmark[0]{Dedicatoria}{dedicatoria} % Sets a PDF bookmark for the dedication
\begin{center} 
\large DEDICATORIA
\end{center}
\newpage


% Pagina de agradecimientos (opcional)
\setcounter{page}{6}

\chapter*{Agradecimientos
\markboth{Agradecimientos}{Agradecimientos}}
\pdfbookmark[0]{Agradecimientos}{agradecimientos}

\bigskip

AGRADECIMIENTOS



% Crea la tabla de contenidos
\tableofcontents

% Crea la lista de cuadros
%\listoftables

% Crea la lista de figuras
\listoffigures
\printglossaries

\clearpage
% Crea la lista de codigos fuentes
%\lstlistoflistings

%Glosario
%\glsaddall


\clearpage

% Define encabezado en numeros arabicos  
\pagenumbering{arabic}

\fancyhf{} % Redefine el encabezado 
\lhead{}
\chead{}
\rhead{\fancyplain{}{\thepage}}
\renewcommand{\headrulewidth}{0.0pt}
\lfoot{}
\cfoot{}
\rfoot{}

\doublespacing

% Incluye los archivos deseados - El contenido de su proyecto de grado/pasantia larga.

\chapter*{Introducción}

\pdfbookmark[0]{Introducción}{introduccion} % Sets a PDF bookmark for the dedication

\label{sect:justificacion}

Ya existen aproximaciones a este enfoque, la mayoría relacionados con: el diseño gráfico como 
la herramienta \emph{Brain Storm} del programa \emph{After Effects} de \emph{Adobe} \cite{HR10}, que muestra variaciones de los
elementos en pantalla para sugerir posibles modificaciones; la inteligencia artificial, como 
la herramienta Gecode \cite{Gecode}, que permite resolver problemas algebraicos y lógicos para obtener
rangos de soluciones posibles; y el desarrollo de videojuegos, como los juegos de la saga
\emph{Diablo, Minecraft} \cite{B12}, o \emph{Dwarven Fortress} \cite{D08}, en los que los mapas son generados de forma aleatoria
pero mantienen coherencia.

Incluso hay otro tipo de herramientas o que guardan bastante similitud a esta idea.
Entre ellas destaca \emph{QuickCheck} \cite{TQP}, que es una herramienta en \emph{Haskell} que 
permite generar casos de prueba de estructuras que se han descrito. Sin embargo, tiene
como limitación que sólo está hecha para realizar pruebas y no generar soluciones de
salida para aplicaciones en produción.

En el presente trabajo de investigación se explica el proceso y diseño de la herramienta de generación
de estas instancias aleatorias. Partiendo en el desarrollo de un lenguaje de
programación propio (sección \ref{chapter:def_lenguaje}), para luego ahondar en el proceso de resolución de
problemas y asignación de valores (sección \ref{chapter:dise_solver_compilador}). Por último se evaluarán los resultados de
las pruebas hechas sobre la herramienta y se concluirá acerca de sus posibles mejoras y
enfoques para trabajos futuros (sección \ref{chapter:imp_y_res}).


% Marco Teorico.
\chapter{Marco teórico} \label{chap:marco_teorico}
\vspace{5 mm}
En este capítulo se presentan los conceptos que conforman la base teórica para comprender el presente trabajo. Primero se brinda una descripción de los términos relativos a la robótica y las partes principales de un robot. Posteriormente se describen algunos conceptos que tienen que ver con la robótica inteligente, como los paradigmas, la inteligencia y la visión artificial para la detección de objetos. 

\section{Robótica} \label{sect:robotica}
 
El presente trabajo se basa en la construcción de un robot humanoide, por lo tanto es importante definir qué es un robot, qué significa que sea humanoide y cuáles son algunos de sus componentes principales.

\begin{itemize}
\item{\textbf{Robótica:} Es la rama de la tecnología que se encarga del diseño, construcción, operación y aplicación de los robots \cite{oxfordRobotics}}.

\item{\textbf{Robot:} Es un agente f\'isico que realiza tareas manipulando su ambiente.
Generalmente un robot esta equipado con actuadores y sensores. Una posible divi\'on para categorizarlos es por su forma, en primera instancia estan los manipuladores o brazo rob\'oticos que normalmente estan anclados a su espacio de trabajo y generalmente desempeñan tareas en f\'abricas o l\'ineas de emsamblaje; como su nombre lo describe suelen ser con forma de brazo. La siguiente categor\'ia se refiere a robots moviles, su principal caracter\'itica es el desplazamiento por lo cual poseen piernas, ruedas o cualquier mecanismo que le permita desplazarse, generalmente en esta clasifica\'ion se incluyen robots que realizan multiples tareas como del hogar o reconocimiento de un espacio entre muchas otras. La \'ultima categor\'ia es mixta son aquellos robots que poseen caracter\'isticas tanto de brazos rob\'oticos como de moviles, en ella se incluye a los robots humanoides que por su nombre estan hechos a semejanza del hombre, sus tareas suelen recurrir un poco mas de esfuerzo ya que en la manipulaci\'on de objetos no poseen la ventaja del anclaje que tienen los brazos rob\'oticos. Los robots reales suelen enfrentarse a ambientes parcialmente observables, estoc\'asticos, din\'amicos y continuos  \cite{peterAndNorvig}.}

%Son agentes físicos que ejecutan tareas para manipular el mundo físico. Para ello deden estar equipados con actuadores y sensores \cite{peterAndNorvig}. La apariencia no es una característica útil para la definición de un robot \cite{AiRobotics}, por lo tanto puede ser de diferentes formas, ya sea con ruedas, con piernas o ninguna de ellas. Una de las formas que puede adoptar un robot es la de humano, de hecho en la cultura popular el término ``robot" generalmente connota una apariencia humana \cite{AiRobotics}. Según el diccionario de la Universidad de Oxford,  el término humanoide se refiere a tener una apariencia o característica parecida a la de un ser humano \cite{oxfordRobotics}, por lo tanto a los robots con forma de humano se les denomina robots humaniodes.}    

\item{\textbf{Sensores:} Son los dispositivos que envian percepci\'on del ambiente al robot, algunos miden los cambios o percepci\'on del mundo que rodea como las c\'amaras, los sonares entre otros. Tambi\'en exiten los que miden la propia movilidad del robot como los giroscopios y aceler\'ometros. En general un sensor es una interfaz de percepci\'on entre el ambiente y el robot  \cite{peterAndNorvig}.}

%Son los dispositivos encargados de percibir el ambiente que rodea al robot. Según Murphy R.R estos miden algún atributo del mundo. Un sensor recibe energía del entorno (sonido, luz, presión, temperatura) y transmite una señal a una pantalla o computador ya sea de forma análoga o digital \cite{AiRobotics}. Algunos sensores son: cámaras, giroscopios, sensores de proximidad, entre otros.}

\item{\textbf{Actuador:} Dispositivos que realizan cambios f\'isicos en el medio ambiente. Por ejemplo ruedas, piernas, pinzas, entre otros \cite{peterNorvig}.}

\item{\textbf{Servomotor:}  Es un motor eléctrico, un actuador, que permite controlar la velocidad y la posici\'on  \cite{AiRobotics}. }

 %http://www.ceiarteuntref.edu.ar/badarte/node/112

\item{\textbf{Giroscopio:} Es un sensor de la categor\'ia para la percepci\'on propia que informan al robot de su propio estado, pertence a los sensores de inercia \cite{peterNorvig}. Mide el momento angular y se utiliza para mantener orientaci\'on o equilibrio.}

%Es un sensor utilizado para medir y mantener la orientación, se mide a través del momento angular \cite{gyro1}. }
\end{itemize}

%****************************************************************************************/
\section{Robótica Inteligente} \label{sect:AgentesInteligentes}

Es importante diferenciar cuando un robot es inteligente o no. Cuando un robot es operado a distancia, y no es capaz de cumplir sus tareas sin la intervención de un humano, entonces no se considera  inteligente. Tampoco se considera inteligente si las tareas que ejecuta se hacen sin sentido o de manera repetitiva. En cambio cuando un robot puede interactuar con el mundo de manera autónoma se considera que es un robot o agente inteligente \cite{AiRobotics}. Existen diferentes estrategias o enfoques de cómo aplicar la inteligencia en un robot. Esta sección se dedica a describir los enfoques que en  \cite{AiRobotics} se definen como paradigmas. 
  
   
\subsection*{Paradigmas de robótica}
Según Robin Murphy en \cite{AiRobotics}, existen tres paradigmas en los cuales se clasifica el diseño de un robot inteligente, estos paradigmas pueden ser descritos de dos maneras: la relación entre las primitivas básicas de la robótica:  percibir, planificar, actuar; o de la forma en que los datos son percibidos y distribuidos en el sistema.

Percibir se refiere al procesamiento útil de la información de los sensores del robot. Planificar, cuando con información útil, se crea un conocimiento del mundo y se generan ciertas tareas que el robot podría realizar. Por último actuar consiste en realizar la acción correspondiente con los actuadores del robot para modificar el entorno. 

\subsection{Paradigma Jerárquico}

Este paradigma es secuencial y ordenado. Primero el robot percibe el mundo y construye un mapa general. En base al mapa ya percibido y sin percibir m\'as, el robot planifica todas tareas necesarias para lograr la meta. Luego ejecuta la secuencia de actividades según su planificaci\'on. Una vez culminada la secuencia se repite el ciclo percibiendo el mundo, planificando y actuando \cite{AiRobotics}. El algoritmo general utilizado en este paradigma se muestra en la imagen \ref{fig:jerarquico}.

\begin{figure}[hbtp]

\centering
\includegraphics[scale=0.7]{imagenes/jerarquico.png} 
\caption{Paradigma Jer\'arquico}
\label{fig:jerarquico}
\end{figure}


\subsection{Paradigma Reactivo}
El paradigma reactivo omite por completo el componente de la planificación y s\'olo se basa en percibir y actuar. El robot puede mantener un conjunto de pares percibir-actuar como se muestra en la figura \ref{fig:reactivo}, \'estos son llamados comportamientos y se ejecutan como procesos concurrentes. Un comportamiento toma datos de la percepción del mundo y los procesa para tomar la mejor acción independientemente de los otros procesos \cite{AiRobotics}.

\begin{figure}[hbtp]

\centering
\includegraphics[scale=0.7]{imagenes/reactivo.jpg} 
\caption{Paradigma Reactivo}
\label{fig:reactivo}
\end{figure}


\subsection{Paradigma Híbrido}
El paradigma híbrido es una mezcla de los dos paradigmas anteriores. Primero se planifica cúal es la mejor manera de cumplir el objetivo principal, descomponiendo la tarea general en sub-tareas y decidiendo que comportamientos sirven para cumplir cada una. De allí en adelante se ejecutan los comportamientos (percibiendo y actuando), hasta que el plan sea ejecutado, si es necesario se puede volver a planificar. Vale la pena acotar que la información de los sensores se encuentra disponible para el planificador, de manera que pueda crear un modelo del mundo y tomar decisiones en base a él  \cite{AiRobotics}. 
Se puede apreciar el la figura \ref{fig:hibrido}
\begin{figure}[hbtp]

\centering
\includegraphics[scale=0.7]{imagenes/hibrido.png} 
\caption{Paradigma H\'ibrido}
\label{fig:hibrido}
\end{figure}

\section{Inteligencia Artificial} \label{sect:Inteligencia_Artificial}
La inteligencia artificial es un término relacionado con la computación y la robótica que ha tenido varias definiciones, ocho de ellas, las cuales nacieron a finales del siglo XX, se encuentran organizadas en \cite{peterNorvig} bajo cuatro categorías: pensar y actuar de forma humana, pensar y actuar de forma racional. Con ello se puede entender que la inteligencia artificial tiene que ver con lograr que un robot resuelve problemas de manera inteligente, es decir, de manera que parezca que el razonamiento y comportamiento humano las ha resuelto.  

\subsection{ Aprendizaje de Máquinas}
El aprendizaje de máquinas es un área de la inteligencia artificial que está relacionada con la pregunta de cómo construir programas de computadora que automáticamente mejoren con la experiencia. Se dice que un programa aprende de la experiencia E con respecto a una tarea T y desempeño P. Si el desempeño en la tarea T, medido por P, mejora con con la experiencia E \cite{Mitchell}
\subsection{Aprendizaje por reforzamiento}
El aprendizaje por reforzamiento es un tipo de aprendizaje de máquinas que se basa en un sistema de recompensas y penalizaciones. Las recompensas se pueden dar en cada estado o una sola vez al llegar al estado final.

El objetivo del agente es aprender de las recompensas para escoger la secuencia de acciones que produzca la mayor recompensa acumulada. \cite{Mitchell}

<<<<<<< HEAD
El agente existe en un entorno descrito por algunos estados S. Puede ejecutar un conjunto de acciones A. Cada vez que ejecuta una acción ‘at’ en algún estado ‘at’ el agente recibe una recompensa ‘rt’. El objetivo es aprender una política pi : S \- A que maximice la suma esperada de esas recompensas con descuento exponencial de las recompensas futuras. \cite{Mitchell} El resultado de tomar las acciones puede ser determinista o no, en el caso de este proyecto no es deterninista, es decir, existen porcentajes de probabilidad de pasar a un estado u otro al tomar una acción en un estado en particular.  
=======
El agente existe en un entorno descrito por algunos estados S. Puede ejecutar un conjunto de acciones A. Cada vez que ejecuta una acción a_{t} en algún estado s_{t} el agente recibe una recompensa r_{t}. El objetivo es aprender una política pi : S /rightarrow A que maximice la suma esperada de esas recompensas con descuento exponencial de las recompensas futuras. (Mitchell) El resultado de tomar las acciones puede ser determinista o no, en el caso de este proyecto no es determinista, es decir, existen porcentajes de probabilidad de pasar a un estado u otro al tomar una acción en un estado en particular.  
>>>>>>> 667cf8b3e5b67281d85d6dc28071a6bdf5c07261

\subsection{ Q- learning}

Es un método de aprendizaje por reforzamiento

Compara las utilidades esperadas de acciones posibles sin necesidad de saber el resultado por tanto no se necesita un modelo del entorno \cite{peterNorvig}


\section{Visión Artificial} \label{sect:Vision_Artificial}

Una manera de obtener información del ambiente es con la visión artificial. Esta consiste en usar un dispositivo (cámara) que capta un rango de espectro electromagnético y produce una imagen. La representación de la imagen se almacena como una matriz de píxeles, cada píxel es un elemento que guarda información de una región en el espacio captado. Si se usa una cámara de luz, la información de cada píxel será el color. \cite{AiRobotics}  

Por lo general luego de obtener una imagen se requiere extraer información de ella, por lo cual se han desarrollado diferentes algoritmos y t\'ecnicas que ayudan en esta tarea. En la sección \ref{sec:Segmentacion} se describe la t\'ecnica de segmentaci\'on por regiones, que es la utilizada en el proyecto. 

Por otro lado, existen varios algoritmos que se dedican a la transformación de las imágenes para reducir
ruidos, compensar problemas de iluminación, extraer formas, identificar objetos, entre otros. En esta sección se describen dos de las técnicas de transformación para reducir el ruido basadas en la dilatación y erosión de la imagen (secci\'on \ref{sec:Transfor}). 
 
\subsection{Segmentaci\'on por regiones}\label{sec:Segmentacion}

La pr\'actica m\'as general en visi\'on de computadoras aplicado a rob\'otica es la identificaci\'on de regiones por un color particular, este proceso se llama segmentaci\'on por regiones. El algoritmo b\'asico consiste en identificar todos los p\'ixeles en una imagen que forman parte de una regi\'on y luego ir al centro de la regi\'on. El primer paso es identificar todos los p\'ixeles en la imagen que compartan un rango de valores con el color particular elegido y agruparlos, aquellos p\'ixeles que no compartan el color son descartados \cite{BookOpenCv}. 

\subsection{Filtros}
El filtrado de imágenes es una técnica para la transformación de imágenes, que consiste en destacar  sus características más relevantes en base a un propósito en particular. 

Generalmente en la tarea de extracción de información de una imagen se utilizan filtros para descartar zonas o características que no son importantes para el patrón deseado y para determinar el área deseada ya sea por patrones de forma o color.

En la investigación, los algoritmos de filtrado aplicados a las imágenes fueron: Clausura Morfológica y Apertura Morfológica, filtros que aplican las técnicas de erosión y dilatación a las imágenes.

\subsection*{Transformaciones Morfológicas}\label{sec:Transfor}
Algunas transformaciones morfológicas básicas son dilatación, erosión, uni\'on e intersecci\'on, se utilizan en amplia variedad de contextos como la eliminación del ruido, aislamiento de elementos individuales y elementos de unión dispares en una imagen en este proyecto de utilizaron dilataci\'on y erosi\'on.\cite{BookOpenCv}

\subsubsection{Dilatación}
La dilatación es una convulsión (patr\'on que se le aplica a toda la imagen) entre alguna imagen (o región de una imagen), que llamaremos $A$ y un núcleo que llamaremos $B$, el núcleo, que puede ser de cualquier forma o tamaño,generalmente una figura geom\'etrica un cuadrado o disco, tiene un solo punto de referencia definido. Para mayor claridad el n\'ucleo es una matriz de tamaño fijo de coeficientes numéricos junto con un punto de referencia en dicha matriz, que normalmente se encuentra en el centro. El núcleo puede ser pensado como una plantilla  o m\'ascara, y su efecto para la dilatación tal como un operador de máximo local sobre la imagen, se calcula el m\'aximo valor de los píxeles común a $B$ y reemplazamos el píxel de la imagen en el punto de referencia con ese valor máximo. Esto causa regiones brillantes dentro de una imagen y la hacen crecer. Este crecimiento es el origen del término `` operador de dilatación" \cite{BookOpenCv}. 

\begin{figure}[hbtp]

\centering
\includegraphics[scale=0.2]{imagenes/erosion-model.jpg}
\caption{Dilatación A: es la imagen original, B: es el n\'ucleo, La estrella es el punto de referencia. Se ve como aumenta la imagen en proporci\'on al patr\'on aplicado }
\end{figure}

\subsubsection{Erosión}
La erosión es la operación inversa a la dilatación. Esta acción del operador es equivalente a el cálculo de un mínimo local sobre el área del núcleo. La erosión genera una nueva imagen a partir de la original, utilizando el siguiente algoritmo: como el núcleo $B$ es analizado sobre la imagen, se calcula el mínimo valor del píxel superpuesto por $B$ y se reemplaza el píxel de la imagen con un punto de referencia de valor mínimo \cite{BookOpenCv}. 
V\'ease en la figura \ref{fig:erosion}

\begin{figure}[hbtp]
\centering
\includegraphics[scale=0.3]{imagenes/erosion.jpg}
\caption{Erosión,  A: es la imagen original, B: es el n\'ucleo, La estrella es el punto de referencia. Se ve como disminuye la imagen en proporci\'on al patr\'on aplicado}
\label{fig:erosion}
\end{figure}

Este cap\'itulo constituyo la base te\'orica que sustenta el proyecto, en el siguiente cap\'itulo se presenta el proceso de desarrollo que se sigui\'o para su culminaci\'on. 



\chapter{Construcci\'on de un Robot Humanoide}\label{chapter:introAdesarrollo}

En el presente capítulo se describen todas las actividades que se llevaron a cabo para lograr construir un robot humanoide capaz de detectar la ubicación de una pelota de un color determinado, buscarla, y al llegar a ella patearla en direcci\'on al arco. También se describen las actividades realizadas para lograr que Junny sepa levantarse en caso de tener un desbalance y caer.
El c\'odigo completo de este proyecto se encuentra disponible en: \url{ https://github.com/jenniferdr/raspberryROS}     




% Seccion de Diseno y construccion 

%%\chapter{Integración de componentes}
\label{chapter:diseno}
\section{Diseno y construccion}
\subsection{construccion}
Para la construcción del robot se ha utilizado el kit de piezas Bioloid Premium de marca Robotis el cual incluye motores Dynamixel Ax-12+, una tarjeta controladora CM-510, un sensor Gyro, un manual, entre otros elementos. El manual incluye las instrucciones de como armar varios modelos de humanoide, el utilizado en este proyecto es el tipo B, haciendo uso de 16 motores. En la figura ~\ref{fig:frontal} y ~\ref{fig:trasera1} se puede observar la estructura del robot que aparece en el manual del kit. 

\begin{figure}[hbtp]
\centering
\includegraphics[scale=0.3]{imagenes/Robot.png}
\caption{vista frontal del robot. Se puede apreciar la identificación ‘ID’ de cada motor Dynamixel Ax-12+. Nota: los motores 9 y 10 no se utilizan}
\label{fig:frontal}
\end{figure}

\begin{figure}[hbtp]
\centering
\includegraphics[scale=0.4]{imagenes/RobotTrasero.png}
\caption{Vista trasera del robot}
\label{fig:trasera1}
\end{figure}

En lugar de la utilización de la tarjeta CM-510 se ha decidido usar la tarjeta controladora Arbotix debido a que la controladora CM-510 no acepta la incorporación actuadores o dispositivos adicionales. La Arbotix permite la incorporación de nuevos actuadores y más dispositivos con sencillez. En la figura ~\ref{fig:trasera2} se puede observar la estructura del robot con la Arbotix incorporada. En la parte interna del tronco del robot se sitúa el sensor Gyro.

\begin{figure}[hbtp]
\centering
\includegraphics[scale=0.3]{imagenes/traseroDeJunny.jpg}
\caption{Vista trasera del robot con la Arbotix}
\label{fig:trasera2}
\end{figure}

\begin{figure}[hbtp]
\centering
\includegraphics[scale=0.7]{imagenes/CIMG0225.jpg}
\caption{Manual de instrucciones y piezas del robot}
\end{figure}

Para el movimiento de la cámara se ha incorporado dos servomotores, uno para el movimiento horizontal y otro para el vertical. La conexión es pin a pin en los puertos especiales para ese tipo de motores (‘Hobby servos’) fuente (ver figura ~\ref{fig:puertosHobby}). La cámara ha sido conectada a la Raspberry Pi en el puerto CSI (ver la figura ~\ref{fig:camACSI}). El resultado de estas tres piezas instaladas en el robot se puede apreciar en la figura ~\ref{fig:servosycam}.

\begin{figure}[hbtp]
\includegraphics[scale=0.3]{imagenes/arbotix_hobby_servo.jpg}
\includegraphics[scale=0.7]{imagenes/arbotix_hobbyservos_lines.jpg}
\caption{Ilustración de los puertos Hobby de la Arbotix}
\label{fig:puertosHobby}
\end{figure}

\begin{figure}[hbtp]
\centering
\includegraphics[scale=1]{imagenes/raspbCam.jpg}
\caption{Camara Raspberry Pi conectada al puerto CSI de la tarjeta}
\label{fig:camACSI}
\end{figure}
 
\begin{figure}[hbtp]
\centering
\includegraphics[scale=0.1]{imagenes/servosYcamara.JPG}
\caption{vista delantera del robot con la cámara y servomotores instalados}
\label{fig:servosycam}
\end{figure}

Los motores Dynamixel se conectan a la controladora Arbotix por medio de los puertos bioloid de la tarjeta. Sin embargo como la tarjeta solo cuenta con tres puertos y el robot posee cuatro extremidades, se ha optado por agregar un expansor de puertos bioloid y así conectar cada extremidad en un puerto diferente. La forma en la que se ha conectado estos motores se ejemplifica en 
la figura ~\ref{fig:arbotixConectados}. 

\begin{figure}[hbtp]
\centering
\includegraphics[scale=0.2]{imagenes/arbotix_servo.png}
\caption{Tarjeta controladora Arbotix y componentes conectados }
\label{fig:arbotixConectados}
\end{figure}

La comunicación de la tarjeta de Arbotix con la computadora, incluso con la Raspberry Pi, se realiza a través del puerto FTDI por medio un chip conectado como lo ilustra la figura ~\ref{fig:arbotixConectados}.

Como fuente de poder se ha utilizado una batería de polímero de litio de 11.1 V y 1 amp. Debido a que no todos los componentes poseen las mismas exigencias con respecto a voltaje y amperaje, se realizó un regulador (ver figura ~\ref{fig:circuito}) con  salida de 5 voltios para la tarjeta Raspberry Pi y los dos micro servomotores, y otra salida de 11.1 V para la tarjeta Arbotix que a su vez alimenta a los componentes conectados en ella (motores Dynamixel y Giroscopio).

\begin{figure}[hbtp]
\centering
\includegraphics[scale=0.8]{imagenes/circuito.jpg}
\caption{Circuito con entrada de 11.1 V. Una salida de 5 V para los micro servomotores analogicos y tarjeta Raspberry Pi. Otra salida de 11 v para alimentar la controladora Arbotix.}
\label{fig:circuito}
\end{figure}

\subsection{componentes de hardware}
En esta sección se describen los principales componentes utilizados para armar la estructura del robot.
\begin{itemize}
\item Bioloid Premium kit: Es un kit (figura ~\ref{fig:kit}) de robótica con piezas modulares que permite armar diferentes tipos de robot pero principalmente humanoides. El fabricante, ROBOTIS, incluye un manual con varios modelos de robots con instrucciones de ensamblaje. Provee una tarjeta controladora, CM-530, a la que se conectan los motores Dynamixel y algunos sensores que se programan a través de la interfaz de ‘RoboPlus’. \cite{robotics}

\end{itemize}

\begin{figure}[hbtp]

\centering
\includegraphics[scale=0.5]{imagenes/product_bioloid17.png}
\caption{Bioloid Premium Kit}
\label{fig:kit}
\end{figure}

\begin{itemize}

\item Motores Dynamixel Ax-12+: Son actuadores inteligentes y modulares que incorporan un reductor de engranajes, un motor DC  (figura ~\ref{fig:motoresDc})
de presión y un circuito de control con funcionalidad de red, todo en un solo paquete \cite{manual}. 
\end{itemize}

\begin{figure}[hbtp]
\label{fig:motoresDc}
\centering
\includegraphics[scale=0.5]{imagenes/AX-12_serie.png}
\caption{Motores Dynamixel conectados en serie}
\end{figure}

\begin{itemize}
\item Gyro: Es un giroscopio (figura ~\ref{fig:gyro}) de la marca Robotis que mide la velocidad angular, diseñado para mantener el balance del robot y
ser usado para otras aplicaciones de movimiento. \cite{gyro} 

\end{itemize}

\begin{figure}[hbtp]
\centering
\label{fig:gyro}
\includegraphics[scale=0.5]{imagenes/gyro.jpg}
\caption{Sensor Gyro}
\end{figure}

\begin{itemize}
\item Arbotix: El controlador ArbotiX es una solución de control avanzado para manejar servos Dynamixel AX/MX/RX/EX y robots
basados en Bioloid. Incorpora un potente microcontrolador AVR, radio inalámbrica XBEE, conductores de motor dual, y cabeceras
de estilo servo de 3 pines para E/S digital y analógica.\cite{arbotix}

\end{itemize}

%\begin{figure}[hbtp]
%\centering
%\includegraphics[scale=0.5]{imagenes/ARBOTIX.JPG}
%\caption{Tarjeta controladora ArbotiX}
%\end{figure}

\begin{itemize}
\item FTDI (Future Technology Devices International) : Es una tarjeta controladora  (figura ~\ref{fig:ftdi})que ofrece el servicio de conversión de 
datos de USB a UART. Permite la comunicación entre diferentes dispositivos \cite{ftdi}.

\end{itemize}

\begin{figure}[hbtp]
\centering
\label{fig:ftdi}
\includegraphics[scale=0.09]{imagenes/DSCF1162.jpg}
\caption{Chip FTDI conectado a la tarjeta Arbotix}
\end{figure}

\begin{itemize}
\item Extensor de puertos bioloid : Permite aumentar el número de cadenas de servos conectados a la tarjeta. (figura ~\ref{fig:ext}) \cite{hub} 
\end{itemize}

\begin{figure}[hbtp]
\centering
\label{fig:ext}
\includegraphics[scale=0.3]{imagenes/Dynamixel-AX-MX-6-Port-Extension-Hub-600x600.jpg}
\caption{Extensor de puertos bioloid}
\end{figure}

\begin{itemize}
\item Servo motor analogico micro TG9 e: Es un pequeño servomotor  (figura ~\ref{fig:Servo}) cuyo torque alcanza 1.50 kg-cm y una velocidad de 60 por
segundo. Permite ser controlado en posición en un rango de 180. \cite{microservo}  

\end{itemize}

\begin{figure}[hbtp]
\centering
\label{fig:Servo}
\includegraphics[scale=0.3]{imagenes/turnigy.jpg}
\caption{Servo motor analógico}
\end{figure}

\begin{itemize}
\item Raspberry Pi: La Raspberry Pi (figura ~\ref{fig:Raspe}) es un ordenador del tamaño de una tarjeta de crédito a la que se puede conectar un 
televisor y un teclado. Se trata de un pequeño ordenador capaz de ser utilizado en proyectos de electrónica, y para muchas 
de las tareas que una PC de escritorio hace, como hojas de cálculo, procesadores de texto y juegos \cite{raspberry}. 

\end{itemize}

%imagen tomada de: %http://rayhightower.com/blog/2012/12/03/ruby-on-raspberry-pi/
\begin{figure}[hbtp]
\centering
\label{fig:Raspe}
\includegraphics[scale=0.1]{imagenes/raspberry_pi_iphone.jpg}
\caption{Tarjeta Raspberry Pi con descripción de los puertos}
\end{figure}

\begin{itemize}
\item Camara Raspberry Pi: Es un sensor encargado de captar imagenes y grabar videos de alta definicion. Se conecta a la Raspberry Pi con un cable de cinta plana de 15 cm en el puerto CSI. Tiene 5 megapíxeles de foco fijo que soporta los modos de vídeo de 1080x30, 720x60 y VGA90. Puede ser manejada con las librerías MMAL, V4L u otras librerías de terceros como la de
Python.(figura ~\ref{fig:came}) \cite{raspberrycam} %(http://www.raspberrypi.org/products/camera-module/)

\end{itemize}

\begin{figure}[hbtp]
\centering
\label{fig:came}
\includegraphics[scale=0.7]{imagenes/1367-01.jpg}
\caption{Camara Raspberry Pi}
\end{figure}


\begin{itemize}
\item Batería de polímero de litio (Lipo): Es la fuente de poder usada para que los motores y componentes electronicos
funcionen. La batería usada es de 11.1 voltios y 1 amperio. \cite{bateria}
\end{itemize}


\begin{figure}[hbtp]
\centering
\includegraphics[scale=0.5]{imagenes/R-LIPOBAT.jpg}
\caption{Batería Lipo}
\end{figure}

\begin{itemize}
\item Circuito con regulador de 5v: Es un circuito diseñado y construido para este proyecto cuya finalidad es regular la 
entrada de la corriente. Por una de las salidas se expulsa 5v y por la otra se mantiene el mismo voltaje de entrada. 
\end{itemize}

%\begin{figure}[hbtp]
%\centering
%\includegraphics[scale=0.7]{imagenes/circuito.jpg}
%\caption{Lipo}
%\end{figure}

\subsection{ herramientas software }

En esta sección se describen las herramientas de software utilizadas para la programación del proyecto.
\begin{itemize}
\item Pypose: Software especializado en el control de los servomotores Dynamixel Ax-12. Una de las más importantes
características es que, luego de haber fijado a mano las posiciones de los motores, permite la lectura simultánea de esas
posiciones para captar la pose del robot. Con esta herramienta es posible formar una secuencia de poses que generen un
movimiento, por ejemplo, caminar. \cite{pypose}

\item ROS: ROS (Robot Operating System) es un framework que proporciona bibliotecas y herramientas para ayudar a los desarrolladores de software a crear aplicaciones robóticas. Proporciona abstracción de hardware,  de dispositivos, bibliotecas, visualizadores, paso de mensajes, gestión de paquetes y más. ROS se encuentra bajo licencia de código abierto, la licencia BSD.

\item OpenCv (Open Source Computer Vision Library): Es una librería de visión de computadoras y aprendizaje de máquinas de código abierto. Ha sido diseñada para acelerar el uso de la percepción de maquinas y para proveer una estructura común en las aplicaciones de visión de computadoras. Registrada bajo la licencia BSD, de código abierto. \cite{opencv}

\item IDE Arduino: Es un entorno de desarrollo para escribir y cargar código en la tarjeta Arduino. Otras tarjetas
con microcontroladores AVR también son compatibles, como la ArbotiX. El lenguaje de programación del IDE de Arduino es una
implementación de Wiring el cual esta basado en Processing.  \cite{arduino}

\end{itemize}

\section{Detección de la pelota}\label{chapter:deteccion}

La recopilación de información del medio ambiente, para detectar la posición de la pelota, se realiz\'o por medio de la cámara Raspberry Pi con el sistema operativo Raspian. Esta mini computadora es una herramienta ya que permite capturar v\'ideos y fotos de alta definici\'on. Como la Raspberry Pi cuenta con un procesador gr\'afico, este se encarga de manejar los datos de la cámara, aliviando la carga del procesador central \cite{raspCamArti}.

Además se ha decidido utilizar OpenCv, otra herramienta para procesar im\'agenes que se describe en la sección \ref{herramientasDetc}. Sin embargo los métodos de captura de OpenCV no funcionan con la c\'amara Raspberry Pi. En la secci\'on \ref{extraerImagen} se explica c\'omo se ha extra\'ido la imagen y en la secci\'on \ref{procesarImagen} se explica la forma en la que se ha procesado la imagen para hallar la posición de la pelota. 

\subsection{Herramientas software para la detecci\'on }\label{herramientasDetc}

Para extraer y procesar la imagen se utilizaron algunas librerias como apoyo. A continuación se presenta la descripción de la librería raspicam\_cv, usada para la extracción de la imagen y la descripción de la librería OpenCV, usada para la detección de la ubicación de la pelota.   

En visi\'on de computadoras existen varias liberias que apoyan y facilitan en procesamiento de im\'agenes, con herramientas de filtrado, transformaciones de im\'agenes, aprendizaje de m\'aquinas entre muchas m\'as. Dos ejemplos de ello son Processing y OpenCv. Processing es un lenguaje de programaci\'on que esta enfocado para iniciar a personas no programadoras en el \'area, por lo tanto explota la retroalimentaci\'on visual para atraer al usuario, posee herramientas de filtrado, de transformaci\'on de im\'agenes y muchas otras.
Por otro lado OpenCv (Open Source Computer Vision Library) es una librería de visión de computadoras y aprendizaje de máquinas de código abierto. Ha sido diseñada para acelerar el uso de la percepción de m\'aquinas y para proveer una estructura común en las aplicaciones de visión de computadoras \cite{opencv}. La decisi\'on de utilizar OpenCv se basa en su mayor amplitud de herramientas ofrecidas, un filtrado de im\'agenes mas amplio, un mayor control en el manejo de las im\'agenes y sus transformaciones.

Raspicam\_cv es una librería que permite obtener im\'agenes de la cámara Raspberry Pi en una estructura de datos compatible con OpenCV  modificada por \cite{emilV}.

\subsection{Obtenci\'on de la imagen}\label{extraerImagen}

Dentro de las librerías oficiales para la cámara Raspberry Pi s\'olo se encuentran implemetadas en el lenguaje interpretado python y algunas aplicaciones para la línea de comandos de linux. Para utilizar la cámara con OpenCV en el lenguaje compilado C++ se requirió realizar una búsqueda de librerias alternas a las oficiales. Una primera solución se encontró en el blog de \cite{pierreR}, en donde explica que los métodos de captura de video de OpenCV no funcionan de manera nativa con el m\'odulo de la cámara de la Raspberry Pi (por ejemplo el método cvCapture). Para lograr extraer la imagen se basó en el código abierto de las aplicaciones raspivid y raspistill. Ha modificado el código para usar el buffer de la cámara y así obtener un objeto compatible con OpenCV. 
raspicam\_cv es la librer\'ia utilizada en el proyecto que del c\'odigo ha sido moficicada convirtiendola una librer\'a por \cite{emilV}.

\subsection{Procesamiento de la imagen}\label{procesarImagen}

Con ayuda de la librería OpenCv, en C++, se filtra y procesa la imagen para obtener la posición de la pelota en un momento dado y de forma autónoma. 

Para encontrar la ubicación de la pelota  se ha decidido aplicar detección por segmentación de regiones, esta técnica consiste en filtrar la imagen por segmentaci\'on de color, por ello es importante que el color de la pelota no se repita en el ambiente y así poder obtener su posición dentro de la imagen. Esta t\'ecnica de segmentaci\'on es la m\'as com\'un en detecci\'on de objetos y muy utilizada en competencias de rob\'otica. La t\'ecnica de detecci\'on por  formas fue puesta a prueba tambi\'en pero con la misma ni se pudo obtener resultados pr\'acticos ya que no lograba detectar correctamente la pelota. Vale la pena acotar que la detecci\'on por formas aplicada no implementaba ning\'un tipo de aprendizaje de m\'aquina.

La imagen es captada en el modelo de color RGB y se transforma al HSV. Luego se aplica la función inRange de OpenCv para obtener una imagen en blanco y negro, en donde se identifica con blanco la zona con el color de la pelota y el resto de la imagen en negro. En el procesamiento de im\'agenes y visi\'on de computaci\'on se trabaja con las im\'genes en escala de grises pues disminuye el tiempo en procesar datos que poseen informaci\'on inutil y aumenta la eficiencia.

Para disminuir el ruido y los posibles elementos aislados que pueda tener la imagen con la que se está trabajando se han aplicado los filtros o transformaciones de morfología en apertura y morfología en clausura de la librería Opencv, basadas en las operaciones básicas de dilatación y erosión. La morfología en apertura es una transformación que consiste en aplicar la operación de erosión seguido de la operación de dilatación. La morfología de clausura es una transformación que aplica la dilatación seguido de la erosión.

De esta forma se logró ubicar la pelota con la cámara Raspberry Pi en 100\% de las pruebas realizadas.



\section{B\'usqueda y Pateo}\label{chapter:busqueda}

Para poder buscar y patear la pelota, además de tener la capacidad para detectarla (como se explicó en el capítulo anterior), debe ser capaz de moverse en su entorno, poder patear, levantarse, tener una representación del mundo que lo ayude a orientarse y tener una estrategia para elegir el conjunto de movimientos que lo lleven a acercarse a la pelota. 

En esta sección se explica el desarrollo de las actividades que han sido necesarias para ejecutar la búsqueda y pateo de la pelota.
%, con excepción de la estrategia para la toma de acciones, que se explicará en la sección \ref{aprendizaje}.

Primero, en la sección \ref{subsection:Herrbusqueda} se da una breve descripción de las herramientas de software que apoyaron las tareas de búsqueda y pateo. En la sección \ref{esqueleto} se explica cuál fue el conjunto de movimientos creados para el esqueleto del robot.

El sensor principal de Junny, es el observador de su ambiente, la c\'amara. \'Esta tiene dos grados de libertad para su movimiento, lo que causa que tenga un gran número de posibles posiciones. Para simplificar, se ha límitado la cantidad de posiciones. En la secci\'on \ref{movCamara} se explica las posiciones que puede adoptar la cámara. Luego en la secci\'on \ref{mundo} se explica la manera en la que Junny organiza la representaci\'on visual que capta del mundo.

Debido a que el movimiento del robot se controla desde la tarjeta Arbotix y la detecci\'on de la pelota se hace desde la Raspberry Pi, se debi\'o establecer la comunicaci\'on entre ambas tarjetas. Este proceso se explica en la secci\'on \ref{comunicacion}.

Una vez con la representaci\'on del mundo, los movimientos programados y la comunicaci\'on de las tarjetas, solo faltaría decidir que acci\'on tomar en cada situación. La estrategia, basada en el paradigma h\'ibrido, se explica en la secci\'on \ref{eleccionAccionesFijas}.        

%Como se mencion\'o anteriormente el robot debe buscar y patear una pelota de tamaño de una pelota de tennis y ya se ha descrito las caracter\'isticas f\'isicas y de software que utiliza el robot, esta secci\'n se enfoca en el comportamiento que describe a este robot.
%Para ello primero se especifica la serie de movimientos implementados y el comportamiento que lo representa.
 
\subsection{Herramientas software para la b\'usqueda de la pelota } \label{subsection:Herrbusqueda}

Para cumplir con el desarrollo de la programación de la búsqueda y pateo de la pelota, se han utilizado algunas herramientas que han facilitado el proceso. A continuación se describen las más destacadas dentro del proyecto. 

\begin{itemize}
\item Pypose: Software especializado en el control de los servomotores Dynamixel Ax-12. Una de las más importantes características es que, luego de haber fijado a mano las posiciones de los motores, permite la lectura simultánea de esas posiciones para captar la pose del robot. Con esta herramienta es posible formar una secuencia de poses que generen un movimiento, por ejemplo, caminar \cite{pypose}. 

\item IDE Arduino: Es un entorno de desarrollo para escribir y cargar código en la tarjeta Arduino. Otras tarjetas con microcontroladores AVR también son compatibles, como la Arbotix. El lenguaje de programación del IDE de Arduino es una implementación de Wiring el cual está basado en Processing \cite{arduino}.  


\item ROS: ROS (Robot Operating System) es un Framework flexible para la escritura de software enfocado en robots. Es una colección de herramientas, bibliotecas y convenciones que tienen por objeto simplificar la tarea de crear un comportamiento complejo y robusto a través de una amplia variedad de plataformas robóticas. ROS se encuentra bajo licencia de código abierto, la licencia BSD \cite{ros}.

\item HServo: Es una librería de código libre distribuida bajo la Licencia Pública General Reducida de GNU. Esta librería se encuentra desarrollada especificamente para la tarjeta Arbotix, ya que con la nueva librería de Arduino Servo se pueden experimentar fallas en el control de los servos. La interfaz es la misma, la diferencia es que solo puede ser usada para servos conectados en los pines 12 y 13 de la tarjeta Arbotix \cite{HServo}.    

\end{itemize}

%\subsection{Comportamiento}

%Debupa es un robot humanoide implemetado de forma autonoma e inteligente que sigue un comportamiento bajo el paradigma h\'ibrido (secci\'on 2 ). El sensor principal (c\'amara ) es el observador del mundo, que posee una serie de movimentos determinados (secci\'on REF) con los cuales escanea el mundo y combinados con la serie de movimiemtos del esqueleto es capaz de encontrar la pelota. Al determinar la posici\'on de la pelota Debupa logr\'o aprender (secci\'on APRENDIZAJE) la mejor acci\'on a relizar para estar mas cerca de ella y al llegar poder patearla.

\subsection{Movimiento del esqueleto}\label{esqueleto}

El robot debe tener la capacidad de ejecutar una variedad de movimientos para poder cumplir la meta de patear la pelota. Por ello se ha programado un conjunto de poses y transiciones o acciones de movimiento que se explican a continuación.

Con fines explicativos, en este proyecto, la palabra ``pose" se referire a la posición específica de los 16 motores que constituyen el esqueleto del robot. Un conjunto de poses ejecutadas en secuencia se denominará ``acción de movimiento".

Las acciones de movimiento establecidas son:

\begin{itemize}
% \item {Caminar 2.5 cm hacia adelante }
 \item {Caminar dos pasos hacia adelante (4.8 cm ) }
% \item {Caminar 9.9 cm hacia adelante }
 \item {Girar a la izquierda (3 cm)}
% \item {Girar 6 cm a la izquierda} 
 \item {Girar a la derecha (3 cm)}
% \item {Girar 6 cm a la derecha}	 
 \item {Levantarse desde la posición boca abajo}
 \item {Levantarse desde la posición boca arriba}
 \item {Patear con la pierna derecha }
 \item {Patear con la pierna izquierda}
 
\end{itemize}

Estas poses han sido fijadas a través de la tarjeta controladora Arbotix y el software Pypose. De esta manera se ha fijado y guardado un conjunto de poses para cada acción de movimiento. Estas acciones de movimientos han sido exportadas para ser utilizadas en el programa, en lenguaje Wiring, a ser ejecutado en Arbotix. La programación en Arbotix se ha realizado bajo el ambiente del IDE de Arduino. 


\subsection{Movimiento de la cámara}\label{movCamara}
La cámara ha sido instalada sobre dos micro servomotores analógicos, otorgándole dos grados de libertad. El servomotor ubicado en la parte inferior se encarga del movimiento horizontal y el superior, del movimiento vertical.

El hecho de  que la cámara tenga dos grados de libertad para moverse es una gran ventaja, ya que se puede obtener un mayor rango de visión. Junny puede mirar hacia la derecha o izquierda sin tener que mover sus piernas, también puede mirar hacia abajo para verificar que la pelota esté en sus pies, para patear, o hacia arriba para ubicar la pelota a mayor distancia.

El número de posibles posiciones para la cámara es muy amplio y para simplificar se ha reducido a 6 posiciones fijas, cuya distribución obedece al objetivo de que la cámara obtenga una amplia visión, sin dejar espacios no visibles. Esta simplificación ayuda en la tarea de la representación del mundo que se explica en la sección \ref{mundo}. En la figura~\ref{posicionesCam} se puede observar la cámara del robot en las diferentes posiciones que se han fijado para ella.  

\begin{figure}[hbtp]
\centering
\includegraphics[scale=0.5]{imagenes/Pantallazo.png}
\caption{Posiciones de la cámara }
\label{posicionesCam}
\end{figure}

Los micro servomotores azules que aparecen en la imagen \ref{posicionesCam} se controlan desde la Arbotix usando la librería HServo. Esta librería s\'olo puede ser usada para los motores conectados en los puertos Hobby A y B (pines 12 y 13) (ver la figura ~\ref{fig:arbotixConectados}). Tener los motores conectados a estos puertos brinda la ventaja de un control más preciso, evitando que los motores generen vibración, ya que los pulsos son generados por temporizadores de hardware. 

\subsection{Representaci\'on del mundo}\label{mundo}

La representación del mundo de Junny se basa en función de la posición de la pelota con respecto al él. 

La cámara tiene 6 (2x3) posibles posiciones, como se muestra en la figura ~\ref{posicionesCam} y desde cada posición de la cámara se obtiene una imagen. La detección de la pelota en alguna de esas imagenes brinda una orientación sobre la ubicación de la pelota. Si se detecta la pelota, por ejemplo, cuando la cámara se encuentra en alguna de las posiciones de la derecha, es un indicativo de que la la pelota se encuentra a la derecha y que si el robot gira se podría posicionar frente a ella. 

Se ha decidido discretizar las posibles posiciones de la pelota por regiones. Estas regiones se muestran enumeradas en la fugura ~\ref{divisionCam}. Cada cuadro blanco demarcado por líneas negras representa la imagen captada en una posición fija de la cámara. 

Las imagenes de la cámara en la posición central superior e inferior son las más importantes y prioritarias, pues si la pelota se detecta en ellas significa que el robot está cerca de poder patearla. Estas dos imagenes se dividen en subregiones para tener mayor precisión en las acciones que Junny deba tomar.

Cuando, por ejemplo, la pelota se encuentra del lado derecho en el cuadro central inferior (región 5) el robot debería girar a la derecha para situarse de frente a la pelota. El área de pateo se encuentra en las regiones 1 y 2.

Las imágenes capturadas desde cada posición de la cámara se solapan un poco para evitar perder de vista a la pelota. Este solapamiento no se muestra en la imagen ~\ref{divisionCam} para simplificar su representación.

%Las acciones específicas a tomar según la región en que se encuentre la pelota se seleccionan por medio de dos estrategias diferentes.

%lo aprendido en el entrenamiento realizado con apredizaje por reforzamiento. Los detalles de este aprendizaje se describen en la seccion \ref{aprendizaje}.

%Para poder ubicar la pelota, sin cambiar la ubicación del robot, se mueve la posición de la cámara hasta encontrarla. En caso de encontrarla, dependiendo de su ubicación dentro de la imagen y la posición de la cámara se toma una acción diferente, en caso de no encontrarla el robot gira con los pies para cambiar su orientación física e iniciar nuevamente el movimiento de la cámara para hallar la pelota. Cuando se tiene la pelota en una posición cercana a los pies se realiza la acción de patear.

%La visión horizontal abarca 3 cuadros, aproximadamente 160 grados, por razones de la estructura del robot no se le ha podido agregar un rango más grande. La visión vertical abarca 2 cuadros, llega a captar la imagen desde sus pies hasta más de 2 metros hacia adelante. 

%A continuación se especifica la acción a tomar en cada region: 
 
%Girar a la Izquierda: Debupa debe girar a la izquierda cuando la pelota se encuentre en alguna de las siguientes regiones: 1, 4, 10, 5, 11, 14.

%Girar a la Derecha: debe girar a la derecha cuando la pelota se ubique en alguna de las siguientes regiones: 7, 13, 17, 3, 9, 18. 

%Caminar hacia adelante: cuando la pelota se ubique en alguna de las siguientes regiones: 12, 6, 2.

%Patear con la pierna izquierda: cuando la pelota se encuentre en la región 15.

%Patear con la pierna derecha: cuando la pelota se encuentre en la región 16.

\begin{figure}[hbtp]

\centering
\includegraphics[scale=0.5]{imagenes/Regiones.jpg}
\caption{Campo de visión del robot con el número de cada región. Cada cuadro blanco demarcado con líneas negras representa la imagen captada desde una posición fija de la cámara. La región de pateo de la pelota se encuentra en las regiones 1 y 2 }
\label{divisionCam}
\end{figure}

\subsection{Comunicación Arbotix - Raspberry (ROS)}\label{comunicacion}

La Raspberry Pi procesa la información de la cámara y la Arbotix controla los actuadores. Para coordinar los movimientos del robot según la posición de la pelota se estableció una forma de comunicación entre ambas tarjetas. 

Se ha establecido la Arbotix como servidor de peticiones y a la Raspberry Pi como cliente. Dentro de la Raspberry Pi se ejecuta el proceso de decidir qué acción debe tomar el robot. Una vez determinada la acción se envía la petición a la Arbotix para que esta la ejecute. Este proceso es bidireccional y síncrono, es decir, la Raspberry envía la petición y se bloquea hasta que la Arbotix retorne la respuesta de su culminación.  

Para la implementación de la comunicación se ha usado ROS con su versión Hydro y se ha utilizado la interfaz de comunicación basada en servicios que no es más que un método de comunicación basado en el paradigma de resquest / reply con el concepto de maestro esclavo.

\subsection{Elecci\'on de la acci\'on}\label{eleccionAccionesFijas}

Una vez con la representaci\'on del mundo, los movimientos programados y la comunicaci\'on de las tarjetas, solo faltaría decidir que acci\'on tomar en cada situación. En primera instancia se ha decidido fijar una acción por cada región en la que se detecte la pelota. 

A continuación se especifica la acción a tomar en cada region: 
 
Girar a la Izquierda: Junny debe girar a la izquierda cuando la pelota se encuentre en alguna de las siguientes regiones: 4,8,10 y 12.

Girar a la Derecha: debe girar a la derecha cuando la pelota se ubique en alguna de las siguientes regiones: 5, 9, 11 y 13. 

Caminar hacia adelante: cuando la pelota se ubique en alguna de las siguientes regiones: 3, 6 y 7.

Patear con la pierna izquierda: cuando la pelota se encuentre en la región 1.

Patear con la pierna derecha: cuando la pelota se encuentre en la región 2.

En caso de no encontrar la pelota en ninguna de las regiones, el robot gira con los pies para cambiar su orientación física e iniciar nuevamente el movimiento de la cámara para hallar la pelota. Cuando se tiene la pelota en una posición cercana a los pies se realiza la acción de patear.



\section{Aprendizaje}\label{aprendizaje}
 
En el área de inteligencia artificial, asociada a robótica, existen variadas técnicas que permiten que un robot pueda aprender a realizar alguna tarea. En este caso particular se utiliz\'o la técnica de aprendizaje por reforzamiento que consiste en dar recompensas positivas o negativas dependiendo del desempeño del robot.

Según \cite{Mitchell} todo aprendizaje se define por la realización de una tarea $T$, cuyo desempeño medido por $D$, mejora con la experiencia $E$. En la investigación presente la tarea $T$ es aprender cuál es el mejor conjunto de ``acciones de movimiento" que se debe tomar con la finalidad de acercarse a la pelota. La experiencia $E$ se da a través un conjunto de pruebas, en las que Junny debe buscar la pelota y posicionarse frente a ella. El desempeño $D$ se mide con respecto a si el robot logra posicionar la pelota en el área de pateo y cuántas acciones de movimiento son necesarias para lograrlo.

Como existe incertidumbre con respecto a la dinámica del ambiente (esto es, no se conoce la función $\delta(s,a)$, definida en la subsección \ref{subsec:Qlearning} se utilizó el modelo de aprendizaje Q-learning, el cu\'al permite mejorar el desempeño de la tarea definida en base a la experiencia de cada prueba. A continuación se presenta y describe la configuración de las caracteristicas particulares del aprendizaje utilizado para este proyecto.

\subsection{Modelo del problema}

El algoritmo para Q-learning se adapta a problemas configurados como procesos de decisión Markovianos (MDP). En este tipo de problemas el resultado de aplicar una acción en un estado particular depende solo de ese estado y esa acción, no de las acciones del pasado. 

En un MDP, un agente representa su mundo a través de un conjunto de estados y tiene un conjunto de acciones que puede ejecutar. Cada vez que el agente identifica el estado en el que se encuentra y escoge una acción para tomar, dependiendo del resultado, este recibe una recompensa determinada.

En la sección \ref{estados} se define el conjunto de estados que conforman el mundo de Junny. En la secci\'on \ref{acciones} se dan a conocer las posibles acciones definidas y en la secci\'on \ref{recompensas} se define la ecuaci\'on para calcular las recompensas.   

\subsubsection{Estados}\label{estados}

De forma general se puede decir que los estados se encuentran definidos en función de la posición de la pelota con respecto al robot. En la sección \ref{mundo} se ha explicado la manera en la que se representa el mundo en base a 13 regiones en las que se puede encontrar la pelota. Para el aprendizaje se ha decidido ampliar el número de regiones para afinar y mejorar el tiempo en el que se llega a la pelota. Estas regiones se muestran el la figura ~\ref{fig:estados2}.

Cada región en la que se pueda detectar la pelota es un estado diferente. Por lo tanto se generan 18 estados, 17 de ellos se corresponden con la detección de la pelota en cada una de las regiones que se muestran en la imagen ~\ref{fig:estados2} y el estado número 18 representa la ocasión en la que no se logra ubicar la pelota en ninguna de las regiones.  

\begin{figure}[hbtp]
\centering
\includegraphics[scale=0.5]{imagenes/Regiones2.jpg}
\caption{Estados definidos seg\'un la región en la que se detecta la pelota}
\label{fig:estados2}
\end{figure}


\subsubsection{Acciones}\label{acciones}

Las posibles acciones a realizar son un subconjunto de las acciones de movimiento definidas en la sección \ref{esqueleto}. Además se agregaron otras acciones para mejorar el desempeño en la búsqueda de la pelota. Las acciones disponibles son:

\begin{itemize}
\item {Caminar un paso hacia adelante (2.5 cm)}
\item {Caminar dos pasos hacia adelante (4.8 cm)}
\item {Caminar cuatro pasos hacia adelante (9.9 cm)}
\item {Girar a la izquierda (3 cm)}
\item {Girar doble a la izquierda (6 cm)} 
\item {Girar a la derecha (3 cm) }
\item {Girar doble a la derecha (6 cm)}
\end{itemize}

\subsubsection{Recompensas}\label{recompensas}

La recompensa, que puede ser positiva o negativa, se otorga a un par $(s,a)$, en donde $s$ es un estado y $a$ es una acci\'on. Se calcula en base a que tanto el robot se acerca a la pelota cuando en el estado $s$ se toma la acci\'on $a$. Es decir, las recompensas se definen en base a la distancia recorrida, con respecto a la pelota, cuando se toma una acci\'on.  

La distancia de una región con respecto al área de pateo, se define con la función $d: r \rightarrow y$ con $r \in \{1,2,3 ...14\}$ y $y \in \{1,2,3 ...10\}$. En donde $r$ es el número de la región y $y$ es la distancia de la región con respecto al \'area de pateo. Se asign\'o una distancia a cada regi\'on como se muestra en la figura ~\ref{fig:distancias}. El valor 1 se le asigna a las regi\'ones m\'as cercanas al robot (zonas de pateo), $d(1)= 1$ y $d(2)=1$; el valor 10, a las regiones m\'as lejanas (regiones 12 y 14). Adicionalmente se define la distancia de la región 18 como $d(14)=10$. 
     
\begin{figure}[hbtp]
\centering
\includegraphics[scale=0.5]{imagenes/Distancias2.jpg}
\caption{Distancias relativas de la pelota establecidas para otorgar la recompensa.}
\label{fig:distancias}
\end{figure}

Junny gana una recompensa positiva cuando, con una acción, la pelota pasa de una región de mayor valor (lejana) a una de menor valor (más cercana). Si la pelota se mantiene en la misma región obtiene recompensa 0. De lo contrario obtiene una recompensa negativa. La ecuación se define como:
\[R(s,a,s') = \dfrac{d(s)-d(s')}{10} \]

El rango de valores para la recompensa se encuentra entre -1 y 1. 


\subsection{Eleci\'on de la acci\'on}

Se entiende que dado un estado $s$ se tienen 7 posibles acciones $\{a_1, a_2, ... a_7\}$ que, en el entrenamiento del robot, él aprende cuál es la mejor a realizar según el estado en el que se encuentra. A mayor valor de $Q(s,a)$ mejor es la acción $a$. Si siempre se toma el máximo valor se favorece la explotación. Si se toman acciones aleatorias se favorece la exploración. Para que el aprendizaje obtenga buenos resultados se debe tener un equilibrio entre la exploración y la explotación. 

Para variar entre la explotación y la exploración se utiliz\'o la función de probabilidad definida como  \[P(a_{i} | s) = \dfrac{k^{Q(s,a_{i})}}{\sum_{j}k^{Q(s,a_{j})}}  \] 

Con valores diferentes de $k$ para cada conjunto de pruebas, como se explica en el capítulo \ref{chapter:resultados}.  
 
 
\subsection{Actualizaci\'on de Q(s,a)}

La actualizaci\'on de Q(s,a) se define por la siguiente formula 
 
\[Q(s,a) = r + \gamma \max_{a'} Q(\delta(s,a),a')\]

Donde $r$ y $\max_{a'} Q(\delta(s,a),a')$ se explican en la subsecci\'on \ref{subsec:Qlearning} el factor $\gamma$ es el factor de descuento que varia de \[   0 \leq  \gamma < 1 \] este par\'ametro es arbitrario y se debe ajustar durante el aprendizaje, en el cap\'itulo \ref{chapter:resultados}  se presentan los $\gamma$ utilizados para los entrenamientos.

\section{Detecci\'on de ca\'idas}

Para poder detectar una posible caída del robot se cuenta con el giroscopio, incluido en el kit Bioloid de Robotis. Este giroscopio brinda información sobre la velocidad angular en los ejes $X$ y $Y$ como se muestra en la figura ~\ref{fig:gyroDireccion}.

\begin{figure}[hbtp]
\centering
\includegraphics[scale=0.5]{imagenes/gyroDireccion.jpg}
\caption{Dirección en la que la velocidad angular de los ejes $X$ y $Y$, del Gyro, aumenta.}
\label{fig:gyroDireccion}
\end{figure}

La información del eje utilizado en este proyecto es el $X$, que indica si el robot experimenta una velocidad angular hacia adelante o hacia atrás. En la figura ~\ref{fig:gyroDireccion1} se muestra la direcci\'on en la que crece la velocidad angular con respecto al robot. El giroscopio puede detectar la velocidad desde $-300^{\circ}/s$ hasta $300^{\circ} /s$. Según la documentación, los valores que se leen del Gyro se dan en un rango de 45 a 445, en donde el valor 45 representa los $-300^{\circ}/s$ y el 455 representa los $300^{\circ} /s$. El valor 250 indica que no hay movimiento angular, es decir $0^{\circ}/s$. %Conociendo esta información, se decidió graficar los valores leídos por el giroscopio para observar su comportamiento mientras el robot ejecutaba una serie de movimientos.

\begin{figure}[hbtp]
\centering
\includegraphics[scale=0.06]{imagenes/robotLateral.jpg}
\caption{Dirección en la que la velocidad angular del eje $X$ crece con respecto al robot.}
\label{fig:gyroDireccion1}
\end{figure}

%Se ha graficado los valores del giroscopio con el robot estático para comprobar que los valores leídos se corresponden con los de la documentación, es decir 250 cuando no hay movimiento. Cada punto representa un valor promediado de diez lecturas. La gráfica resultante es la que se puede apreciar en la figura ~\ref{fig:grafica1}. El primer valor promediado es 284 mientras que el último es 260. Por este motivo se decidió agregar un tiempo de 15 segundos del robot en reposo antes de comenzar la búsqueda de la pelota, de esta manera los valores del giroscopio se estabilizan. 

Para detectar una posible caída ha sido necesario verificar constantemente la velocidad angular del eje X. Se ha establecido un límite inferior y uno superior para identificar dentro de qué valores puede estar la velocidad angular sin que se considere como una caída del robot, sino como efecto de su movimiento. En caso de estar fuera de los límites, se considera como una caída y el robot ejecuta la acción de movimiento para levantarse.  

Se han observado los valores del giroscopio con el robot estático para verificar si los valores leídos se corresponden con los de la documentación, es decir 250 cuando no hay movimiento. Para obtener información más confiable se realiza un promedio entre diez lecturas para cada valor que se toma en cuenta. El resultado ha sido que los primeros valores no inician en 250 sino, en aproximadamente 284 y luego va disminuyendo. Por este motivo se decidió agregar un tiempo de 15 segundos del robot en reposo antes de comenzar la búsqueda de la pelota, de esta manera los valores del giroscopio se estabilizan. 

Después de la pausa de 15 segundos, el primer valor promediado se toma como el valor que representa la velocidad angular cero, a este se le llamará valor central. Los siguientes valores se comparan con el valor central. Si la diferencia se encuentra dentro del rango $(-80,100)$ significa que los movimientos no han sido tan bruscos como para considerarse una caída. De lo contrario, si se obtienen valores fuera del rango, se considera una caída y se ejecuta la acción de levantarse.   

%Una de las primeras gráficas se muestra en la figura X. Cada valor gráficado representa una lectura promediada de diez veces, mientras el robot ejecuta el movimiento de caminar repetidas veces. Se puede observar que los primeros valores se encuentran entre 300 y 350, luego los valores disminuyen en conjunto.  

\begin{figure}[hbtp]
\centering
\includegraphics[scale=0.4]{imagenes/grafica1.jpg}
\caption{Gráfica de lecturas de velocidad angular en el eje X.}
\label{fig:grafica1}
\end{figure} 
\section{Orientaci\'on al arco}


Finalmente se procedi\'o a realizar la busqueda del arco, el procedimiento es completamente an\'alogo a el procedimiento de busqueda de la pelota, con la diferencia que la representaci\'on del mundo difiere un poco y se presenta en la figura TAL, en esta ocasi\'on se cuentan con 7 estados y el mismo conjunto de acciones que se presentaron en la seccion TAL.

El procedimiento integrado es el siguiente, detecci\'on y busqueda de la pelota, entonces se detecta y coloca al arco intentando rodear la pelota, por \'ultimo se verifica que a\'un la pelota se encuentre en la zona de pateo y se procede a patear hacia el arco.

Los resultados obtenidos en esta tarea en particular no han sido muy favorables, pues en casos directos logra ajustarse con la pelota para realizar un buen tiro, siembargo para el rodeo de la pelota es importante poseer un grado de libertad que Junny no posee lo cual dificultas realizar esta tarea de la forma optima.







\chapter{Experimentos y Resultados}\label{chapter:resultados}
En este capítulo se describen los experimentos realizados para analizar el desempeño del robot. Los experimentos se han dividido en tres partes. El primer conjunto de experimentos (sección \ref{sec:experimentosMov}) se ha realizado para verificar el desempeño y balance del robot al ejecutar uno o varios movimientos en secuencia (experimentos simples). El segundo y tercer conjunto de experimentos se bas\'o en verificar el desempeño del robot al buscar la pelota. En el segundo conjunto (sección \ref{sec:experimentosintegrados}) la forma de escoger la acciónes de movimiento han sido fijadas para cada región en la que se detecta la pelota. Mientras que para el tercer conjunto de experimentos (secci\'on \ref{sec:experimentosAprend}) la forma de escoger las acciones de movimiento ha sido resultado del aprendizaje por reforzamiento. Finalmente se describe los experimentos con el entrenamiento terminado y la orientaci\'on al arco (experimentos completos) en la secci\'on ~\ref{completos}. 

Algunos videos de los experimentos se encuentran disponibles en: \url { https://www.dropbox.com/sh/b1qxstva3o2hmu8/AAAbLx5ztJBwGylXtVWWbfJva?dl=0}
\section{Experimentos de Movimientos}\label{sec:experimentosMov}

Se realizaron una serie de experimentos para comprobar la movilidad del robot en cuanto a un conjunto de acciones de movimiento (mencionadas en la sección \ref{esqueleto}). Estas acciones son:
\begin{enumerate}

\setlength{\itemsep}{0.5pt}
\item Caminar un paso hacia adelante 
\item Caminar dos pasos hacia adelante 
\item Caminar cuatro pasos hacia adelante 
\item Girar hacia la derecha 
\item Girar doble a la derecha 
\item Girar hacia la izquierda 
\item Girar doble a la izquierda
\item Patada con la pierna derecha 
\item Patada con la pierna izquierda

\end{enumerate}
La primera etapa consistió en verificar el desempeño de las 9 unidades de acciones, para ello se procedió a realizar la ejecución de cada unidad 50 veces, 450 ejecuciones en total. Se tomó nota de las veces que lograba realizar el movimiento sin fallas, es decir, de forma exitosa y de aquellas que lograba completar el movimiento pero con fallas. Los resultados obtenidos fueron un 99.6\% de casos exitosos y un 0.4\% de casos en los que pudo recuperarse de fallas. El 0.4\% representa un solo caso en el que el robot se ha caído y levantado. Los resultados se muestran en el cuadro \ref{fig:individuales}.  

\begin{table}
\centering
\begin{tabular}{|c|c|c|c|c|}
\hline 
Movimiento & Cantidad de pruebas & Correctas & Fallidas & Con recuperaci\'on \\ 
\hline 
1 & 50 & 100\% & 0\% & NA \\ 
\hline 
2 & 50 & 98\% & 2\% & 0\% \\ 
\hline 
3 & 50 &  100\% & 0\% & NA \\ 
\hline 
4 & 50 &  100\% & 0\% & NA \\ 
\hline 
5 & 50 &  100\% & 0\% & NA \\ 
\hline 
6 & 50 &  100\% & 0\% & NA \\ 
\hline 
7 & 50 &  100\% & 0\% & NA \\ 
\hline 
8 & 50 &   98\% & 2\% & 2\% \\ 
\hline 
9 & 50 &  100\% & 0\% & NA \\ 
\hline 
\end{tabular} 
\caption{Resultados pruebas individuales de movimiento}
\label{fig:individuales}
\end{table}
Para la segunda etapa se evalu\'o la ejecuci\'on de combinaciones de una serie de unidades de movimiento. Las posibles combinaciones que se pueden generar con las acciones es $9 ^ 2$ por lo cual se consideró s\'olo un subconjunto de estas combinaciones, aquellas que el robot realiza con más frecuencia al caminar. Las combinaciones elegidas fueron: 

\begin{enumerate}
\setlength{\itemsep}{0.5pt}
\item Caminar hacia adelante y patear con la pierna derecha 
\item Caminar hacia adelante y patear con la pierna izquierda
\item Caminar hacia adelante, girar hacia la derecha y patear con la pierna derecha
\item Caminar hacia adelante, girar hacia la derecha, patear con la pierna izquierda 
\item Caminar hacia adelante, girar hacia la izquierda y patear con la derecha
\item Caminar hacia adelante, girar hacia la  izquierda y  patear con la pierna izquierda
\item Girar hacia la izquierda y patear con la pierna derecha
\item Girar hacia la derecha y patear con la izquierda
\item Girar hacia la derecha y patear con la pierna derecha
\item Girar hacia la izquierda y patear con la pierna izquierda

\end{enumerate}
Se realizaron 30 ejecuciones de cada una de estas combinaciones, 300 ejecuciones en total. El resultado ha sido satisfactorio, obteniendo un 99\% de casos exitosos y un 1\% de casos con fallas, de las que ha podido recuperarse. La mayoría de las fallas se presentaron cuando al patear se caía, sin embargo lograba recuperarse. Los resultados de estas pruebas se resumen en el cuadro ~\ref{fig:combinadas}.
 
\begin{table}
\centering
\begin{tabular}{|c|c|c|c|c|}
\hline 
Combinaci\'on & Cantidad de pruebas & Correctas & Fallidas & Con recuperaci\'on \\ 
\hline 
1 & 30 & 96.7\% & 3.3\% & 3.3\% \\ 
\hline 
2 & 30 & 100\% & 0\% & NA \\ 
\hline 
3 & 30 & 100\% & 0\% & NA \\ 
\hline 
4 & 30 & 100\% & 0\% & NA \\ 
\hline 
5 & 30 & 96.7\% & 3.3\% & 3.3\% \\ 
\hline 
6 & 30 & 100\% & 0\% & NA \\ 
\hline 
7 & 30 & 100\% & 0\% & NA \\ 
\hline 
8 & 30 & 100\% & 0\% & NA \\ 
\hline 
9 & 30 & 100\% & 0\% & NA \\ 
\hline 
10 & 30 & 100\% & 0\% & NA \\ 
\hline 
\end{tabular} 
\caption{Resultados de los experimentos con las acciones combinadas elegidas}
\label{fig:combinadas}
\end{table}

%\begin{figure}[th!]
%\begin{tikzpicture}
%\pie[pos={10 ,0}, radius =2, explode =0.3, text = legend, rotate =180]{96/Exitosas, 4/Con fallas corregidas}
%\end{tikzpicture}
%\caption{Segunda etapa: 300 ejecuciones de acciones combinadas}
%\label{fig:etp2}
%\end{figure}
Para determinar la precisi\'on y desempe\~no de la detecc\'ion de la pelota se realizaron experimentos de enfoque r\'apido los cuales consistieron en la colocaci\'on de la pelota a una distancia fija, mover la c\'amara rapidamente y regresar a la posici\'on y observar si ubicaba la pelota correctamente, las distancias utilizadas fueron 50cm y 80cm del robot. Adem\'as para determinar posibles Falsos Positivos, es decir, detectar una pelota inexistente, se realiz\'o el mismo experimento pero sin la pelota. Los resultados obtenidos se presentan en el cuadro ~\ref{fig:deteccion}.

\begin{table}
\centering
\begin{tabular}{|c|c|c|}
 \hline 
  & Cantidad de Pruebas & Correctas \\ 
 \hline 
 A 50cm & 50 & 100\% \\ 
 \hline 
 A 80 cm & 50 & 100\% \\ 
 \hline 
 No pelota & 50 & 100\% \\ 
  \hline 

 \end{tabular}  
\caption{Resultados de las pruebas de detecci\'on}
\label{fig:deteccion} 

\end{table}

\section{Experimentos con Comportamientos Integrados}
\label{sec:experimentosintegrados}

El segundo conjunto de experimentos consistió en la realización de pruebas de desempeño en donde se observó el comportamiento global del robot con todos sus comportamientos integrados (detección, búsqueda y pateo). La forma de elegir la acciones de movimiento es la que se describe en la sección \ref{eleccionAccionesFijas} (sin aprendizaje). 

El  primer caso (CasoI) consistió en la colocación de la pelota en la zona de pateo (las regiones 1 y 2 de la figura ~\ref{divisionCam}). El número de pruebas en este primer caso fue de 10, con una duración aproximada de 30 segundos cada una.

%\begin{figure}
%\begin{tikzpicture}
%\pie[pos={10 ,0}, radius =2, explode =0.3, text = legend, rotate =180]{90/Pateó, 10/Falló}
%\end{tikzpicture}
%\caption{Pruebas de movimientos integrados CASO I}
%\label{fig:caso1}
%\end{figure}

El segundo caso (CasoII) consistió en en la colocación inicial de la pelota a una distancia de 50 cm en línea recta al robot. Se realizaron 50 pruebas, siendo la duración promedio de 2 minutos con 12 segundos.

%\begin{figure}
%\begin{tikzpicture}
%\pie[pos={10 ,0}, radius =2, explode =0.3, text = legend, rotate =180]{70/Exitosas, 10/Con fallas corregidas, 20/Fallidas}
%\end{tikzpicture}
%\caption{Pruebas de movimientos integrados CASO II}
%\label{fig:caso2}
%\end{figure}

El tercer caso  (CasoIII) consistió en en la colocación inicial de la pelota a una distancia de 50 cm en línea recta al robot y 50 cm a la izquierda formando asi una diagonal. Se realizaron 10 pruebas, siendo la duración promedio de 3 minutos con 29 segundos. 

%\begin{figure}
%\begin{tikzpicture}
%\pie[pos={10 ,0}, radius =2, explode =0.3, text = legend, rotate =180]{80/Exitosas, 20/Fallidas}
%\end{tikzpicture}
%\caption{Pruebas de movimientos integrados CASO III}
%\label{fig:caso3}
%\end{figure}

El cuarto caso (Caso IV) consistió en en la colocación inicial de la pelota a una distancia de 50 cm en línea recta al robot y 50 cm a la derecha. Se realizaron 10 pruebas, siendo la duración promedio de 4 minutos con 48 segundos. 

%\begin{figure}
%\begin{tikzpicture}
%\pie[pos={10 ,0}, radius =2, explode =0.3, text = legend, rotate =180]{90/Exitosas, 10/Fallidas}
%\end{tikzpicture}
%\caption{Pruebas de movimientos integrados CASO IV}
%\label{fig:caso4}
%\end{figure} 
\begin{table}
\centering

\begin{tabular}{|p{3cm}|p{2cm}|c|p{2cm}|c|p{2cm}|}
\hline 
& Cantidad de 
pruebas realizadas & Correctas & Con fallas recuperadas & Fallidas & Tiempo Promedio \\ 
%\hline 
%Subconjunto de combinadas & 300 & 96\% & 4\% & NA & NA \\ 
\hline 
Caso I & 10 & 90\% & 10\% & 0\% & 30 s \\ 
\hline 
Caso II & 10 & 70\% & 10\% & 20\% & 2m 12s \\ 
\hline 
Caso III & 10 & 80\% & 0\% & 20\% & 3m 29s \\ 
\hline 
Caso IV & 10 & 90\% & 0\% & 10\% & 4m 48s \\ 
\hline 
\end{tabular} 

\caption{Resultados de los casos I, II, III, IV descritos en la secci\'on de \ref{sec:experimentosintegrados}}
\label{fig:casos}

\end{table}

Los resultados de los casos del I al IV se pueden observar en el cuadro ~\ref{fig:casos}. Con un total de 40 pruebas de comportamiento integrado, los resultados obtenidos han sido satifactorios con un porcentaje de logro del 82\% más un 5.5\% en el cual se logró recuperar y finalizar la tarea, esto a pesar del 12.5\% de fallas que se produjo durante la tarea. %se pudo observar de manera general que la razón de dichas fallas ha sido ocasionada por problemas de batería.



\section{Experimentos con Aprendizaje}\label{sec:experimentosAprend}

Finalmente, el tercer conjunto de experimentos consistió en verificar los resultados de la aplicaci\'on del apredizaje por reforzamiento para la búsqueda de la pelota. Estos resultados se presentan a continuaci\'on.

Dentro de las fórmulas para el aprendizaje-Q se encuentran dos constantes que han sido configuradas de diferentes maneras para analizar cuál de las combinaciones brinda mejores resultados. Las constantes son la tasa de descuento $\gamma$ de la fórmula del aprendizaje-Q y la constante $K$, de la fórmula de probabilidad explicada en la secci\'on \ref{subsec:eleccionAccion}, que ajusta la distribución de probabilidad que se le da a una acción dependiendo de su valor $Q(s,a)$. Se utilizaron las siguientes combinaciones para las constantes:

\begin{itemize}
 \setlength\itemsep{0.3pt}
\item $K$ = 1 , $\gamma$ = 0.1
\item $K$ = 2 , $\gamma$ = 0.1 
\item $K$ = 2 , $\gamma$ = 0.7
\item $K$ = 3 , $\gamma$ = 0.1
\item $K$ = 3 , $\gamma$ = 0.7
\item $K$ = 5 , $\gamma$ = 0.1
\item $K$ = 5 , $\gamma$ = 0.7
   
\end{itemize}

Una prueba completa o prueba de comportamiento integrado se define como la colocaci\'on del robot y la pelota en posiciones iniciales arbitrarias, entonces el robot debe ser capaz de detectar la pelota, trasladarse hacia ella y patearla. Las pruebas se realizaron en un espacio de $1.8 m \times 1.6 m$
en donde se aseguraba que el color particular de la pelota no se repitiera, de ser así esa prueba se consideraba inválida.

Cada una de estas combinaciones se entrenó con 20 ejecuciones completas. Con los valores $Q(s,a)$ para cada estado y acción inicializados en cero (0), exceptuando los estados de la zona de pateo, que han sido inicializados con el valor 1. Una vez completados los entrenamientos se realizó un conjunto de pruebas para medir el desempeño del resultado de cada combinación y reconocer la mejor usando el porcentaje de pateos efectivos. Se realizó una prueba, de 10 ejecuciones cada una, para cada combinación. En total se realizaron 210 ejecuciones. La elecci\'on de s\'olo 20 pruebas para el conjunto de entrenamiento se deb\'io a que los motores son delicados, por lo cual mientras se aumentaba el n\'umero de pruebas aumentaba el riesgo de quemarlos.

Para la combinaci\'on de $K = 1$ y $ \gamma = 0.1 $ los resultados obtenidos se presentan en el cuadro ~\ref{tabla:entramientos}. Cuando $K = 1$ la probabilidad de elegir la acci\'on, dado un estado, es igual para todas las acciones, por lo tanto la elección de la acción para este caso ha sido uniformemente aleatoria. Los resultados de las combinaciones de par\'ametros ajustables se presentan en el cuadro ~\ref{tabla:entramientos}.
\begin{table}
\centering
\begin{tabular}{|c|c|c|c|}
\hline 
& Cantidad & Correctas & Fallidas \\ 
\hline 
$K = 1  \gamma = 0.1$ & 20 & 80\% & 20\% \\ 
\hline 
$K = 2  \gamma = 0.1$ & 20 & 80\% & 20\% \\ 
\hline 
$K = 2  \gamma = 0.7$ & 20 & 60\% & 40\% \\ 
\hline 
$K = 3  \gamma = 0.1$ & 20 & 90\% & 10\% \\ 
\hline 
$K = 3  \gamma = 0.7$ & 20 & 80\% & 20\% \\ 
\hline 
$K = 5  \gamma = 0.1$ & 20 & 80\% & 20\% \\ 
\hline  
$K = 5  \gamma = 0.7$ & 20 & 70\% & 30\% \\ 
\hline 

\end{tabular} 

\caption{Resultados de los distintos par\'ametros con aprendizaje}
\label{tabla:entramientos}


\end{table}
%
%\begin{figure}
%\begin{tikzpicture}
%\pie[pos={10 ,0}, radius =2, explode =0.3, text = legend, rotate =180]{80/Exitosas, 20/Fallidas}
%\end{tikzpicture}
%\caption{Prueba con  $K = 1$ y $ \gamma = 0.1 $}
%\label{fig:k1y1}
%\end{figure} 
%
%\begin{figure}[h]
%\begin{tikzpicture}
%\pie[pos={10 ,0}, radius =2, explode =0.3, text = legend, rotate =180]{80/Exitosas, 20/Fallidas}
%\end{tikzpicture}
%\caption{Prueba con $K = 2$ y $ \gamma = 0.1 $}
%\label{fig:k2y1}
%\end{figure} 
%
%\begin{figure}[h]
%\begin{tikzpicture}
%\pie[pos={10 ,0}, radius =2, explode =0.3, text = legend, rotate =180]{60/Exitosas, 40/Fallidas}
%\end{tikzpicture}
%\caption{Prueba  con $K = 2$ y $ \gamma = 0.7 $}
%\label{fig:k2y7}
%\end{figure} 
%
%\begin{figure}[h]
%\begin{tikzpicture}
%\pie[pos={10 ,0}, radius =2, explode =0.3, text = legend, rotate =180]{90/Exitosas, 10/Fallidas}
%\end{tikzpicture}
%\caption{Prueba con $K = 3$ y $ \gamma = 0.1 $}
%\label{fig:k3y1}
%\end{figure} 
%
%\begin{figure}[h]
%\begin{tikzpicture}
%\pie[pos={10 ,0}, radius =2, explode =0.3, text = legend, rotate =180]{80/Exitosas, 20/Fallidas}
%\end{tikzpicture}
%\caption{Prueba con $K = 3$ y $ \gamma = 0.7 $}
%\label{fig:k3y7}
%\end{figure} 
%
%\begin{figure}[h]
%\begin{tikzpicture}
%\pie[pos={10 ,0}, radius =2, explode =0.3, text = legend, rotate =180]{80/Exitosas, 20/Fallidas}
%\end{tikzpicture}
%\caption{Prueba con $K = 5$ y $ \gamma = 0.1 $}
%\label{fig:k5y1}
%\end{figure} 
%
%\begin{figure}[h]
%\begin{tikzpicture}
%\pie[pos={10 ,0}, radius =2, explode =0.3, text = legend, rotate =180]{70/Exitosas, 30/Fallidas}
%\end{tikzpicture}
%\caption{Prueba con $K = 5$ y $ \gamma = 0.7 $}
%\label{fig:k5y7}
%\end{figure} 

Los resultados de los entrenamientos arrojaron que la mejor combinaci\'on fue $K = 3$ y $ \gamma = 0.1 $ que obtuvo un desempeño favorable del 90\% de las pruebas realizadas.

Con el objetivo de lograr un mejor desempeño de Junny, se inicializ\'o los valores de $Q(s,a)$ con valores de recompesas negativas para aquellas acciones cuyos efectos, en un estado particular, serían obviamente err\'oneos. Con esto se ayuda al aprendizaje del robot, disminuyendo la probabilidad de tomar acciones ineficientes. Adem\'as se aumentó el n\'umero de pruebas de entrenamiento a 30 con el mismo objetivo.

Para medir el desempeño general de Junny, con respecto al último entrenamiento, se realizó una prueba de 15 ejecuciones colocando la pelota a distintas distancias. Los resultados de esta \'ultima prueba  son de 100\% de éxitos con $K = 3$ y $ \gamma = 0.1 $ e inicializaci\'on de Q(s,a). %se presenta en el gr\'afico \ref{fig:mejor}. 
%
%\begin{figure}[h]
%\begin{tikzpicture}
%\pie[pos={10 ,0}, radius =2, explode =0.3, text = legend, rotate =180]{100/Exitosas, 0/Fallidas}
%\end{tikzpicture}
%\caption{Prueba con $K = 3$ y $ \gamma = 0.1 $ e inicializaci\'on de Q(s,a)}
%\label{fig:mejor}
%\end{figure} 

Para un análisis más profundo de los resultados, se estableci\'o el n\'umero estimado de acciones esperadas, dado un estado inicial, que debe realizar el robot para llegar hasta la pelota. Con esto se puede tener una referencia de cuantas acciones debió realizar Junny en cada ejecución y comparar con el número de acciones que realmente tomó. La manera de calcular la eficiencia ha sido dividiendo la cantidad de acciones esperadas entre la cantidad de acciones realizadas. Por lo tanto mientras mas cercano a uno (1) se considera una mejor tasa de eficiencia. Para valores mayores a uno (1) significa que se realizaron menos acciones de lo esperado. Esto es posible ya que el número de acciones esperadas se calcula en base al peor de los casos.  

Para la última prueba se obtuvo una eficiencia de $0.72$. Sin embargo bajo los mismos parametros de $K = 3$ y $ \gamma = 0.1 $ con el resultado del entrenamiento con 20 pruebas (y valores de $Q(s,a)$ inicializados en cero) se obtuvo una eficiencia de $0.64$. Por lo tanto con el aumento de 10 pruebas de entrenamiento e inicialización de los valores $Q(s,a)$ se mejoró aproximadamente un 12\%. Este es un indicativo de que con mayor cantidad de pruebas este porcentaje mejorar\'a y la eficiencia aumentará.

%Así mismo al comparar los tiempos promedios, con respecto a la última prueba, se obtienen los siguientes resultados. El tiempo promedio de las pruebas cuyas acciones esperadas eran menor a 8 (estados mas cercanos) fue de 4m y 19s en contraste con el promedio del tiempo de las pruebas cuyas acciones esperadas eran mayores a 8 que fue 3m y 21s.

Se realiz\'o un experimento comparativo para observar el desempe\~no del comportamiento sin aprendizaje y con aprendizaje, por ello, se hizo el mismo experimento de los casos II, III, IV (explicados en la sección anterior) pero con el comportamiento generado después del entrenamiento con la combinación $K = 3 $ y $ \gamma = 0.1$. Los resultados obtenidos fueron un mayor porcentaje de aciertos para el aprendizaje pero a un costo de tiempo que se puede observar en el cuadro ~\ref{tabla:comparacion}, cada prueba en ambos casos ejecut\'o 10 veces.

\begin{table}
\centering
\begin{tabular}{|c|c|c|c|}
\hline  & Caso II & Caso III & Caso IV \\ 
\hline 
Correctas con aprendizaje & 100\% & 100\% & 100\% \\ 
\hline 
Correctas sin aprendizaje & 70\% & 80\% & 90\% \\ 
\hline 
Fallidas con aprendizaje & 0\% & 0\% & 0\% \\ 
\hline 
Fallidas sin aprendizaje & 20\% & 20\% & 10\% \\ 
\hline 
Recuperadas con aprendizaje & NA & NA & NA \\ 
\hline 
Recuperadas sin aprendizaje & 10\% & 0\% & 0\% \\ 
\hline 
Tiempo promedio con aprendizaje & 2m 35 s & 5m 38s & 4m 14s \\ 
\hline 
Tiempo promedio sin aprendizaje & 2m 12s & 3m 29 s & 4m 48s \\
\hline
%\caption{Cada prueba tiene un numero de corridas de 10}
\end{tabular} 
\caption{Comparaci\'on de movimientos predeterminados y aprendizaje}
\label{tabla:comparacion}

\end{table}

 Un resultado a destacar es que el tiempo promedio de pruebas a la derecha (Caso IV) es en proporci\'on  menor a las pruebas a la izquierda (Caso III) a pesar que las mismas son sim\'etricas. Esto indica que los siguientes entrenamientos deben hacer \'enfasis en los estados a la izquierda del robot.
 
\section{Experimentos completos} \label{completos}

Finalmente se realizaron experimentos con aprendizaje y patada con orientaci\'on, basicamente el robot part\'ia de un punto fijo y la pelota era colocada arbitrareamente en el campo de entrenamiento, el deb\'ia llegar a la pelota, ubicar el arco, posicionarse con respecto al arco y la pelota para realizar un pateo y meter gol.
Las m\'etricas usadas fueron el pateo hacia el arco que generaba un gol y el que pateaba pero no lograba entrar el bal\'on. Estos resultados se muestran en la figura \ref{fig:orientacion}.


\begin{figure}[h]
\centering
\begin{tikzpicture}
\pie[pos={10 ,0}, radius =1.5, explode =0.3, text = legend, rotate =180]{53.33/Goles, 6.666/TiroAlArco , 40/Fallos}
\end{tikzpicture}
\caption{Prueba con 15 ejecuciones, con aprendizaje y orientaci\'on al arco }
\label{fig:orientacion}
\end{figure} 




\chapter{Conclusiones y recomendaciones} 

\label{chap:conclusiones}

Logrando los objetivos planteados, se consiguió diseñar e implantar primero un 
lenguaje que facilitará la especificación de los problemas; y luego la 
herramienta en si (Parser, compilador e instanciador)
que permite generar instancias aleatorias, de objetos descritos como entradas,
cumpliendo las especificaciones de los mismos, así como también las 
restricciones que contienen. A partir de esta herramienta se lograron obtener
soluciones a los problemas planteados para pruebas. Se pudieron obtener
grupos de una, varias o incluso todas las soluciones posibles para las pruebas.

El lenguaje diseñado es cómodo y bastante expresivo en comparación con la
mayoría de las librerias de \textbf{CSP} o con la implementación de un algoritmo
procedural en un entorno imperativo, por lo que resulta fácil
acostumbrarse a su sintaxis. Además el diseño del lenguaje permite añadir 
nuevas funcionalidades sin tener que hacer demasiados cambios para que
se mantenga funcional.

El tiempo de ejecución de la herramienta fue para todos los ejemplos bastante
breve, en donde la escritura en disco fue la causa de la mayor demora en la 
corrida.
 
Además, se lograron también buenos resultados al interpretar las respuestas como
objetos finales. Para esto se probaron casos que abarcan una gran cantidad de
áreas de la computación, desde las más abstractas como las bases de datos hasta
las más ilustrativas como la computación gráfica.

Se analizaron las principales formas para optimizar la herramienta, se 
implementaron las que involucran la intervención y se especificaron
las automáticas para ser implementadas en trabajos futuros.

Ahora bien, la herramienta es funcional y cumple con las expectativas
iniciales, pero quedan muchas funcionalidades adicionales por ser
implementadas. Para eso, a continuación se proporcionará una lista de recomendaciones 
para posibles trabajos futuros que re-implementen o continúen
con este ambicioso proyecto:

\begin{itemize}
\item{Utilizar un lenguaje que permita modificar fácilmente la estructura
del lenguaje diseñado y de la herramienta: Esto permitirá que se puedan integrar
nuevas funciones rápidamente.}

\item{Implementar el funcionamiento de las listas: Con esto se reduciría
en gran cantidad la cantidad de variables de igual comportamiento que se
deben especificar como entrada.}

\item{Implementar el funcionamiento de las funciones: Con esto se pueden
aplicar restricciones mucho más complejas y que permitan expresar mejor
algunos problemas.}

\item{Añadir la posibilidad de recibir parámetros a la corrida de la 
herramienta: Con esto podrían pedirse respuestas para objetos similares pero
con diferencias sin tener que reescribir la entrada por completo.}

\item{Implementar la posibilidad de describir restricciones mediante
disyunciones: Con esto se aumentan las posibles descripciones de los
objetos pero significa aumentar en gran medida la complejidad del proceso de
resolución.}

\item{Implementar el sistema de decisión de método de resolución para
los subsistemas: actualmente esto no representa una mejora significativa, pero
considerando la implementación de las demás recomendaciones, este se vuelve
fundamental para mantener la herramienta eficiente.}

\item{Mejorar el motor de resolución de sistemas en \textit{Prolog} o substituirlo
con uno más complejo y eficiente. Adicionalmente, implementar otros métodos de
resolución, en especial el de tipo numérico, o bien crear un motor de 
resolución dedicado a este tipo de problemas.}
\end{itemize}

Luego de finalizado el proyecto se descubrieron una gran cantidad de posibles
usos para la herramienta. Estos seguramente, son sólo una minúscula
fracción de las posibilidades que se pueden generar con este proyecto. Cada una
de estas son en realidad una gran cantidad de soluciones a un mismo problema.
Sin tener que crear un programa específico sino un pequeño pero representativo
modelo de lo que se quiere.

Se tiene la seguridad de que este trabajo puede ser la piedra angular de nuevas 
formas de diseño y generación de contenido. Que sean accesibles a personas
de múltiples especialidades y que resuelvan cualquier cantidad de problemas.
Faltan aún muchas mejoras para poder catalogar la herramienta como un producto
finalizado y mucho menos un producto comercial. Aún así se tiene la expectativa
de que este proyecto será continuado y mejorado. ¿Quién sabe? Quizá sirva
como base para lo que en un futuro conozcamos como creatividad artificial...


% Crea el glosario
%\makeglossaries
%\makenomenclature
% Incluye el glosario
%\section{Glosario de t\'erminos}
\begin{tabular}{r p{16cm}l }
    Framework & \emph{Marco de trabajo}. Es un conjunto de técnicas, conceptos y estilos de trabajo que se establecen para resolver un problema particular y que sirve de referencia para solucionar problemas similares.\\

    TightVNC & Es un paquete de software que sirve para controlar la interfaz gráfica  de ordenadores remotos.\\

    AVR & Es una familia de microcontroladores de instrucciones reducidas de la compañía Atmel.\\

    IDE & \emph{Integrated development environment / Entorno de desarrollo integrado}. Es un programa diseñado para facilitar la programación en uno o varios lenguajes. Usualmente incluye herramientas de compilación, editor de textos y depurador.\\

    ROS & \emph{Robot Operating System / Sistema de operación para robots}. Es un framework que provee herramientas para ayudar a desarrolladores de aplicaciones para robots.\\

    Licencia BSD & \emph{Berkeley Software Distribution / distribución de software berkeley}. Es una licencia para software libre otorgada principalmente a sistemas BSD.\\
    
    XBEE & Es una familia de módulos de radio, con protocolo de comunicación inalámbrica basado en radio frecuencias.\\
    
    CSI & \emph{Camera Serial Interface / Interfaz serial para cámaras }. Es un estándar que define la interfaz de comunicación entre una cámara y un procesador. Es comúnmente utilizado en dispositivos móviles.\\
    
   MMAL & \emph{(Multi-Media Abstraction Layer / Capa de abstracción multimedia}: Es una librería que brinda una interfaz de bajo nivel para controlar dispositivos que se ejecutan en el núcleo de video de la Raspberry Pi, como el módulo de cámara.\\
   
   V4L & \emph{Video 4 Linux / video para linux}. Es una interfaz de programación de video para Linux. Algunos dispositivos soportados son cámaras web USB. \\
   
   RGB & \emph{Red Green Blue / rojo, verde, azul}: Es un modelo de color que se basa en la intensidad de los colores primarios de la luz (rojo, verde y azul).\\
   
   HSV & \emph{Hue, Saturation, Value / Matiz, Saturación, Valor}. Es un modelo de color que se basa en las cualidades de matiz, saturación y valor del color. \\
   
	HDMI & \emph{High-Definition Multimedia Interface/ interfaz multimedia de alta definición}. Es una interfaz para transferir datos de audio y video digital entre un dispositivo y un monitor, proyector, televisor digital o dispositivo de audio digital. \\
	
	SD & \emph{Secure Digital / Digital Seguro}. Es un formato de tarjetas de memoria de almacenamiento digital. Existen tarjetas SD con diferentes características en cuanto su clase, capacidad de almacenamiento y tamaño físico.
	
\end{tabular}

%VGA90 (): Es un modo de resolución gráfica para pantallas.. No se bien como es la broma…. >.<
 
 


  
 

  

 


% Establece las citas y bibliografia
\bibliographystyle{plainurl.bst}
\bibliography{myrefs}
\nomenclature{$a$}{The number of angels per unit area}%
\nomenclature{$N$}{The number of angels per needle point}%
\nomenclature{$A$}{The area of the needle point}%

% Crea el apendice
\appendix
%\chapter{Archivos intermedios}

%---------------------------------------------------------------------------------------%
\section {Representación de Dominio}

\label{archivos_intermedios:dom}
El archivo contiene los valores posibles para la variable que se codificó, el formato es 
valor y punto. En el caso de los \textit{flotantes} y \textit{doubles} el valor va entre comillas dobles.
Un ejemplo del nombre del archivo para un nodo con \texttt{id 5} es: \texttt{X5.dom}.
Un ejemplo del contenido del archivo suponiendo que es de tipo entero y tiene valores 
posibles \texttt{10}, \texttt{15} y \texttt{20}, es como el de la figura \ref{fig:ej_dominio}.
\begin{figure}[h]
\begin{lstlisting}[mathescape]
10.
15.
20.
\end{lstlisting}
\caption[Dominio Enteros]
{Dominio Enteros}
\label{fig:ej_dominio}
\end{figure}

%---------------------------------------------------------------------------------------%
\section {Representación de Rango}

\label{archivos_intermedios:siran}
El archivo que representa el rango para una variable independiente en el problema, tiene
en cada linea un valor correspondiente a la variable que se codificó. El nombre del 
archivo por ejemplo para el nodo con el \texttt{id 10} es: \texttt{s\_i\_10.ran}. El contenido
del archivo para una para una variable tipo \emph{string} que tenga valores posibles \texttt{hola},
\texttt{hello} y \texttt{hallo}, luce como la figura \ref{fig:ej_si}.

\begin{figure}[h]
\begin{lstlisting}[mathescape]
hola
hello
hallo
\end{lstlisting}
\caption[Rango Independiente String]
{Rango Independiente String}
\label{fig:ej_si}
\end{figure}

\label{archivos_intermedios:ran}
El archivo que representa el rango para un sistema, tiene los \texttt{id} de los nodos
que están involucrados, todos estos se encuentran al inicio del archivo separados por saltos de linea. Luego el símbolo \texttt{\#} y los
valores que satisfacen el sistema separados por espacio, el orden en el que se encuentran
corresponde al \texttt{id} de las variables que se encuentran al principio del archivo. 
El nombre del archivo por ejemplo para el sistema \texttt{0} es: \texttt{s\_0.ran}. El contenido del 
archivo para un sistema con una variable tipo float con \texttt{id 5} y otra entera con 
\texttt{id 9} y vectores solución \texttt{15.0 10} y \texttt{12.0 10}, luce como la figura \ref{fig:ej_s}.

\begin{figure}[h]
\begin{lstlisting}[mathescape]
5
9
#
15.0 10
12.0 10
\end{lstlisting}
\caption[Rango de Sistema]
{Rango de Sistema}
\label{fig:ej_s}
\end{figure}

%---------------------------------------------------------------------------------------%
\section {Sistemas en \emph{Prolog}}
\label{archivos_intermedios:pl}

Los archivos de \emph{Prolog} generados para resolver los sistemas lucen como el de la figura 
\ref{fig:ej_pl}.

\begin{figure}[h]
\begin{lstlisting}[mathescape]
probar(Vector):-
 nth0(0,Vector,X_4),
 nth0(1,Vector,X_8),
 !,
 (X_4 > 11),
 (X_4 < X_8).

main:- 
 consult(include/prolog/funciones_csp),
 inicializarEnteros('temp/X4.dom', X4),
 inicializarEnteros('temp/X8.dom', X8),
 solucionar([X4,X8],Vectores),
 escribirVectores(Vectores, [4,8], 'temp/s_0.ran').
\end{lstlisting}
\caption[Sistema para \emph{Prolog}]
{Sistema para \emph{Prolog}}
\label{fig:ej_pl}
\end{figure}

En la sección del main lo que hace es: primero cargar un conjunto de funciones auxiliares,
luego cargar los valores que estén en los archivos de dominio \ref{archivos_intermedios:dom}.
Al ya tener todos estos valores cargados, se generan todas las tuplas posibles, a estos se les
llamará ``vectores solución'', si cumplen las restricciones se imprimen estos vectores en el
formato que se describi'o en \ref{archivos_intermedios:ran}.

Lo que hace es que para el predicado probar, se toma el vector que se quiere verificar que cumpla 
la restricción y luego tomar de esa lista de valores que esta en 
\texttt{Vector} y comprobar que \texttt{X\_4} y \texttt{X\_8} cumplan la restricción que se codificó.


%\section{Glosario de t\'erminos}
\begin{tabular}{r p{16cm}l }
    Framework & \emph{Marco de trabajo}. Es un conjunto de técnicas, conceptos y estilos de trabajo que se establecen para resolver un problema particular y que sirve de referencia para solucionar problemas similares.\\

    TightVNC & Es un paquete de software que sirve para controlar la interfaz gráfica  de ordenadores remotos.\\

    AVR & Es una familia de microcontroladores de instrucciones reducidas de la compañía Atmel.\\

    IDE & \emph{Integrated development environment / Entorno de desarrollo integrado}. Es un programa diseñado para facilitar la programación en uno o varios lenguajes. Usualmente incluye herramientas de compilación, editor de textos y depurador.\\

    ROS & \emph{Robot Operating System / Sistema de operación para robots}. Es un framework que provee herramientas para ayudar a desarrolladores de aplicaciones para robots.\\

    Licencia BSD & \emph{Berkeley Software Distribution / distribución de software berkeley}. Es una licencia para software libre otorgada principalmente a sistemas BSD.\\
    
    XBEE & Es una familia de módulos de radio, con protocolo de comunicación inalámbrica basado en radio frecuencias.\\
    
    CSI & \emph{Camera Serial Interface / Interfaz serial para cámaras }. Es un estándar que define la interfaz de comunicación entre una cámara y un procesador. Es comúnmente utilizado en dispositivos móviles.\\
    
   MMAL & \emph{(Multi-Media Abstraction Layer / Capa de abstracción multimedia}: Es una librería que brinda una interfaz de bajo nivel para controlar dispositivos que se ejecutan en el núcleo de video de la Raspberry Pi, como el módulo de cámara.\\
   
   V4L & \emph{Video 4 Linux / video para linux}. Es una interfaz de programación de video para Linux. Algunos dispositivos soportados son cámaras web USB. \\
   
   RGB & \emph{Red Green Blue / rojo, verde, azul}: Es un modelo de color que se basa en la intensidad de los colores primarios de la luz (rojo, verde y azul).\\
   
   HSV & \emph{Hue, Saturation, Value / Matiz, Saturación, Valor}. Es un modelo de color que se basa en las cualidades de matiz, saturación y valor del color. \\
   
	HDMI & \emph{High-Definition Multimedia Interface/ interfaz multimedia de alta definición}. Es una interfaz para transferir datos de audio y video digital entre un dispositivo y un monitor, proyector, televisor digital o dispositivo de audio digital. \\
	
	SD & \emph{Secure Digital / Digital Seguro}. Es un formato de tarjetas de memoria de almacenamiento digital. Existen tarjetas SD con diferentes características en cuanto su clase, capacidad de almacenamiento y tamaño físico.
	
\end{tabular}

%VGA90 (): Es un modo de resolución gráfica para pantallas.. No se bien como es la broma…. >.<
 
 


  
 

  

 

\chapter{Consideraciones Especiales: Obstaculos y Soluciones} \label{chapter:consideraciones}

Durante el desarrollo del proyecto se presentaron algunos obstaculos que lograron ser resueltos. A continuación se describe la solución de algunos de esos obstáculos. 
\begin{itemize}
\item Errores de la tarjeta Arbotix para cargar programas:\\
	
	\textbf{Problema:} El IDE de Arduino 1.0.1 no funciona para quemar programar en la tarjeta Arbotix \\
  	\textbf{Soluci\'on:} Utilizar la versión 1.0.5 del IDE de Arduino  que compila los programas sin problema. \\
	

	 \textbf{Problema:} El gestor de arranque de la tarjeta Arbotix no estaba guardado.\\
	 \textbf{Soluci\'on:} Con el gestor de arranque de 'Sanguino' , se cargó a la Arbotix a través del    			    dispositivo programador para AVR llamado ' ISP programmer' que se muestra en la figura         \ref{fig:ISPprog}. 

	\begin{figure}[hbtp]
	\centering
	\includegraphics[scale=0.3]{imagenes/ISP.jpg}
	\caption{Programador para AVR.}
	\label{fig:ISPprog}
	\end{figure}
	
	

 \item Quema de motores Dynamixel:\\

	\textbf{Problema:} Debido al uso prolongado pero necesario los motores Dynamixel AX-12 se dañaban, ya sea por motor DC o en chip interno.\\
	\textbf{Soluci\'on:} Fue controlar el torque y la temperatura máxima a la que pueden llegar los motores. En caso de llegar a estas cotas máximas los motores se apagan automáticamente. Las cotas máximas han sido de $30\deg$ centígrados para la temperatura y 800 kgf-cm para el torque. Para lograr esto se ha tenido que modificar la librería Ax12 agregando procedimientos que permitieran establecer la temperatura y el torque máximo.\\

En el archivo ax12.h se agregaron las siguientes definiciones de funciones:

\begin{lstlisting}

#define SetTemperature(id,temp) (ax12SetRegister(id,AX\_LIMIT\_TEMPERATURE, temp))
#define SetAlarm(id) (ax12SetRegister(id,AX\_ALARM\_SHUTDOWN, 0x04)) 
#define SetTorqueL(id, tor) (ax12SetRegister2(id,AX\_MAX\_TORQUE\_L, tor)) 
\end{lstlisting}
 

El archivo ax12.h viene con el paquete de la pagina oficial para el código de Arbotix. Como se indica en las instrucciones, este archivo se debe ubicar en la carpeta sketchbook de Arduino. Luego desde el IDE de Arduino llamamos a las funciones definidas con los valores deseados. De esta manera se ha solucionado el problema de la quema de motores.

\end{itemize}

\textbf{Recomendaciones}  

Para la instalación del sistema operativo Raspbian en la tarjeta Raspberry Pi se recomienda tener en cuenta que algunas tarjetas SD no funcionan adecuadamente. Si al prender la mini computadora solo se prende el led rojo, como ha ocurrido en este proyecto, se debe verificar que la tarjeta SD esta haciendo buen contacto con el puerto en que se conecta. Si se verifica esto último y aún asi no prende, es probable que se deba intentar con otra tarjeta SD. Al inicio de este proyecto se ha usado una tarjeta mini-SD de 32GB, como no ha funcionado se ha reemplazado con una tarjeta SD de 4GB. Esta última ha funcionado, sin embargo se ha quedado sin capacidad de almacenamiento al instalar OpenCV, por lo que se ha reemplazado nuevamente por una tarjeta SD de 16GB. Esta ha sido suficiente para instalar todo lo necesario con holgura.    

Como no se contaba con un monitor con entrada HDMI o VGA se debió buscar una solución alterna para observar la interfaz gráfica de Raspbian de la Raspberry Pi. Se utilizó el programa TightVNC para la visualización y control de la interfaz de Raspbian desde un computador remoto. Ha sido necesario poder observar lo que el robot percibe para llevar un control y una supervisión de su comportamiento. 

Por último, sería conveniente advertir que para la instalación de ROS en la Raspberry Pi, la versión que se debe obtener es la más reciente, de lo contrario podrían ocurrir problemas de sincronización en la comunicación de las tarjetas.   




\clearpage
%\printglossary[type=\acronymtype]
%\printglossary
%\printnomenclature
%\printglossary

\end{document}
