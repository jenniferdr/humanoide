\chapter{Conclusiones y Recomendaciones} \label{chapter:conclusiones}

\label{chap:conclusiones}

La presente investigación ha nacido de la motivación por hacer que en Venezuela se incursione en proyectos que involucren humanoides autónomos e inteligentes. Se ha inspirado especialmente en la categoría Robocup soccer de la competencia internacional Robocup. Desde 1997, fecha en la que inició la competencia, Venezuela nunca ha participado en categorías con humanoides, mientras que países latinoamericanos como México, Brasil y Colombia sí han tenido avances en este campo. Si bien este proyecto no cumple con todas las reglas de la competencia se espera que éste pueda dar pie a continuar investigaciones dentro de Venezuela.

Los componentes utilizados en este proyecto son relativamente económicos comparados con otros en el mercado. La integración del kit Bioloid Premium con la Arbotix y la Raspberry Pi, ha hecho posible construir un humanoide inteligente sin tener que invertir exorbitantes cantidades de dinero. Una de las contribuciones más importantes es la coordinación y paralelismo exitoso entre todos los componentes utilizados.

Se logr\'o el dise\~no y construci\'on de Junny con la tarjeta Arbotix favorablemente, obteniendo un 99.7\% de moviemientos correctos simples y un 96\% en movimientos combinados los cuales son utilizados para el desplazamiento, la detecci\'on de la pelota arroj\'o un 100\% de aciertos en las pruebas de enfoque realizadas. La integraci\'ion de todos los componentes involucrados fue satisfactoria logrando asi un comportamiento aut\'omono e inteligente, los resultados obtenidos de las pruebas con desplazamiento predeterminado varian segu\'un en caso cuya menor tasa es de 70\% de pruebas positivas y la mayor es de 90\% estos casos Junny logr\'o llegar y patear la pelota directamente. Con aprendizaje por reforzamiento para el desplazamiento hacia la pelota luego del entrenamiento se obtuvo un 100\% de pateo directo al llegar a la pelota y una tasa de eficiencia de 0.72, finalmente en la orientaci\'on para el pateo al arco se obtuvo un 53.3\% de aciertos con resultado de gol. 
El porcentaje obtenido para la orientaci\'on al arco puede deberse a que el mismo requiere rodear la pelota para posicionarse correctamente donde la pelota este entre el arco y Junny, para realizar el rodeo se require movimientos de desplazamiento laterales que las caderas no pueden hacer, esto debido a que el grado de libertad faltante tuvo que ser eliminado para el ahorro de motores ya que se presento un incoveniente con los mismos para m\'as informaci\'on puede revisar los ap\'endices \ref{chapter:consideraciones} 

Se puede observar que en la comparaci\'on de movimientos predeterminados y aprendizaje que este \'ultimo obtuvo un mejor resultado en acertividad sin embargo  lo hizo compromentiendo el tiempo, de igual manera como se observo que con el aumento de pruebas de entrenamiento aumenta la eficiencia por lo tanto se puede decir que con el continuo entrenamiento Junny podr\'ia disminuir estos tiempos. Adem\'as un dato curioso a detacar es que el tiempo promedio de pruebas a la derecha (Caso IV) es en proporci\'on mucho mayor a las pruebas a la izquierda (Caso II) a pesar que las mismas son sim\'etricas esto nos indica que los siguientes entrenamientos deben hacer enfasis en los estados a la derecha del robot. 

De cualquier manera es recomendable la utilizaci\'on del aprendizaje pues no s\'olo logra realizar la tarea con mayor fidelidad sino que adem\'as es adaptable si las condiciones f\'isicas del robot llegaran a cambiar. Pues solo haría falta volverlo a entrenar sin tener que cambiar el código del programa.
 
Las recomendaciones directas con este proyecto ser\'ian perfeccionar el aprendizaje para obtener una mayor eficiencia en el n\'umero acciones realizadas, aumentando el número de regiones y acciones disponibles y aumentando el número de ejecuciones para el entrenamiento. De esta manera se podría reducir el tiempo que tarda en llegar a la pelota. Además agregarle otro grado de libertad a las 
caderas de Junny para que obtenga mejor movilidad y pueda rodear la pelota mientras busca el arco sin alejarla con ello tambi\'en se puede plantear la utilizaci\'on de otros motores que logren resistir m\'as las exigencias en velocidad, toque y fuerza del proyecto.

Tambie\'en se puede probar con aprendizaje la deteccio\'on de la pelota , realizando por formas pues aunque el resultado de la deteccio\'n utilizada fue satisfactorio tiene la limitante de que el color debe ser unico en el ambiente lo cual es improbable en la vida real.  

Otras mejoras que se pueden incorporar al proyecto a largo plazo podrían ser: añadir aprendizaje de m\'aquinas para hacer que el robot pueda predecir la posición de una pelota en movimiento para que pueda patearla en el momento indicado. Aprendizaje para saber que tipo de patada es mejor seg\'un la posición del arco, y aprendizaje para la búsqueda del arco.

Se espera que este proyecto se un primer paso y un impulso para continuar las investigaci\'ones con robots humanoides y que Junny pueda tener un compa\~nero de juego para incluir numerosos aprendizajes y habilidades m\'as.

%- Hacer mas pruebas
%- Aprendizaje para tipos de patada
%- Para la pelota en movimiento
%- Para movimiento de la camara
%- para la busqueda del arco
%- Ponerle el movimiento de las caderas
%- Otro robot
%- Comunicacion con otro robot
% Mejor detector de caidas incorporando otro tipo de giroscopio que de info mejor ... no se,, 
