\chapter{Conclusiones y Recomendaciones} \label{chapter:conclusiones}

\label{chap:conclusiones}

La presente investigación ha nacido de la motivación por hacer que en Venezuela se incursione en proyectos que involucren humanoides autónomos e inteligentes. Se ha inspirado especialmente en la categoría Robocup soccer de la competencia internacional Robocup. Desde 1997, fecha en la que inició la competencia, Venezuela nunca ha participado en categorías con humanoides, mientras que países latinoamericanos como México, Brasil y Colombia sí han tenido avances en este campo. Si bien este proyecto no cumple con todas las reglas de la competencia se espera que éste pueda dar pie a continuar investigaciones dentro de Venezuela.

Los componentes utilizados en este proyecto son relativamente económicos comparados con otros en el mercado. La integración del kit Bioloid Premium con la Arbotix y la Raspberry Pi, ha hecho posible construir un humanoide inteligente sin tener que invertir exorbitantes cantidades de dinero. Una de las contribuciones más importantes es la coordinación y paralelismo exitoso entre todos los componentes utilizados.

De los resultados obtenidos se evidencia que el aprendizaje por reforzamiento si bien genera mas trabajo, es recomendable su utilizaci\'on no solo por el hecho que logr\'o en relativamente pocas pruebas un 100\% de efectividad adem\'as es adaptable si las condiciones f'\'isicas del robot llegaran a cambiar pues lo que aprendio se mantiene. Lo cual no ocurre cuando los movimientos estan preestablecidos.
 
Las recomendaciones directas con este proyecto ser\'ian perfeccionar aun mas el aprendizaje para obtener una mayor eficiencia en el n\'umero acciones realizadas, agregarle otro grado de libertas a las caderas de Junny.

Las mejoras que se pueden incorporar al proyecto podrían ser: añadir aprendizaje por reforzamiento para hacer que el robot pueda predecir la posición de una pelota en movimiento para que pueda patearla en el momento indicado, aprendizaje para saber que tipo de patada es mejor seg\'un donde se encuentre el arco, para la busqueda del arco



- Hacer mas pruebas
- Aprendizaje para tipos de patada
- Para la pelota en movimiento
- Para movimiento de la camara
- para la busqueda del arco
- Ponerle el movimiento de las caderas
- Otro robot
- Comunicacion con otro robot
