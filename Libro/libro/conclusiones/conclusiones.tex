\chapter{Conclusiones y Recomendaciones} \label{chapter:conclusiones}

\label{chap:conclusiones}

La presente investigación ha nacido de la motivación por hacer que en Venezuela se incursione en proyectos que involucren humanoides autónomos e inteligentes. Se ha inspirado especialmente en la categoría Robocup soccer de la competencia internacional Robocup. Desde 1997, fecha en la que inició la competencia, Venezuela nunca ha participado en categorías con humanoides, mientras que países latinoamericanos como México, Brasil y Colombia sí han tenido avances en este campo. Si bien este proyecto no cumple con todas las reglas de la competencia se espera que éste pueda dar pie a continuar investigaciones dentro de Venezuela.

Los componentes utilizados en este proyecto son relativamente económicos comparados con otros en el mercado. La integración del kit Bioloid Premium con la Arbotix y la Raspberry Pi, ha hecho posible construir un humanoide inteligente sin tener que invertir exorbitantes cantidades de dinero. Una de las contribuciones más importantes es la coordinación y paralelismo exitoso entre todos los componentes utilizados.

Las mejoras que se pueden incorporar al proyecto podrían ser: la inclusión de aprendizaje por reforzamiento para patear de forma exitosa, incluir que la patada sea en dirección al arco e incluso añadir aprendizaje por reforzamiento para hacer que el robot pueda predecir la posición de una pelota en movimiento para que pueda patearla en el momento indicado.



