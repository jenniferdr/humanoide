\chapter{Conclusiones y Recomendaciones} \label{chapter:conclusiones}

\label{chap:conclusiones}

La presente investigación nace de la motivación de que en Venezuela se incursione en proyectos con humanoides autónomos e inteligentes. Se ha inspirado especialmente en la categoría Robocup soccer de la competencia internacional Robocup. Desde 1997, fecha en la que se inicia la competencia, Venezuela no ha participado en categorías con humanoides, mientras que países latinoamericanos como México, Brasil y Colombia sí han tenido presencia en estas categor\'ias. Si bien este proyecto no cumple con todas las reglas de la competencia se espera que éste pueda dar pie a continuar investigaciones en el pa\'is.

Los componentes utilizados en este proyecto son relativamente económicos comparados con otros en el mercado. La integración del kit Bioloid Premium con la Arbotix y la Raspberry Pi, ha hecho posible construir un humanoide inteligente sin tener que invertir exorbitantes cantidades de dinero. Una de las contribuciones más importantes es la coordinación y paralelismo exitoso entre todos los componentes utilizados.

Se logr\'o el dise\~no y construci\'on de Junny con la tarjeta Arbotix, obteniendo un 99.7\% de movimientos correctos simples y un 96\% en movimientos combinados, los cuales son utilizados para el desplazamiento, la detecci\'on de la pelota arroj\'o un 100\% de aciertos en las pruebas de enfoque realizadas. La integrac\'ion de todos los componentes involucrados fue satisfactoria logrando as\'i un comportamiento aut\'omono e inteligente, los resultados obtenidos de las pruebas con desplazamiento predeterminado varian seg\'un el caso, cuya menor tasa es de 70\% caso II de pruebas positivas y la mayor es de 90\% en los casos I y IV. Con aprendizaje por reforzamiento, para el desplazamiento hacia la pelota luego del entrenamiento, se obtuvo un 100\% de pateo directo al llegar a la pelota y una tasa de eficiencia de 0.72, finalmente en la orientaci\'on para el pateo al arco se obtuvo un 53.3\% de aciertos con resultado de gol. 
El porcentaje obtenido para la orientaci\'on al arco puede deberse a que  \'el mismo requiere rodear la pelota para posicionarse correctamente donde la pelota est\'e, entre el arco y Junny, para realizar el rodeo se require movimientos de desplazamiento laterales que las caderas no pueden hacer, esto debido a que el grado de libertad faltante tuvo que ser eliminado para el ahorro de motores. Para m\'as informaci\'on puede revisar los ap\'endices \ref{chapter:consideraciones} 

Se puede observar en la comparaci\'on de pruebas con movimientos predeterminados y pruebas con aprendizaje que estas \'ultimas obtuvieron un mejor resultado en acertividad. Sin embargo  lo hizo compromentiendo el tiempo, de igual manera se observ\'o que con el aumento de pruebas de entrenamiento aumenta la eficiencia, por lo tanto se puede decir que con el continuo entrenamiento Junny podr\'ia disminuir estos tiempos. 

A pesar del costo del tiempo se recomienda la utilizaci\'on del aprendizaje pues no s\'olo logra realizar la tarea con mayor fidelidad sino que adem\'as es adaptable por ejemplo si las condiciones f\'isicas del robot llegaran a cambiar en este caso s\'olo haría falta volverlo a entrenar sin tener que cambiar el código del programa.

Un resultado espec\'ifico de este proyecto fue la aceptaci\'on del art\'iculo "Intregraci\'on de Arbotix, Raspberry Pi y motores Dynamixel Ax-12+ con el objetivo de la construcción de un robot humanoide que busque  y patee pelotas" \cite{junny} en el Congreso de Reconocimiento de Patrones, Control Inteligente y Comunicaciones de la Universidad de Cuenca en Ecuador a realizarse en Diciembre 2014.

% Poner la referencia del Paper o del congreso o de ambos??
Para la continuaci\'on de este proyecto se recomienda perfeccionar el aprendizaje para obtener una mayor eficiencia en el n\'umero de acciones realizadas, aumentando el número de regiones y acciones disponibles y aumentando el número de ejecuciones para el entrenamiento. De esta manera se podría reducir el tiempo que tarda en llegar a la pelota. Además agregarle otro grado de libertad a las 
caderas de Junny para que obtenga mejor movilidad y pueda rodear la pelota mientras busca el arco sin alejarla. Tambi\'en se puede plantear la utilizaci\'on de otros motores que logren resistir m\'as las exigencias en velocidad, toque y fuerza del proyecto.

Otra sugerencia puede ser probar con aprendizaje para la detecci\'on de la pelota , realiz\'andolo en base a formas, pues, aunque el resultado de la deteccio\'n utilizada fue satisfactorio tiene la limitante de que el color debe ser \'unico en el ambiente lo cual es improbable en la vida real.  

Otras mejoras que se pueden incorporar al proyecto a largo plazo podrían ser: añadir aprendizaje de m\'aquinas para hacer que el robot pueda predecir la posición de una pelota en movimiento, de manera que pueda patearla en el momento indicado; a\~nadir aprendizaje para saber que tipo de patada es mejor seg\'un la posición del arco, a\~nadir aprendizaje para la búsqueda del arco y considerar la detecci\'on de obstaculos.

Se espera que este proyecto sea un primer paso y un impulso para continuar las investigaci\'ones con robots humanoides y que Junny pueda tener un compa\~nero de juego para incluir numerosos aprendizajes y habilidades m\'as.

%- Hacer mas pruebas
%- Aprendizaje para tipos de patada
%- Para la pelota en movimiento
%- Para movimiento de la camara
%- para la busqueda del arco
%- Ponerle el movimiento de las caderas
%- Otro robot
%- Comunicacion con otro robot
% Mejor detector de caidas incorporando otro tipo de giroscopio que de info mejor ... no se,, 
