\chapter{Conclusiones y recomendaciones} 

\label{chap:conclusiones}

Logrando los objetivos planteados, se consiguió diseñar e implantar primero un 
lenguaje que facilitará la especificación de los problemas; y luego la 
herramienta en si (Parser, compilador e instanciador)
que permite generar instancias aleatorias, de objetos descritos como entradas,
cumpliendo las especificaciones de los mismos, así como también las 
restricciones que contienen. A partir de esta herramienta se lograron obtener
soluciones a los problemas planteados para pruebas. Se pudieron obtener
grupos de una, varias o incluso todas las soluciones posibles para las pruebas.

El lenguaje diseñado es cómodo y bastante expresivo en comparación con la
mayoría de las librerias de \textbf{CSP} o con la implementación de un algoritmo
procedural en un entorno imperativo, por lo que resulta fácil
acostumbrarse a su sintaxis. Además el diseño del lenguaje permite añadir 
nuevas funcionalidades sin tener que hacer demasiados cambios para que
se mantenga funcional.

El tiempo de ejecución de la herramienta fue para todos los ejemplos bastante
breve, en donde la escritura en disco fue la causa de la mayor demora en la 
corrida.
 
Además, se lograron también buenos resultados al interpretar las respuestas como
objetos finales. Para esto se probaron casos que abarcan una gran cantidad de
áreas de la computación, desde las más abstractas como las bases de datos hasta
las más ilustrativas como la computación gráfica.

Se analizaron las principales formas para optimizar la herramienta, se 
implementaron las que involucran la intervención y se especificaron
las automáticas para ser implementadas en trabajos futuros.

Ahora bien, la herramienta es funcional y cumple con las expectativas
iniciales, pero quedan muchas funcionalidades adicionales por ser
implementadas. Para eso, a continuación se proporcionará una lista de recomendaciones 
para posibles trabajos futuros que re-implementen o continúen
con este ambicioso proyecto:

\begin{itemize}
\item{Utilizar un lenguaje que permita modificar fácilmente la estructura
del lenguaje diseñado y de la herramienta: Esto permitirá que se puedan integrar
nuevas funciones rápidamente.}

\item{Implementar el funcionamiento de las listas: Con esto se reduciría
en gran cantidad la cantidad de variables de igual comportamiento que se
deben especificar como entrada.}

\item{Implementar el funcionamiento de las funciones: Con esto se pueden
aplicar restricciones mucho más complejas y que permitan expresar mejor
algunos problemas.}

\item{Añadir la posibilidad de recibir parámetros a la corrida de la 
herramienta: Con esto podrían pedirse respuestas para objetos similares pero
con diferencias sin tener que reescribir la entrada por completo.}

\item{Implementar la posibilidad de describir restricciones mediante
disyunciones: Con esto se aumentan las posibles descripciones de los
objetos pero significa aumentar en gran medida la complejidad del proceso de
resolución.}

\item{Implementar el sistema de decisión de método de resolución para
los subsistemas: actualmente esto no representa una mejora significativa, pero
considerando la implementación de las demás recomendaciones, este se vuelve
fundamental para mantener la herramienta eficiente.}

\item{Mejorar el motor de resolución de sistemas en \textit{Prolog} o substituirlo
con uno más complejo y eficiente. Adicionalmente, implementar otros métodos de
resolución, en especial el de tipo numérico, o bien crear un motor de 
resolución dedicado a este tipo de problemas.}
\end{itemize}

Luego de finalizado el proyecto se descubrieron una gran cantidad de posibles
usos para la herramienta. Estos seguramente, son sólo una minúscula
fracción de las posibilidades que se pueden generar con este proyecto. Cada una
de estas son en realidad una gran cantidad de soluciones a un mismo problema.
Sin tener que crear un programa específico sino un pequeño pero representativo
modelo de lo que se quiere.

Se tiene la seguridad de que este trabajo puede ser la piedra angular de nuevas 
formas de diseño y generación de contenido. Que sean accesibles a personas
de múltiples especialidades y que resuelvan cualquier cantidad de problemas.
Faltan aún muchas mejoras para poder catalogar la herramienta como un producto
finalizado y mucho menos un producto comercial. Aún así se tiene la expectativa
de que este proyecto será continuado y mejorado. ¿Quién sabe? Quizá sirva
como base para lo que en un futuro conozcamos como creatividad artificial...
