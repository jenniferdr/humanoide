\section{Robótica} \label{sect:robotica}
 
Para definir un lenguaje formal se requiere describir:
\begin{itemize}
\item{\textbf{Robot:} Es un agente artificial, activo, cuyo entorno no es el mundo físico. El término activo descarta de esta definición a las piedras, el término artificial descarta a los animales, y el término físico descarta a los agentes de software puros o softbots, cuyo entorno lo constituyen los sistemas de archivos, bases de datos y redes de cómputo. \cite{peterNorvig}}

\item{\textbf{Robótica:} Es la rama de la tecnología que se encarga del diseño, construcción, operación y aplicación de los robots. \cite{oxfordRobotics}}

\item{\textbf{Sensores:}  Son los encargados de percibir el ambiente que rodea al robot. Según Murphy R.R son dispositivos que miden algún atributo del mundo. Un sensor recibe energía del entorno (sonido, luz, presión, temperatura, etc) y transmite una señal a una pantalla o computador ya sea de forma análoga o digital. \cite{AiRobotics}}

\item{\textbf{Actuador:}  Es aquella parte del robot que convierte comandos de software en movimientos físicos.  \cite{peterNorvig}}

\item{\textbf{Servomotor:}  Es un motor eléctrico, considerado como actuador, que permite ser controlado tanto en velocidad como en posición. }

 %http://www.ceiarteuntref.edu.ar/badarte/node/112


\item{\textbf{Giróscopio:} Es un sensor utilizado para medir y mantener la orientación, se mide a través del momento angular. \cite{gyro1}}
\end{itemize}

%****************************************************************************************/
\section{Robótica Inteligente (Agentes Inteligentes)} \label{sect:AgentesInteligentes}
\subsection{Paradigmas de robótica}
En la robótica inteligente, según Robin Murphy en \cite{AiRobotics}, existen tres paradigmas en los cuales se clasifica el diseño de un robot inteligente, estos paradigmas pueden ser descritos de dos maneras: la relación entre las primitivas básicas de la robótica  percibir, planificar, actuar ,o de la forma en que los datos son percibidos y distribuidos en el sistema.

Percibir se refiere al procesamiento útil de la información de los sensores del robot. Planificar, cuando con información útil, se crea un conocimiento del mundo y se generan ciertas tareas que el robot podría realizar. Por último actuar consiste en realizar la acción correspondiente con los actuadores del robot para modificar el entorno. 

\subsubsection{ Paradigma Jerárquico}

Este paradigma es secuencial y ordenado. Primero el robot percibe el mundo y construye un mapa global. En base al mapa ya percibido y con “los ojos cerrados”, el robot planifica todas tareas necesarias para lograr la meta. Luego ejecuta la secuencia de actividades según la planificación realizada. Una vez culminada la secuencia se repite el ciclo percibiendo el mundo, planificando y actuando. \cite{AiRobotics}

\subsubsection{Paradigma Reactivo}
El paradigma reactivo omite por completo el componente de la planificación y solo se basa en percibir y actuar. El robot puede mantener un conjunto de pares percibir-actuar, estos son llamados comportamientos y se ejecutan como procesos concurrentes. Un comportamiento toma datos de la percepción del mundo y los procesa para tomar la mejor acción independientemente de los otros procesos. \cite{AiRobotics}
