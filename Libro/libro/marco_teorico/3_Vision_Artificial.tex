\section{Visión Artificial} \label{sect:Vision_Artificial}

Una manera de obtener información del ambiente es con la visión artificial. Esta consiste en usar un dispositivo (cámara) que capta un rango de espectro electromagnético y produce una imagen. La representación de la imagen se almacena como una matriz de píxeles, cada píxel es un elemento que guarda información de una región en el espacio captado. Si se usa una cámara de luz, la información de cada píxel será el color. \cite{AiRobotics}  

Por lo general luego de obtener una imagen se requiere extraer información de ella, por lo cual se han desarrollado diferentes algoritmos y t\'ecnicas que ayudan en esta tarea. En la sección \ref{sec:Segmentacion} se describe la t\'ecnica de segmentaci\'on por regiones, que es la utilizada en el proyecto. 

Por otro lado, existen varios algoritmos que se dedican a la transformación de las imágenes para reducir
ruidos, compensar problemas de iluminación, extraer formas, identificar objetos, entre otros. En esta sección se describen dos de las técnicas de transformación para reducir el ruido basadas en la dilatación y erosión de la imagen (secci\'on \ref{sec:Transfor}). 
 
\subsection{Segmentaci\'on por regiones}\label{sec:Segmentacion}

La pr\'actica m\'as general en visi\'on de computadoras aplicado a rob\'otica es la identificaci\'on de regiones por un color particular, este proceso se llama segmentaci\'on por regiones. El algoritmo b\'asico consiste en identificar todos los p\'ixeles en una imagen que forman parte de una regi\'on y luego ir al centro de la regi\'on. El primer paso es identificar todos los p\'ixeles en la imagen que compartan un rango de valores con el color particular elegido y agruparlos, aquellos p\'ixeles que no compartan el color son descartados \cite{BookOpenCv}. 

\subsection{Filtros}
El filtrado de imágenes es una técnica para la transformación de imágenes, que consiste en destacar  sus características más relevantes en base a un propósito en particular. 

Generalmente en la tarea de extracción de información de una imagen se utilizan filtros para descartar zonas o características que no son importantes para el patrón deseado y para determinar el área deseada ya sea por patrones de forma o color.

En la investigación, los algoritmos de filtrado aplicados a las imágenes fueron: Clausura Morfológica y Apertura Morfológica, filtros que aplican las técnicas de erosión y dilatación a las imágenes.

\subsection*{Transformaciones Morfológicas}\label{sec:Transfor}
Algunas transformaciones morfológicas básicas son dilatación, erosión, uni\'on e intersecci\'on, se utilizan en amplia variedad de contextos como la eliminación del ruido, aislamiento de elementos individuales y elementos de unión dispares en una imagen en este proyecto de utilizaron dilataci\'on y erosi\'on.\cite{BookOpenCv}

\subsubsection{Dilatación}
La dilatación es una convulsión (patr\'on que se le aplica a toda la imagen) entre alguna imagen (o región de una imagen), que llamaremos $A$ y un núcleo que llamaremos $B$, el núcleo, que puede ser de cualquier forma o tamaño,generalmente una figura geom\'etrica un cuadrado o disco, tiene un solo punto de referencia definido. Para mayor claridad el n\'ucleo es una matriz de tamaño fijo de coeficientes numéricos junto con un punto de referencia en dicha matriz, que normalmente se encuentra en el centro. El núcleo puede ser pensado como una plantilla  o m\'ascara, y su efecto para la dilatación tal como un operador de máximo local sobre la imagen, se calcula el m\'aximo valor de los píxeles común a $B$ y reemplazamos el píxel de la imagen en el punto de referencia con ese valor máximo. Esto causa regiones brillantes dentro de una imagen y la hacen crecer. Este crecimiento es el origen del término `` operador de dilatación" \cite{BookOpenCv}. 

\begin{figure}[hbtp]

\centering
\includegraphics[scale=0.2]{imagenes/erosion-model.jpg}
\caption{Dilatación A: es la imagen original, B: es el n\'ucleo, La estrella es el punto de referencia. Se ve como aumenta la imagen en proporci\'on al patr\'on aplicado }
\end{figure}

\subsubsection{Erosión}
La erosión es la operación inversa a la dilatación. Esta acción del operador es equivalente a el cálculo de un mínimo local sobre el área del núcleo. La erosión genera una nueva imagen a partir de la original, utilizando el siguiente algoritmo: como el núcleo $B$ es analizado sobre la imagen, se calcula el mínimo valor del píxel superpuesto por $B$ y se reemplaza el píxel de la imagen con un punto de referencia de valor mínimo \cite{BookOpenCv}. 
V\'ease en la figura \ref{fig:erosion}

\begin{figure}[hbtp]
\centering
\includegraphics[scale=0.3]{imagenes/erosion.jpg}
\caption{Erosión,  A: es la imagen original, B: es el n\'ucleo, La estrella es el punto de referencia. Se ve como disminuye la imagen en proporci\'on al patr\'on aplicado}
\label{fig:erosion}
\end{figure}

Este cap\'itulo constituyo la base te\'orica que sustenta el proyecto, en el siguiente cap\'itulo se presenta el proceso de desarrollo que se sigui\'o para su culminaci\'on. 
