\section{Inteligencia Artificial} \label{sect:Inteligencia_Artificial}
La inteligencia artificial es un término relacionado con la computación que puede ser aplicado a la robótica para crear robots inteligentes. El término ``inteligencia artificial"\:
 ha tenido varias definiciones. Ocho de ellas, las cuales nacieron a finales del siglo XX, se encuentran organizadas en \cite{peterNorvig} bajo cuatro categorías: pensar y actuar de forma humana, pensar y actuar de forma racional. De ellas se puede entender que la inteligencia artificial tiene que ver con lograr que una máquina o un robot resuelva problemas de manera inteligente, es decir, de manera que parezca que el razonamiento y comportamiento humano las ha resuelto.  

\subsection{Aprendizaje de Máquinas}

El aprendizaje de máquinas es un área de la inteligencia artificial enfocada en lograr construir programas de computadora que automáticamente mejoren con la experiencia. "Se dice que un programa aprende de la experiencia E con respecto a una tarea T y desempeño P si el desempeño en la tarea T, medido por P, mejora con la experiencia E"  \cite{Mitchell}.

\subsection{Aprendizaje por reforzamiento}
El aprendizaje por reforzamiento es un tipo de aprendizaje de máquinas que se basa en un sistema de recompensas positivas y negativas. Existen dos maneras en las que se puede otorgar las recompensas, darla cada vez que se llega a un estado o una sola vez al llegar al estado final. Un estado se define como la configuraci\'on de un conjunto de características, reelevantes para el problema, percibidas del mundo en un momento particular.

El agente existe en un entorno conpuesto por un conjunto de estados S. En cada estado puede ejecutar un una acción del conjunto de acciones A. Cada vez que ejecuta una acción en algún estado el agente recibe una recompensa. El objetivo es aprender una política $\pi$:S$\to$A que maximice la suma esperada de esas recompensas con descuento exponencial de las recompensas futuras \cite{Mitchell}. Es decir, aprender cuál es la mejor acción en cada estado.

El resultado de tomar las acciones puede ser determinista o no, en el caso de este proyecto no es determinista, es decir, existen porcentajes de probabilidad de pasar a un estado u otro al tomar una acción en un estado en particular.
%El agente debe escoger una acción en cada estado y dependiendo de sus recompensas, aprender si es favorable o no la aplicaci\'on de esas acciones en esos estados.
  
\subsection{Q- learning}\label{subsec:Qlearning}

Es un método de aprendizaje por reforzamiento que consiste en comparar las utilidades esperadas de las posibles acciones a tomar sin necesidad de saber el estado resultante, por tanto no se necesita tener un modelo del entorno \cite{peterNorvig} (esto es, cómo funciona el ambiente o qu\'e estado se alcanza como consecuencia de tomar cada acción).

La forma de aprender la política $\pi$:S$\to$A es de forma indirecta, a través de la función $Q(s,a)$. La función representa el valor de la máxima recompensa acumulada, con descuento de las recompensas futuras, que puede ser alcanzada comenzando desde el estado $s$ y aplicando $a$ como la primera acción \cite{Mitchell}. La ecuación se puede escribir como:
\begin{equation}
Q(s,a) = r(s,a) + {\gamma V^*(\delta(s,a))}
\end{equation}


  %L' = {L}{\sqrt{1-\frac{v^2}{c^2}}}
En donde $r(s,a)$ es la recompensa dada según el resultado de haber tomado la acción $a$ en el estado $s$. $\delta(s,a)$ es el estado obtenido luego de tomar la acción $a$ en el estado $s$. $\gamma$ es el descuento que se le aplica a las recompensas futuras. La funcion $V^*(s')$ genera el máximo valor $Q$ que puede ser alcanzado desde el estado $s'$. Esto es,

\begin{equation}
V^*(s') = \max_{a'} (Q(s',a'))
\end{equation}



De esta forma se obtiene una definición recursiva,
\begin{equation}
Q(s,a) = r + \gamma \max_{a'} Q(\delta(s,a) , a')  
\end{equation}

\noindent
que puede ser calculada de manera iterativa, inicializando los valores de Q con números aleatorios \cite{Mitchell}. Como la función $\delta(s,a)$ no es conocida, es necesario poner en ejecución al agente para que, una vez tomada la acción, se observe el estado resultante que desencadenó y así poder calcular el resultado de la ecuación. La recompensa $r$ se puede definir en base  a qu\'e tan bueno es el estado resultante. 


