\chapter{Experiementos y Resultados}\label{chapter:resultados}
En este capítulo se describen los experimentos realizados para alanizar el desempeño del robot. Los experimentos se han dividido en tres partes. El primer conjunto de experimentos (sección \ref{sec:experimentosMov}) se ha realizado para verificar el desempeño y balance del robot al ejecutar uno o varios movimientos en secuencia. El segundo y tercer conjunto de experimentos se bas\'o en verificar el desempeño del robot al buscar la pelota. En el segundo conjunto (sección \ref{sec:experimentosintegrados}) la forma de escoger la acciónes de moviento han sido fijadas para cada región en la que se detecta la pelota. Mientras que para el tercer conjunto de experimentos (secci\'on \ref{sec:experimentosAprend}) la forma de escoger las acci\'ones de movimiento ha sido resultado del aprendizaje por reforzamiento. 
 
\section{Experimentos de Movimientos}\label{sec:experimentosMov}

Se realizaron varios experimentos para comprobar la movilidad del robot en cuanto a un conjunto de acciones de movimiento (mencionadas en la sección \ref{esqueleto}). Estas acciones son: caminar hacia adelante (4.8 cm), girar hacia la derecha (3 cm), girar hacia la izquierda (3 cm), patada con la pierna derecha, patada con la pierna izquierda, levantarse en la posición supino, levantarse en posición prono.

La primera etapa consistió en verificar el desempeño de las unidades de acciones, para ello se procedió a realizar la ejecución de cada unidad 50 veces, 250 ejecuciones en total. Se tomó nota de las veces que lograba realizar el movimiento sin fallas, es decir, de forma exitosa y de aquellas que lograba completar el movimiento pero con fallas. Los resultados obtenidos fueron un 99.6\% de casos exitosos y un 0.4\% de casos en los que pudo recuperarse de fallas. El 0.4\% representa un solo caso en el que el robot se ha caído y levantado.  

En la segunda etapa se evalu\'o la ejecuci\'on de combinaciones de varias unidades de movimiento. Las combinaciones elegidas fueron: 

\begin{itemize}
\setlength{\itemsep}{1pt}
\item Caminar hacia adelante y patear con la pierna derecha 
\item Caminar hacia adelante y patear con la pierna izquierda
\item Caminar hacia adelante, girar hacia la derecha y patear con la pierna derecha
\item Caminar hacia adelante, girar hacia la derecha, patear con la pierna izquierda 
\item Caminar hacia adelante, girar hacia la izquierda y patear con la derecha
\item Caminar hacia adelante, girar hacia la  izquierda y  patear con la pierna izquierda
\item Girar hacia la izquierda y patear con la pierna derecha
\item Girar hacia la derecha y patear con la izquierda
\item Girar hacia la derecha y patear con la pierna derecha
\item Girar hacia la izquierda y patear con la pierna izquierda
 \end{itemize}

Se realizaron 30 ejecuciones de cada una de estas combinaciones, 300 ejecuciones en total. Los resultados se muestran en el gráfico de la figura ~\ref{fig:etp2}. El resultado ha sido satisfactorio, obteniendo un 96\% de casos exitosos y un 4\% de casos con fallas, de las que ha podido recuperarse. La mayoría de las fallas se presentaron cuando al patear se caía, sin embargo lograba recuperarse.  

\begin{figure}[th!]
\begin{tikzpicture}
\pie[pos={10 ,0}, radius =2, explode =0.3, text = legend, rotate =180]{96/Exitosas, 4/Con fallas corregidas}
\end{tikzpicture}
\caption{Segunda etapa: 300 ejecuciones de acciones combinadas}
\label{fig:etp2}
\end{figure}

\section{Experimentos con Comportamientos Integrados}
\label{sec:experimentosintegrados}

El segundo conjunto de experimentos consistió en la realización de pruebas de desempeño en donde se observó el comportamiento global del robot con todos sus comportamientos integrados (detección, búsqueda y pateo). La forma de elegir la acciones de movimiento es la que se describe en la sección de búsqueda y pateo (sin aprendizaje). 

El  primer caso (CasoI) consistió en la colocación de la pelota en la zona de pateo (las regiones 1 y 2 de la figura ~\ref{divisionCam}). El número de pruebas en este primer caso fue de 10, cuya duración aproximada de cada prueba fue de 30 segundos. En la figura ~\ref{fig:caso1} se puede observar los resultados obtenidos de esta prueba.

\begin{figure}
\begin{tikzpicture}
\pie[pos={10 ,0}, radius =2, explode =0.3, text = legend, rotate =180]{90/Pateó, 10/Falló}
\end{tikzpicture}
\caption{Pruebas de movimientos integrados CASO I}
\label{fig:caso1}
\end{figure}

El segundo caso (CasoII) consistió en en la colocación inicial de la pelota a una distancia de 50 cm en línea recta al robot. Se realizaron 10 pruebas, siendo la duración promedio de 2 minutos con 12 segundos. Los resultados de dicho caso se pueden observar en la figura ~\ref{fig:caso2}. 

\begin{figure}
\begin{tikzpicture}
\pie[pos={10 ,0}, radius =2, explode =0.3, text = legend, rotate =180]{70/Exitosas, 10/Con fallas corregidas, 20/Fallidas}
\end{tikzpicture}
\caption{Pruebas de movimientos integrados CASO II}
\label{fig:caso2}
\end{figure}

El tercer caso  (CasoIII) consistió en en la colocación inicial de la pelota a una distancia de 50 cm en línea recta al robot y 10 cm a la izquierda formando asi una diagonal. Se realizaron 10 pruebas, siendo la duración promedio de 3 minutos con 29 segundos. Los resultados de dicho caso se pueden observar en la figura ~\ref{fig:caso3}. 

\begin{figure}
\begin{tikzpicture}
\pie[pos={10 ,0}, radius =2, explode =0.3, text = legend, rotate =180]{80/Exitosas, 20/Fallidas}
\end{tikzpicture}
\caption{Pruebas de movimientos integrados CASO III}
\label{fig:caso3}
\end{figure}

El cuarto caso (Caso IV) consistió en en la colocación inicial de la pelota a una distancia de 50 cm en línea recta al robot y 10 cm a la derecha. Se realizaron 10 pruebas, siendo la duración promedio de 4 minutos con 48 segundos. Los resultados de dicho caso se pueden observar en la figura ~\ref{fig:caso4}. 

\begin{figure}
\begin{tikzpicture}
\pie[pos={10 ,0}, radius =2, explode =0.3, text = legend, rotate =180]{90/Exitosas, 10/Fallidas}
\end{tikzpicture}
\caption{Pruebas de movimientos integrados CASO IV}
\label{fig:caso4}
\end{figure} 

Con un total de 40 pruebas de comportamiento integrado, los resultados obtenidos han sido satifactorios con un porcentaje de logro del 82\% mas un 5.5\% en el cual se logró recuperar y finalizar la tarea, con 12.5\% de fallas en la tarea se pudo observar de manera general que la razón de dichas fallas ha sido ocasionada por problemas de batería.
 

\section{Experimentos con Aprendizaje}\label{sec:experimentosAprend}

Finalmente los resultados de la aplicaci\'on del apredizaje Q, para la busqueda de la pelota se presentan a continuaci\'on.
Se utilizar\'on las siguientes combinaciones de tasa de descuento $\gamma$ , exploraci\'on y explotaci\'on por medio del par\'ametro ajustable $k$ detallado en \ref{subsec:eleccionAccion}.

\begin{itemize}
\item $K$ = 1 , $\gamma$ = 0.1 (Aleatorio)
\item $K$ = 2 , $\gamma$ = 0.1 
\item $K$ = 2 , $\gamma$ = 0.7
\item $K$ = 3 , $\gamma$ = 0.1
\item $K$ = 3 , $\gamma$ = 0.7
\item $K$ = 5 , $\gamma$ = 0.1
\item $K$ = 5 , $\gamma$ = 0.7
   
\end{itemize}

Cada una de estas combinaciones se entreno con 20 pruebas completas (comportamiento integrado), una vez entrenadas se realizaron las siguientes pruebas para medir su desempeño y establecer la mejor tasa de aprendizaje por medio del porcentaje de pateos positivos y fallas. Se realizaron 10 pruebas con el resultados del aprendizaje para cada una de las combinaciones establecidas, eso nos da un total de 210 pruebas realizadas. 

Los resultados de los entrenamientos arrojaron que la mejor combinaci\'on de fue K = 3 y Descuento = 0.1 que obtuvo un desempeno favorable del 90\% de las pruebas realizadas.

Para esta mejor combinaci\'on se inicializo la matrix de estados x acciones donde solo se daban recompensas negativas a las acciones que por ning\'un motivo debia tomar en dicho estado. Con ello y el aumento de la cantidad de pruebas de entrenamiento a 30 se pretendio mejoras el desempeño general del robot. 

Se realizaron pruebas de desempeño con la cantidad de 15 pruebas donde se obtuvo un 100\% de pruebas positivas


FALTTTAAAAN
- Explicar mejor que significa K y descuento ?
- FALTA describir mejor como es una prueba
- No se si hacer graficos
- Explicar mejor como fue la inicializacion
- analisis en profundidad de tiempos y acciones

   


