\chapter{Implementación y resultados}\label{chapter:imp_y_res}

En el presente capítulo se explicará la implementación realizada según la 
especificación presentada los capítulos \ref{chapter:def_lenguaje} y \ref{chapter:dise_solver_compilador}. Así como
también se presentarán los resultados obtenidos para los ejemplos dispuestos en el apendice 
\ref{ejemplos_lenguaje}.

\section{Implementación}
La implementación del \emph{Parser} fue realizada en \emph{C++} usando las herramientas \emph{Flex}
\cite{Flex} y \emph{Bison} \cite{Bison} para el reconocimiento del programa, estos 
corresponden al \emph{Lexer} y \emph{Parser} discutidos en la sección \ref{sect:Compiladores}.

La herramienta de \textbf{CSP} (discutida en la sección \ref{sect:CSP}) usada fue \emph{Prolog}
\cite{Prolog}. \emph{Prolog} es un lenguaje basado en el paradigma de programación
lógica. Basa su funcionamiento en el razonamiento de expresiones lógicas y la 
unificación de términos para satisfacer tales expresiones, usando el mecanismo 
de inferencia conocido como resolución y una estrategia de \emph{backtracking} para 
poder explorar los distintos espacios de búsqueda.
Esto permitió que la escritura del mecanismo de resolución por \emph{backtracking} sea
muy simple y no tener que ahondar en el \emph{Solver} para poder dedicar más esfuerzo
a los objetivos principales del proyecto.
La comunicación con \emph{Prolog} se logra a través de archivos y se realizan llamadas al sistemas
operativo para ejecutarlo (que será descrito más adelante en esta sección)
desde la herramienta. Los archivos intermedios generados se 
encuentran por defecto en un directorio temporal y son de 4 tipos diferentes:

\begin{itemize}
\item{El archivo de dominio de valores discutido en la sección \ref{solver:dom}. Una 
  explicación con más detalles se encuentra en el apéndice \ref{archivos_intermedios:dom}.}
\item{El archivo correspondiente a una variable independiente en el sistema es reconocida
  en la herramienta y se deja sólo para optimizar cálculos. Una explicación con más
  detalles se encuentra en el apéndice \ref{archivos_intermedios:siran}.}
\item{Cuando una variable se encuentra relacionada con otras o tiene al menos una
  restricción se devuelve la solución en un archivo de que se llama rango. Una 
  explicación con más detalles se encuentra en el apéndice \ref{archivos_intermedios:ran}.}
\item{Por último se encuentran los archivos de \emph{Prolog} que contienen las instrucciones
  para la resolución del sistema y generación de archivo de salida. Una explicación con más 
  detalles se encuentra en el apéndice \ref{archivos_intermedios:pl}.}
\end{itemize}

Finalmente las instancias generadas son del formato \emph{JSON} \cite{JSON} y se devuelven
por defecto en un directorio que contendrá todas las salidas obtenidas por la herramienta.

Para cada uno de los ejemplos se crearon programas a manera de librerías, en los que
se reconocieron las soluciones obtenidas por la herramienta para convertirlas en objetos
dentro de un lenguaje o contexto específico. Estos programas fueron implementados en 
los lenguajes de programación \emph{Python} \cite{Python} y \emph{$C\#$} \cite{CS} debido a la facilidad que ofrecen a la
hora de interpretar los archivos en formato \emph{JSON} y por las necesidades de librerías para
mostrar o utilizar los objetos generados.
%----------------------------------------------------------------------------------------%
\section{Pruebas realizadas a nivel del Solver}\label{imp_y_res:sol}
Para las pruebas se estructuraron y probaron varios problemas como entrada para
la herramienta. Estos problemas fueron los considerados más importantes en cuanto a utilidad e interés 
para las pruebas. 

En el caso de los ejemplos contenidos en el apéndice \ref{ejemplos_lenguaje} y
otros que tenían un nivel de complejidad parecido en cuanto cantidad de variables
involucradas y restricciones, lograron generar resultados satisfactorios y muy rápido,
tal y como se muestra en la tabla \ref{table:comparacion_resultados}.

\begin{table}[h]
  \centering	
\rowcolors{2}{}{gray!50}
\begin{tabular}{|l||l|l|l|l|}
\hline
    Instancias & 1 & 100 & 1000 & 10000 \\
    \hline
    \hline
    ~ & real	0m0.064s & real	0m0.166s & real	0m0.262s & real	0m1.194s \\
    Base de Datos  & user	0m0.056s & user	0m0.060s & user	0m0.152s & user	0m0.924s \\
    apéndice \ref{ejemplo:bd} & sys	0m0.004s & sys	0m0.028s & sys	0m0.028s & sys	0m0.192s \\
    \hline
    \hline
    ~ & real	0m0.099s & real	0m0.114s & real	0m0.490s & real	0m6.043s \\
    Computación Gráfica  & user	0m0.092s & user	0m0.100s & user	0m0.368s & user	0m5.184s \\
    apéndice \ref{ejemplo:cg} & sys	0m0.000s & sys	0m0.012s & sys	0m0.060s & sys	0m0.744s \\
    \hline
    \hline
    ~ & real	0m1.059s & real	0m1.190s & real	0m1.224s & real	0m1.785s \\
   HTML  & user	0m1.004s & user	0m1.044s & user	0m1.068s & user	0m1.460s \\
   apéndice  \ref{ejemplo:html} & sys	0m0.044s & sys	0m0.048s & sys	0m0.072s & sys	0m0.208s \\
    \hline
\end{tabular}
\caption{Comparación de resultados}
\label{table:comparacion_resultados}
\end{table}

Sin embargo, una de las pruebas demostró que un sistema sencillo con pocas variables, pero con muchos
valores posibles en el dominio, puede hacer que la herramienta interrumpa su ejecución.
Por ejemplo si se tienen 3 variables con 1000 valores posibles en el dominio por cada una, se generaran
$1000 \times 1000 \times 1000$ vectores solución al comenzar la ejecución, los cuales no caben en la 
memoria de \emph{Prolog}. Aunque, si se contara con un sistema de 
evaluación perezosa como el de \emph{Haskell} no pasaría esto ya que se generarían 
vectores solución a medida que se necesitan.

Otras pruebas que ocasionan problemas son aquellas que tienen muchas restricciones. 
Debido a que \emph{Prolog} tardaba mucho en su ejecución, estas producen un cierre inesperado del programa. 
Esto ocurre porque busca evaluar las restricciones en un orden que probablemente es poco conveniente
para el \emph{solver}.

Es responsabilidad del programador el escribir una descripción de las estructuras
que permita un mejor desempeño de la herramienta, ya que no especificar el dominio de una variable,
o tener restricciones en un orden poco conveniente de revisión, repercute 
negativamente sobre el desempeño de la herramienta, al igual que en cualquier otro lenguaje,
el colocar en determinado orden instrucciones o hacer mal uso de ciertos recursos, suelen tener
resultados no deseados.

Los resultados generados abarcan la totalidad de las respuestas posibles. Es decir,
para cada ejecución de la herramienta, todas las instancias válidas como solución
fueron generadas. Por ende se puede afirmar que la herramienta cumple con la condición
de completitud en cuanto a los resultados generados.

Todos los resultados cumplen con las restricciones sobre las cuales están fundamentados y
delimitados. Por lo que no se producen resultados ``erróneos'' o no válidos.

Por otro lado, la herramienta no es capaz de detectar repeticiones entre
las respuestas proporcionadas. Así que algunas respuestas tienen más probabilidad de 
aparecer que otras en el proceso de selección aleatoria.

%----------------------------------------------------------------------------------------%

\chapter{Optimización}
\label{optimizaciones}
Durante el proceso de compilación hay varios momentos en los que se pueden 
llevar a cabo algunos procesos que permitan optimizar algunas partes y conseguir
mejoras en el tiempo de ejecución y en el espacio utilizado tanto en memoria 
como en disco.
Aunque muchas de estas decisiones son automatizadas, hay otras que dependen de 
las necesidades del usuario y de su conocimiento sobre el uso de las soluciones 
que desea generar.

\section{Optimización Manual o dependiente}
En esta parte se analizarán las decisiones que puede tomar el usuario 
(programador) que permiten conseguir mayor velocidad en los cálculos o menor uso
de memoria.

\subsection{A nivel del Parser o reconocedor sintáctico}\label {imp_y_res:parser}
A nivel de el reconocimiento sintáctico hay muy pocas decisiones que puede tomar un programador para 
optimizar el funcionamiento del mismo o el compilador. Esto, obviando claro la 
fuerte dependencia que existe entre los recursos utilizados, con 
respecto al nivel de complejidad del modelo de entrada.

La cantidad de variables y la cantidad de valores que puede tener inicialmente 
cada una, afecta proporcionalmente la cantidad de memoria utilizada por el 
programa. Evidentemente las restricciones de dominio de la variable disminuyen 
la cantidad de valores que puede tomar.

Por otro lado la cantidad de restricciones de rango que involucren mismos 
sistemas de variables, incrementan considerablemente la complejidad de los 
cálculos necesarios, para determinar los rangos de solución de los sistemas.

\subsection{Partes de compilación}\label {imp_y_res:comp}
Las partes en la compilación están diseñadas para permitir segmentar el proceso y
hacer cada parte independiente del resto. Esto permite interrumpir la 
compilación y concluirla más adelante desde la última parte completada.

Mediante un flag en el comando de compilación es posible especificar hasta que 
parte de la compilación se desea llegar y luego con otro flag se puede continuar 
desde este punto sin tener que repetir todas las partes anteriores. A está opción
que continua el proceso basado en los archivos generados por el compilador es llamado instanciador y 
tiene la particularidad de que no conserva los archivos que genera luego de 
obtener las soluciones finales.

Dependiendo del tipo de instancia a conseguir, el problema a resolver o contexto
donde se solicita el uso de la herramienta, es más conveniente compilar hasta 
cierta parte:

\subsubsection{Luego del reconocimiento sintáctico}
Compilar hasta esta parte es recomendable solo en un caso: 

\begin{itemize}
\item {Para probar la correctitud de la entrada: se puede saber si tiene
errores sin la necesidad de obtener resultados.}
\end{itemize} 
 
\subsubsection{Luego de la primera parte}
Hasta este punto solo se han generado la estructura de salida y el grafo:

\begin{itemize}
\item {Para problemas pequeños en donde no se quiera almacenar datos: si un 
problema tiene variables con una gran cantidad de valores posibles en su dominio
y tiene restricciones de considerable complejidad, entonces es probable que no
se desee almacenar ningún resultado intermedio de los cálculos.} 
\end{itemize} 


\subsubsection{Luego de la segunda parte}
Hasta este punto se conserva: la estructura de salida, el grafo y los archivos
de descripción de dominio para cada variable. Esta optimización es preferible para el 
siguiente caso:

\begin{itemize}
\item {Para problemas con dominios particularmente difíciles de resolver:
de este modo el calculo queda almacenado para no repetirlo en las próximas 
corridas.} 
\end{itemize}


\subsubsection{Luego de la tercera parte}
Hasta este punto se conservan: la estructura de salida y los archivos
de descripción de rango para cada sistema de variables. Esta optimización es preferible para los 
siguientes casos:

\begin{itemize}
\item {Para problemas con rangos difíciles de resolver:
de la misma forma que con el caso anterior el cálculo de los rangos quedan
almacenados para no repetirlos luego.} 

\item{Para problemas con rangos pequeños: aún si los dominios son muy
permisivos, si las restricciones de rango limitan mucho las soluciones
de los sistemas de variables, conviene almacenar estas soluciones.}
\end{itemize}

Esta parte es la considerada predilecta para problemas con muchas soluciones 
finales, ya que almacena las soluciones de los sistemas antes de ser
multiplicados, acción que indiscutiblemente requerirá mucho más espacio de 
almacenamiento.

\subsubsection{Luego de la cuarta parte}
Hasta este punto se conservan: la estructura de salida y los archivos con las 
soluciones finales sin formato. Esta optimización es preferible para el 
siguiente caso:

\begin{itemize}
\item {Cuando se necesita obtener resultados rápidamente: luego de esta parte
los resultados ya están calculados y solo falta darles formato para que
sean interpretados por el destino. Dejar la compilación hasta este punto es 
favorable si no importa ocupar mucho espacio almacenando estos resultados,
a cambio de obtenerlos rápidamente.} 
\end{itemize}

\subsubsection{Luego de la quinta parte}
Hasta este punto se cuenta ya con todos los archivos solución del problema.

\begin{itemize}
\item {Luego de esta parte no hace falta ejecutar el instanciador ya que las 
soluciones ya fueron generadas, lo que resta es elegir mediante algún
método la solución deseada e interpretarla.}

\item{Este caso es preferible si se desea contar directamente con las 
soluciones y no ejecutar la herramienta. Esto es considerablemente útil si 
se requieren los problemas durante la ejecución de otro programa y no se
quiere que el cálculo de estas soluciones repercuta en el uso de recursos
para el cálculo.}

\item{Esta forma de compilación calcula todos los resultados y almacena cada
uno en un archivo con un formato genérico. Esto representa en la mayoría de los
casos un gasto grande de memoria. Por esto, este caso no debe tomarse a la ligera
y solo ser usado si realmente es preferible gastar espacio de almacenamiento
en vez de hacer los cálculos para hallar las soluciones. O si se sabe que se van
a necesitar muchas soluciones en diferentes momentos y no vale la pena 
calcular las soluciones una y otra vez.} 
\end{itemize}

\section{Optimización automatizada}
Se diseñaron varios mecanismos para la optimización a tiempo de corrida que
no dependieran de el uso explícito de opciones por parte del usuario. Estos
mecanismos no fueron implementados por motivos de complejidad y de tiempo. Sin
embargo, se considera de fundamental importancia explicarlos para que sean
implementados en un futuro y mejoren en desempeño de la herramienta lo más que
se pueda. Sin más que decir se presentan a continuación los mecanismos de optimización
automáticos que se diseñaron:

\subsection{Diferenciación de tipos de problemas}
Se han catalogado los problemas de entrada en dos tipos, esta distinción se
basa en la forma en que están estructuradas las entradas y la facilidad que 
ofrecen para ser resueltos por un método u otro. Antes de continuar sería 
conveniente explicar estos métodos de resolución:

\subsubsection{Método de resolución Algebraico}
Es posible obtener los valores solución a un sistema de variables
planteando, tratando este como un sistema de ecuaciones tradicional y resolviéndolo
por métodos numéricos. Dado que a nivel de \emph{Solver} todas las variables
son manejadas como flotantes y enteros es viable conseguir los vectores
solución de estos sistemas para cualquier sistema consistente.

Además este método tiene como ventaja la posibilidad de separar
completamente el cálculo del dominio de las variables respecto al de rango
de los sub-sistemas. Por ende es posible calcular los dominios o
los rangos en cualquier orden, incluso simultáneamente, de forma tal que
se puedan intercambiar las partes dos (sección \ref{solver:dom}) y tres 
(sección \ref{solver:cal_rangos}) dependiendo de la forma del problema.

\subsubsection{Método de resolución por \emph{Backtracking}}
A tiempo de publicación de este documento el método por \emph{backtracking}
es el único completamente implementado y por lo tanto el mecanismo
por defecto utilizado para conseguir soluciones a los sistemas de variables.
Este se basa en la utilización del método algorítmico de \emph{backtracking} para
buscar todas las posibles soluciones a la asignación de valores de para
variables.

Es sabido que este método tiene numerosas desventajas en cuanto a eficiencia
se refiere, en especial por que recorre todo el espacio de búsqueda para
conseguir soluciones viables y no es capaz de distinguir entre caminos viables
de los que no. Aún así y dado a la genericidad de la herramienta y de los
posibles problemas proporcionados, se considera que es una opción por defecto
bastante aceptable. Adicionalmente, este método puede ser mejorado
y convertirse en un \emph{backtracking} no cronológico \cite{RN09} (si el problema
en cuestión lo permite).

\subsection{Método de búsqueda especializado o híbrido}
Sería recomendable establecer un sistema que discrimine los tipos de
problema y basado en esto, utilice ciertas heurísticas para mejorar
el desempeño del cálculo, incluso es posible utilizar métodos de
aprendizaje de máquina para reconocer patrones y asignarlos a las mejores
formas de resolución para los mismos \cite{M97}.

Existen varias herramientas de software que se especializan en la resolución
de problemas matemáticos pero que también permiten incluir cálculos para
problemas lógicos y/o alfabéticos. Varios incluso, permiten especificar
los sistemas basados en satisfacción de restricciones. El problema, es que 
estas herramientas no calculan todas las soluciones posibles y solo
consiguen una de ellas o que solo calculan soluciones basadas en las relaciones
entre variables y no considerando los dominios.

Estos pasos faltantes pueden ser solucionados por esta herramienta, haciendo
que se calculen todas las soluciones posibles (o el número solicitado) y 
haciendo un filtrado de los vectores solución basándose en las restricciones
de dominio. Este planteamiento se comenzó a trabajar durante la elaboración
de la herramienta. Sin embargo, se decidió no continuarlo porque implicaba 
modificar el software de resolución utilizado, llegando entonces a un nivel 
de complejidad muy grande y al mismo tiempo enfocando demasiado tiempo en un
tema ajeno al objetivo principal del trabajo de investigación.

\subsection{Decisión sobre el uso del método de resolución}
Podría considerarse que un método es siempre mejor que otro, pero eso es 
totalmente relativo al tipo de problema que se resuelve. Entonces, lo más prudente sería
poder discernir entre uno y otro tomando en cuenta las características 
específicas del problema a resolver y posiblemente las experiencias
previas que se tengan para estas características específicas.

Considerando lo tratado en la sección \ref{sis_variables} es claro que cada sistema de
variables es independiente y por lo tanto su resolución es indiferente del
resto de los sistemas, por esto, cada sistema de variables puede resolverse con
un método de resolución indiferentemente del resto. Haciendo entonces
la decisión de elegir un método u otro un problema particular de cada
sistema y no del problema de entrada en general.

Entonces sería recomendable contar con múltiples métodos de resolución
en vez de uno solo y poder utilizar uno u otro dependiendo de la entrada
y más específicamente las propiedades de cada sistema de variables en
particular.

Dado que actualmente solo está implementado el sistema de resolución por
\emph{backtracking} utilizando \emph{Prolog}, no es necesario decidir entre métodos de 
resolución. Ahora bien lo ideal sería contar con una gran cantidad de métodos
para que pueda optimizarse la resolución de cualquier sistema.

Una vez más es notoria la importancia de establecer un sistema que se
encargue de asignar automáticamente un método de resolución a cada sistema 
de variables. Idealmente un sistema que implemente aprendizaje de máquinas
para que se puedan abarcar gran cantidad de posibilidades y pueda adaptarse a
las necesidades del usuario.

\subsection{Recomendaciones para el diseño del sistema de decisión} 
Hay una serie de consideraciones que deben tomarse en cuenta, para diseñar e implementar
el sistema que distingue que método utilizar a la hora de resolver cada 
sistema de variables.

\subsubsection{Comparación de métodos de resolución}
Cada uno de los métodos de resolución tiene sus ventajas y desventajas
en general o frente a algún tipo de sistema en particular:

El método por \emph{backtracking} evalúa todas las posibles asignaciones de valores
del vector solución y mantiene solo las que satisfagan las restricciones de
rango. Tiene como ventaja que no depende de la complejidad de la
interrelación de las restricciones de rango. Esto, por otro lado es una desventaja si el dominio es muy grande
y las restricciones de rango son relativamente simples.

Mientras tanto, el método numérico resulta ser más efectivo para problemas en
los que los tamaños de los dominios sean grandes y el sistema de ecuaciones
resultante de las restricciones de rango sea relativamente complejo. Esto es
debido a que en este caso la obtención de valores posibles de rangos es
primeramente independiente de los dominios, es decir, se obtienen directamente
vectores solución y luego se filtran los verdaderamente válidos según las
posibilidades de asignación por los dominios. Entonces no hace falta revisar
cada valor de los dominios cruzándose con los demás para calcular los rangos.
Esto incide en disminuir de forma exponencial la cantidad de operaciones que
se deben realizar (esto comparando con el método de \emph{backtracking} con un
sistema que tenga gran cantidad de posibles dominios y muchas variables).

La eficiencia de un sistema híbrido depende del método utilizado para hallar
las soluciones, por ende no es posible generalizar los beneficios o desventajas
de este tipo de método de resolución.

Dado esto último por los momentos solo es posible hacer una comparación entre
el método numérico contra el de \emph{backtracking}. Para esto, es necesario 
considerar ciertos factores. Se describirán a continuación los que
fueron considerados como más importantes:

\textbf{Amplitud:} Evalúa la cantidad de dominio total multiplicando todos los
espacios de dominio de las variables. Esta medida permite saber cuan poco 
preferible es utilizar \emph{backtracking} para resolver el problema, ya que mientras
mayor sea el dominio total de las variables mucho mayor será la cantidad de
operaciones necesarias para llevar a cabo el \emph{backtracking}.

\textbf{Disipación:} Evalúa la cantidad promedio de variables involucradas en
las restricciones de rango. Este indicador permite aportar información
referente a la complejidad del sistema de ecuaciones, y por lo tanto, lo
preferible que resultaría resolver el sistema mediante métodos numéricos.

\textbf{Proporción de limitación:} Indica la proporción entre la cantidad
de variables en un sistema, en relación a la cantidad de restricciones
de rango que este contiene. Permite al igual que el factor de disipación tener
información sobre la complejidad del sistema de ecuaciones.

En general si el factor de amplitud es muy grande en relación a la facilidad
de resolver el sistema por método numérico, entonces no vale la pena utilizar
el método por \emph{backtracking}. Ahora bien, si la cantidad de dominios entre las 
variables es pequeño, indiferentemente de la complejidad del sistema de 
restricciones, es preferible utilizar el método de \emph{backtracking}.
