\chapter{Definición de lenguaje}\label{chapter:def_lenguaje}

\textbf{Amorfinator} es un lenguaje desarrollado con la finalidad de facilitar la 
generación de instancias aleatorias de estructuras de datos. Por ende, está 
diseñado para proporcionar a sus usuarios una sintaxis intuitiva que permita 
especificar la estructura final deseada, así como las funciones para verificar 
u obtener valores y también las estructuras auxiliares que sirven como 
herramientas para obtener los resultados finales.

La sintaxis fue diseñada para aceptar palabras reservadas escritas tanto en
ingles como en español, incluso en ambos idiomas al mismo tiempo.

%----------------------------- Estructura general -----------------------------%
\section{Estructura general}
La estructura general se basa en la idea de centrar toda la importancia en una 
única estructura de salida, que contendrá la definición del objeto deseado a 
generar. Para facilitar esto, se decidió diseñar una estructuración por bloques 
que separen la salida final de otras instancias auxiliares y funciones.
Estos bloques son los siguientes:

%%---------------------------- Estructura de salida --------------------------%%
\subsection{Estructura de la salida}
Esta es la estructura principal que describe cómo se quiere recibir el objeto 
generado por el lenguaje. Este bloque comienza con la palabra \texttt{salida} 
(\texttt{exit}) seguido del nombre de la estructura y luego un bloque con el contenido. 
Este bloque está entre un conjunto de llaves (\texttt{$\lbrace$} y \texttt{$\rbrace$}) y está 
formado por dos partes:
%%%------------------------------- Descripción ------------------------------%%%
\subsubsection{Descripción}
Esta parte contiene todos los atributos de los objetos (variables), sus valores o 
``comportamientos''. De esto último se explicará en la sección de variables \ref{def_variables}. 

\textbf{Sintaxis:} Su sintaxis es \texttt{descripcion} (\texttt{description}) seguido del 
bloque con las variables.

\textbf{Ejemplo:}
\begin{figure}[h]
\begin{lstlisting}[mathescape]
descripcion{
  int a0 $\sim$ {
    # > 0;
    # < 25;
    # % 2 == 0
  };
  int a1;
  list<int> l0;
}
\end{lstlisting}
\caption[Ejemplo de descripción de objeto]
{Ejemplo de descripción de objeto}
\label{ejemplo_de_descripcion}
\end{figure}

El objeto de la figura \ref{ejemplo_de_descripcion} tiene:
\begin{itemize}
 \item {Un entero \texttt{a0} que es positivo, menor que \texttt{25} y es par.}
 \item {Un entero cualquiera \texttt{a1}.}
 \item {Una lista de enteros \texttt{l0} con una cantidad aleatoria de elementos.}
\end{itemize}

%%%------------------------- Bloque de Restricciones ------------------------%%%
\subsubsection{Bloque de restricciones} \label{subsec:restricciones}
Es el bloque que contiene el conjunto de restricciones para los atributos del 
objeto. En esta parte, es posible hacer referencia a varias variables del objeto 
en una restricción y restringir el comportamiento de las estructuras de datos no
básicos del objeto.	

\textbf{Sintaxis:} Su sintaxis es \texttt{restriccion} (\texttt{restriction}) seguido del 
bloque con las restricciones. Este bloque puede ser especificado como una 
expresión booleana en\emph{ sintaxis por delimitación} (\emph{la sintaxis por delimitación}
se explica más adelante en la sección \ref{sint_delimitacion}).

\textbf{Ejemplo:}
\begin{figure}[h]
\begin{lstlisting}[mathescape]
restriccion {
  a > b + length(l0);
  c $\sim$ normal(a,b);
}
\end{lstlisting}
\caption[Ejemplo de bloque de restricciones]
{Ejemplo de bloque de restricciones}
\label{ejemplo_de_bloque_de_restricciones}
\end{figure}

Este bloque de restricciones de la figura \ref{ejemplo_de_bloque_de_restricciones} 
significa que se quiere que el valor de $a$ debe ser mayor que la suma del valor 
de \texttt{b} y el largo de la lista \texttt{l0}. Además se indica que se quiere que el 
valor de \texttt{c} tenga un valor aleatorio pero que este sea obtenido aleatoriamente
bajo una distribución normal.


\label{sint_delimitacion}
Para facilitar la lectura y edición de expresiones booleanas de gran tamaño. Se ha 
diseñado un tipo de sintaxis llamada \emph{Sintaxis por Delimitación}. Esta se 
basa en agrupar las expresiones que operan todas bajo conjunciones incluyéndolas 
en un bloque de \texttt{$\lbrace$ $\rbrace$} y las disyunciones entre \texttt{[ ]} y 
separadas por \texttt{;}.

%%--------------------------- Estructura auxiliares --------------------------%%
\subsection{Estructuras auxiliares}
Sirven para especificar estructuras que no serán retornadas por el lenguaje pero 
son de utilidad para realizar los cálculos necesarios para obtener los valores 
finales. Además también pueden formar parte del objeto de salida. 

\textbf{Sintaxis:} Su sintaxis es similar a la de la estructura de salida y se 
diferencian en que en vez de tener como etiqueta \texttt{salida} (\texttt{exit}) tiene la
etiqueta \texttt{auxiliar} (\texttt{aux}).

\textbf{Ejemplo:}
\begin{figure}[h]
\begin{lstlisting}[mathescape]
aux bar {
  descripcion {
    int a2;
    string a3;
  }
}
\end{lstlisting}
\caption[Ejemplo de estructuras auxiliares]
{Ejemplo de estructuras auxiliares}
\label{ejemplo_de_estructuras_auxiliares}
\end{figure}

Lo que quiere decir la figura \ref{ejemplo_de_estructuras_auxiliares} es que los 
objetos que sean del tipo \texttt{bar} tienen un entero \texttt{a2} y un string 
\texttt{a3} aleatorios.

%%--------------------------------- Funciones --------------------------------%%
\subsection{Funciones}
En este bloque se especifica el conjunto de funciones que son definidas por el 
usuario para realizar cálculos y minimizar la repetición de código. Dentro de una 
función se puede usar variables aleatorias usando la sintaxis del lenguaje.

\textbf{Sintaxis:} Para especificar un bloque de funciones se debe usar la etiqueta 
\texttt{funcion} (\texttt{function}) y luego dentro de unas llaves \texttt{$\{$ $\}$} colocar 
las funciones separadas al final por un \texttt{;}. Una función dentro del bloque de 
funciones tiene la siguiente estructura: 

\begin{figure}[h]
\begin{lstlisting}[mathescape]
$<tipo> <nombre\_funcion> ([<parametros> (,<parametros>)*]) $
$[{<variables\_aleatorias>}] = $
$<expresion>$ 
$|$ 
$if (<expresion\_booleana>) then <expresion> $ 
$[elseif (<expresion\_booleana>) <expresion>]+$ 
$else <expresion>$ 
\end{lstlisting}
\caption[Sintaxis de funciones]
{Sintaxis de funciones}
\label{sintaxis_funciones}
\end{figure}

Donde cada uno de las declaraciones en \ref{sintaxis_funciones} significa:
\begin{itemize}
 \item {\textbf{tipo:} Es el tipo devuelto por la función, solo soporta tipos
  básicos.}
 \item {\textbf{nombre\_funcion:} Es el nombre de la función.}
 \item {\textbf{parametros:} Son el tipo y nombre de la variable que recibe la 
  función.}
 \item {\textbf{variables\_aleatorias:} Son el conjunto de variables aleatorias 
  que se desean tener en la función.}
 \item {\textbf{expresion:} Es la expresión que devuelve la función cuyo tipo 
  debe ser el mismo que el de la función.}
 \item {\textbf{expresion\_booleana:} Es una expresión que devuelve un valor 
  booleano siempre.}
\end{itemize}

\textbf{Ejemplo:}
\begin{figure}[h]
\begin{lstlisting}[mathescape]
funciones {
  bool par (int i) = if (i % 2 == 0) then true else false
}
\end{lstlisting}
\caption[Ejemplo de funciones]
{Ejemplo de funciones}
\label{ejemplo_funciones}
\end{figure}

Aquí en la figura \ref{ejemplo_funciones} se tiene la especificación de una 
función que recibe un número y verifica si es par, en caso de serlo devuelve 
\texttt{true} y en caso contrario devuelve \texttt{false}.

%------------------------------- Tipos de datos -------------------------------%
\section{Tipos de datos}
Los siguientes son los tipos de datos definidos en \textbf{Amorfinator}:

%%------------------------------ Datos Basicos -------------------------------%%
\subsection{Datos Básicos}

\begin{itemize}
 \item {\textbf{booleano (bool):} Admite solo los valores \texttt{true} y \texttt{false}.}
 \item {\textbf{entero (int):} Admite valores numéricos enteros (Z) entre 
  $-2^{31}$ y $2^{31}$.}
 \item {\textbf{floatante (float):} Admite valores numéricos reales entre -1.18e-38  y 3.4e38.}
 \item {\textbf{caracter (char):} Representa caracteres del alfabeto UTF-8.}
\end{itemize}

%%----------------------------- Datos Complejos ------------------------------%%
\subsection{Datos Complejos}

\begin{itemize}
 \item {\textbf{Double (Double):} Admite valores numéricos reales con mayor 
  precisión y mayor rango de representación. Con números entre -2.23e-308 y 1.79e308.}
 \item {\textbf{Palabra (String):} Representa una cadena de caracteres.}
\end{itemize}

%%--------------------------- Estructuras basicos ----------------------------%%
\subsection{Estructuras Básicas de datos}

\begin{itemize}
 \item {\textbf{Vector2:} Representa una tupla de un mismo tipo de dos elementos.}
 \item {\textbf{Vector3:} Representa una tupla de un mismo tipo de tres elementos.}
 \item {\textbf{Vector4:} Representa una tupla de un mismo tipo de cuatro elementos.}
 \item {\textbf{Lista$<$\emph{Tipo}$>$:} Representa una lista de un mismo tipo de 
  datos. Esta también permite operaciones para los tipos \gls{PILA}, \gls{COLA}
  y \gls{ARREGLO}.}
\end{itemize}

%-------------------------------- Expresiones ---------------------------------%
\section{Expresiones}	
Una expresión es una pequeña estructura que comprende valores y operadores y que 
tiene un valor, este valor puede o no estar determinado al especificar la 
expresión dependiendo de si las variables que la componen están o no instanciadas. 
Permiten definir comportamientos de las variables, relaciones entre varias de 
ellas o simplemente calcular un valor.

\textbf{Sintaxis:} Una expresión puede ser un \emph{string} complejo, una variable, 
acceso a una variable, una operación binaria de dos expresiones, una expresión 
puede estar dentro de uno más paréntesis, puede tener un operador unario, una 
función que devuelva un valor, puede ser un numero, booleano, lista, caracter, 
variable \texttt{\#}, vectores de 2, 3 y 4 dimensiones.

\textbf{Ejemplos:}
\begin{figure}[h]
\begin{lstlisting}[mathescape]
2 + (3 * sqrt(42))
[1,2,3,4]
"Hola Mundo"
length([1,2,3,4])
\end{lstlisting}
\caption[Ejemplos de expresiones]
{Ejemplo de expresiones}
\label{ejemplo_expresiones}
\end{figure}

%%-------------------------- Expresiones Matemáticas -------------------------%%
\subsection{Expresiones Matemáticas}
Son expresiones compuestas por varios operadores matemáticos de tipo operación y
uno de tipo comparación. Es el equivalente a representar una ecuación algebraica.
En este tipo de ecuaciones es posible utilizar variables o funciones que 
representen valores numéricos y el resultado de esta expresión es una expresión 
lógica que es la evaluación de el operador de comparación.

\textbf{Sintaxis:} La forma en que se escriben las expresiones matemáticas no es 
diferente a la forma en que se escriben en otros lenguajes de programación, 
simplemente se basa en un conjunto de valores que operan entre ellos mediante un 
conjunto de operadores matemáticos, algunos valores son desconocidos y son 
representados como variables.

\textbf{Ejemplos:}
\begin{itemize}
 \item{\textbf{5 $+$ 3 $>$ 2} En este caso el operador de comparación es el \texttt{$>$} y 
  el resultado de evaluar esta expresión es \texttt{true} ya que el lado \texttt{5$+$3} da como
  resultado \texttt{8} y es mayor que \texttt{2}.}

 \item{\textbf{x == 34 / 2} Para esta ecuación el operador de comparación sería 
  el signo \texttt{==}, dado que existe una ecuación y solo una variable, es posible 
  obtener un valor único para la misma. En este caso el valor de \texttt{x} sería \texttt{17}. 
  Luego de obtener el valor, la ecuación se evaluaría como cierta (\texttt{true}). Si 
  en el caso contrario no pudiese asignarse un valor a una variable porque todas 
  las asignaciones contradicen otra expresión o restricción, entonces la 
  expresión no podría resolverse y la solución general que se estaría verificando
  en el momento sería descartada. Existe también la posibilidad de que esta 
  expresión esté bajo un contexto de negación, en cuyo caso la ecuación completa
  deberá retornar \texttt{false} y por ende el valor de \texttt{x} estaría siendo obligado a 
  ser diferente de \texttt{14}. Esto sería equivalente a haber planteado la ecuación de 
  la siguiente forma: \texttt{x $!=$ 34 $/$ 2}.}	

 \item{\textbf{2x - y == 2} En este ejemplo el operador de comparación que también 
  se podría considerar principal es el signo \texttt{==}. En este caso la primera parte 
  de la ecuación está formada por una operación con las variables \texttt{$x \& y$}. 
  Esta ecuación no puede tener un valor booleano final hasta que se hayan 
  conseguido los valores de ambas variables involucradas. Si se consiguiese el 
  valor de una de ambas variables sería posible conseguir el valor de la otra 
  utilizando un despeje por métodos matemáticos. De lo contrario los posibles 
  valores soluciones de ambas variables formarían un conjunto infinito.}
\end{itemize}

%%--------------------------- Expresiones Booleanas --------------------------%%
\subsection{Expresiones Booleanas}
Las expresiones booleanas representan un argumento lógico cuyos valores finales 
posibles son \texttt{cierto} o \texttt{falso} (\texttt{true} o \texttt{false}) y están compuestas por sub expresiones que
devuelven un valor boleeano unidas por operadores unarios y binarios booleanos o 
funciones. Una expresión booleana podría estar fácilmente compuesta por expresiones 
matemáticas enlazadas por operadores lógicos. Un ejemplo de esto sería como el 
mostrado en la figura \ref{ejemplo_expresiones_booleanas}.

\begin{figure}[h]
\begin{lstlisting}[mathescape]
(x + z == 10 && x > 5 && x - z == 4) || (x == 4 && z == 2)
\end{lstlisting}
\caption[Ejemplos de expresiones booleanas]
{Ejemplo de expresiones booleanas}
\label{ejemplo_expresiones_booleanas}
\end{figure}

Para este caso una sola expresión booleana permite declarar dos sistemas de 
ecuaciones en los que solo hay una asignación posible para cada par de variables. 
En la primera parte el único valor para que la expresión puede resultar cierta es
\texttt{x $=$ 7} y \texttt{z $=$ 3}, mientras que para la segunda los valores son evidentes. Ambos 
casos tienen la misma prioridad para ser considerados y dado que los valores 
posibles dan conjuntos disjuntos se podría obtener cualquiera de las soluciones 
pero no ambas al mismo tiempo.

Existe otra sintaxis creada para facilitar la escritura de las estructuras 
booleanas y es la que se trata en la sección \ref{subsec:restricciones}.

%---------------------------------- Variables ---------------------------------%
\section{Variables}
\label{def_variables}
Las variables se asemejan a las existentes en C++, con la diferencia de que 
existe la posibilidad de especificar las restricciones sobre ellas. Para esto
se crearon 4 formas posibles de declarar una variable:

\begin{enumerate}
\item {Variable de valor constante, es decir una variable a la cual se le asigna 
  el valor desde su declaración, usando el símbolo \texttt{$=$}.}
\item {Variable aleatoria por defecto, los detalles se encuentran en la sección 
  \ref{var:aleatoria_por_defecto}.}
\item {Variables con restricciones propias, en este caso se usa el símbolo 
  \texttt{$\sim$}, los detalles de las restricciones se tratan en la sección 
  \ref{var:restricciones_propias}.}
\item {Variables por elección, este último usa los signos unidos \texttt{$=\sim$}
  para poder expresar que la variable toma uno de los valores contenidos dentro
  de la lista. Los detalles se encuentran en la sección \ref{var:eleccion}.}
\end{enumerate}

%%---------------------- Variables aleatoria por defecto ---------------------%%
\subsection{Variable aleatoria por defecto}
\label{var:aleatoria_por_defecto}
Esta forma de definición consiste en declarar una variable que tomará un valor 
de su tipo correspondiente de forma pseudoaleatoria por el lenguaje.
	
\textbf{Sintaxis:} La forma de definir una variable aleatoria por defecto es 
simplemente declarar una variable sin asignarle ningún valor. 
	
\textbf{Ejemplos:}
\begin{figure}[h]
\begin{lstlisting}[mathescape]
int numero;
char letra;
Vector2 coordenadas;
\end{lstlisting}
\caption[Ejemplo de variable aleatoria por defecto]
{Ejemplo de variable aleatoria por defecto}
\label{ejemplo_variable_aleatoria_por_defecto}
\end{figure}

En cualquiera de estos casos de la figura \ref{ejemplo_variable_aleatoria_por_defecto} 
la variable recibirá un valor aleatorio siempre que no exista una restricción que 
la condicione o que la relacione con otro valor.
En el caso de no estar asignada y no estar restringida una variable obtendrá un 
valor aleatorio obtenido automáticamente por el lenguaje según sus propias librerías.
		
%%-------------------- Variables con restricciones propias -------------------%%
\subsection{Variables con restricciones propias}
\label{var:restricciones_propias}
Esta forma de definición consiste en declarar una variable sin asignarle un valor
al igual que en la forma anterior, pero restringiendo las posibles asignaciones 
de valores tanto como se quiera.

\textbf{Sintaxis:} La sintaxis de este tipo de definición variable necesita primero 
  la especificación del tipo de variable, luego se define el nombre de la variable 
  y finalmente se define un conjunto de restricciones que se quieren para esta 
  variable. Este bloque se especifica así:
	\begin{enumerate}
    \item {\textbf{$\sim$:} Esto representa que la variable no esta 
      instanciada sino que es aleatoria y tiene restricciones propias.}
    \item {Restricción: Es una expresión booleana o una función con retorno 
      booleano. Solo puede involucrar a la misma variable que es la que se está 
      definiendo. Para facilitar la escritura en este contexto se puede referir 
      a esta variable con el símbolo \texttt{\#}. En el caso de querer utilizar varias
      restricciones se abre un bloque definido por los símbolos \texttt{$\lbrace$} 
      y  \texttt{$\rbrace$} y las restricciones terminan con \texttt{;} que sirve 
      también a modo de separación.}
	\end{enumerate}

\textbf{Ejemplo:}

\begin{figure}\begin{lstlisting}[mathescape]
int par $\sim${ 
  # >= 0;
  # <= 30;
  # % 2 == 0;
};
\end{lstlisting}
\caption[Ejemplo de variable con restricciones propias]
{Ejemplo de variable aleatoria con restricciones propias}
\label{ejemplo_variable_aleatoria_con_restricciones_propias}
\end{figure}

En la figura \ref{ejemplo_variable_aleatoria_con_restricciones_propias} la variable 
 solo puede ser mayor o igual a \texttt{0}, menor que \texttt{30} y par.

%%-------------------------- Variables por elección --------------------------%%
\subsection{Variables por elección}\label{var:eleccion}
Estas variables no tienen un valor fijo asignado, y solo puede obtener alguno de 
los valores especificados.
	
\textbf{Sintaxis:} La forma de definir una elección es declarar un bloque de 
opciones. Esto se hace incluyendo dentro de un par de \texttt{[ ]} el conjunto de 
valores que puede tomar a variable separados por el símbolo \texttt{$\mid$}.
Es posible especificar el porcentaje de probabilidad con el que se quiere que
sea elegido el valor de entre los posibles valores del conjunto, esto se realiza 
especificando un número que equivale al porcentaje asignado al un valor, este 
porcentaje se escribe luego del valor y separado por espacios o tabulaciones.
	
\textbf{Ejemplos:}
\begin{figure}[h]
\begin{lstlisting}[mathescape]
int foo =$\sim$ [ 4| 8| 15| 16| 23| 42 ];
Caracteres nombre =$\sim$ [ "Miguele"| "Jose"| "Kratrin"| "Juan" ];
Str clasifica =$\sim$[ "Excelente" 20%| "Bueno" 50%| "Malo" 29.5%| "Pesimo" ];
\end{lstlisting}
\caption[Ejemplo de variable por elección]
{Ejemplo de variable por elección}
\label{ejemplo_variable_por_eleccion}
\end{figure}

Como se puede observar en la figura \ref{ejemplo_variable_por_eleccion} en el 
último ejemplo es posible asignar un valor que represente el porcentaje de 
probabilidad que tiene una opción de ser elegida. 
En el caso de que la suma de todas las opciones de un mismo bloque
sea menor a \texttt{100} también lanzará un error.

%-------------------------------- Restricciones -------------------------------%
\section{Restricciones}
Son los elementos que permiten filtrar o especificar el comportamiento o los 
conjuntos posibles de valores que pueden ser asignados a las variables. Existen 
varias formas de especificar  las restricciones:

\begin{itemize}
\item {En primer lugar, hay restricciones que solamente involucran a una sola variable,
mientras que hay otras que relacionan a varias. Para facilitar el orden y 
formato, las que involucran a solo una variable pueden escribirse en el bloque 
de definición de la misma y refiriéndose a esta variable con el signo \texttt{\#}. Estas restricciones 
pueden también escribirse en el área de restricciones generales para la 
estructura si se desea.}
\item {Por otro lado las expresiones que involucran a varias variables o a estructuras 
no básicas deben escribirse necesariamente en el área de restricciones generales.}
\end{itemize}

Las restricciones también pueden clasificarse por el uso, tal y como se muestra
en las siguientes subsecciones.

%%-------------------------- Restricciones booleanas -------------------------%%	
\subsection{Restricciones booleanas}
Estas restricciones vienen siendo expresiones boolenas como las que ya se 
explicaron, que contienen entre sus términos más de una variable. Sirven para 
establecer comportamientos de dependencia entre variables.
	
%%------------------------ Distrubuciones estadísticas -----------------------%%	
\subsection{Distribuciones estadísticas}
Este tipo de restricción permite especificar que el valor de la variable se 
distribuye basado en una distribución estadística ``estándar''. Esto limita a 
que no pueda ser utilizada para verificar un valor, sino para asignar una
instancia del conjunto de valores del dominio de la variable obtenida según la distribución.
	
\textbf{Sintaxis:} La forma de especificar que una variable se debe obtener de una función probabilística 
 es con el símbolo \texttt{\#} (dependiendo del contexto) seguido del símbolo \texttt{$\sim$} y 
luego la función de distribución con sus parámetros correspondientes. Éstas 
funciones de distribución están especificadas en el lenguaje y corresponden a 
las distribuciones usuales como: Distribución normal, exponencial, gamma, etc.
	
\textbf{Ejemplos:}
\begin{figure}[h]
\begin{lstlisting}[mathescape]
nota $\sim$ normal(13,4);
\end{lstlisting}
\caption[Ejemplo de variable restringida por distribución estadística]
{Ejemplo de variable restringida por distribución estadística}
\label{ejemplo_variable_dist_estadistica}
\end{figure}

Este caso de la figura \ref{ejemplo_variable_dist_estadistica} representa una 
variable que se quiere que tenga un valor aleatorio pero que este dentro de los 
valores que comprende una distribución normal.
%%------------------------------ Cuantificadores -----------------------------%%
\subsection{Cuantificadores}
Permite expresar restricciones para los elementos de una lista. 

\textbf{Sintaxis:} Primero se especifica el cuantificador que permite seleccionar 
los elementos de la lista. Entre estos están las opciones:
	
\begin{itemize}
 \item \textbf{para todos (for all)}: Incluye a todos los elementos de la lista.
 \item {\textbf{para al menos (for al least) N}: Incluye al menos N elementos de 
  la lista.}
 \item {\textbf{para a lo sumo (for at most)}: Incluye a lo sumo N elementos de 
  la lista.}
 \item {\textbf{para cualquier (for any)}: Se cumple si para algún elemento de 
  la lista se cumple la proposición.}
 \item {\textbf{para exactamente (for exactly) N}: Se cumple si la proposición 
  se cumple para exactamente N elementos de a lista.}	
\end{itemize}
	
Luego se especifica la variable con la que se referirá a cada elemento de la lista 
y de que lista proviene. Para esto se especifica una variable, luego los signos
\texttt{$<-$} seguidos del nombre de  la variable lista, todo esto dentro de  
paréntesis. Por último se escribe el signo \texttt{$\sim$} y un bloque de 
restricciones.	
	
\textbf{Ejemplos:}
\begin{figure}[h]
\begin{lstlisting}[mathescape]
for all (i <- l0) $\sim$ {
  0 <= i <= 100;
};
for at least 5 (i <- l1) $\sim$ {
  0 <= i <= 100;
};
for exactly 10 (i <- l2) $\sim$ {
  0 <= i <= 100;
};
\end{lstlisting}
\caption[Ejemplos de variable restringida por cuantificador]
{Ejemplos de variable restringida por cuantificador}
\label{ejemplo_variable_por_cuantificador}
\end{figure}

Para cada caso de la figura \ref{ejemplo_variable_por_cuantificador} lo que dice 
cada restricción es:
\begin{itemize}
 \item {Una lista de números llamada \texttt{l0} con todos los valores aleatorios entre \texttt{0} y \texttt{100}.} 
 \item {Una lista de números llamada \texttt{l1} con al menos \texttt{5} de los valores aleatorios  entre \texttt{0} y \texttt{100}.}
 \item {Una lista de números llamada \texttt{l2} con exactamente \texttt{10} de los valores aleatorios entre \texttt{0} y \texttt{100}.}
\end{itemize}
