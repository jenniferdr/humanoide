\chapter{Presentación del problema}\label{chapter:presentacion_del_problema}
¿De qué forma obtener resultados no determinísticos podría traer una ventaja?
Ciertamente si un resultado es completamente aleatorio su utilidad es muy pobre y 
el cálculo sería considerado como un desperdicio. Sin embargo, se pudiera definir
un objeto con ciertas propiedades y limitarlo para que cumpla con ciertas condiciones.
Se tendría entonces una instancia obtenida aleatoriamente de un objeto cuya utilidad sería definida por su especificación y
su constitución estaría bajo las limitantes que se establezcan.

Partiendo de esto, se ha planteado el diseño y desarrollo de una herramienta que se base en
estos principios no deterministas,para generar soluciones variadas
a este tipo de problemas pero que cuyas respuestas se mantengan al margen de la 
especificación proporcionada. Los beneficios de esta herramienta son muchos, entre ellos podría destacarse:

\begin{itemize}
\item {Generación de casos de pruebas.}
\item {Creación y llenado de bases de datos con ejemplos consistentes.}
\item {Resolución de problemas.}
\item {Computación gráfica.}
\item {Computación especializada en creación de simuladores.}
\item {Diseño y creación de videojuegos.}
\end{itemize}

Todos los anteriores son los ejemplos de circunstancias más básicas que podrían atacarse con
esta herramienta. Dicho esto queda más claro que el objetivo general es diseñar e implementar esta 
herramienta. De forma que se pueda generar instancias aleatorias de objetos basadas en una 
definición y cuyos valores satisfagan ciertas condiciones y restricciones definidas como entrada.

\label{sect:objetivos}
Mientras que como objetivos parciales o específicos se postulan los siguientes:
\begin{itemize}
\item {Diseñar y elaborar un lenguaje de programación que facilite especificar un 
modelo a crear, incluyendo su estructuración, así como sus restricciones internas.}
\item Implementar un reconocedor sintáctico escrito en \emph{C++} utilizando las herramientas \emph{Flex} y \emph{Bison}.
\item Elaborar un compilador para el lenguaje que resuelva los problemas asociados a restricciones
y asocie los valores posibles a las variables.
\item {Desarrollar un interpretador que sirva como auxiliar al compilador y permita dividir
el proceso en partes para facilitar el cálculo cuando se requiere para generar múltiples
instancias de un mismo objeto.}
\item {Estudiar las posibles formas de optimizar la herramienta para aumentar su velocidad y
mejorar su uso de memoria.}
\item {Investigar y proponer formas de ampliar las funcionalidades de la herramienta y mejorar las
ya existentes de forma que trabajos futuros tengan sólidas referencias y guías.} 
\end{itemize}
