\newglossaryentry{C++}
{
  name=C++,
  description={Es un lenguaje de programaci\'on imperativo y orientado a objetos.}
}
\newglossaryentry{AVR}
{
  name=AVR,
  description={Es una familia de microcontroladores de instrucciones reducidas de la compañía Atmel.}
}

\newglossaryentry{framework}
{
  name=framework,
  description={(Marco de trabajo) Es un conjunto de técnicas, conceptos y estilos de trabajo que se establecen para resolver un problema particular y que sirve de referencia para solucionar problemas similares.}
}

\newglossaryentry{TightVNC}
{
  name=TightVNC,
  description={Es un paquete de software que sirve para controlar la interfaz gráfica  de ordenadores remotos.}
}

\newglossaryentry{IDE}
{
  name=IDE,
  description={(Integrated development environment / Entorno de desarrollo integrado): Es un programa diseñado para facilitar la programación en uno o varios lenguajes. Usualmente incluye herramientas de compilación, editor de textos y depurador.}
}

\newglossaryentry{ROS}
{
  name=ROS,
  description={(Robot Operating System / Sistema de operación para robots ): Es un framework que provee herramientas para ayudar a desarrolladores de aplicaciones para robots.}
}

\newglossaryentry{BSD}
{
  name=Licencia BSD,
  description={(Berkeley Software Distribution / distribución de software berkeley) : Es una licencia para software libre otorgada principalmente a sistemas BSD.}
}

\newglossaryentry{XBEE}
{
  name=XBEE,
  description={Es una familia de módulos de radio, con protocolo de comunicación inalámbrica basado en radio frecuencias.}
}

\newglossaryentry{CSI}
{
  name=CSI,
  description={(Camera Serial Interface / Interfaz serial para cámaras ): Es un estándar que define la interfaz de comunicación entre una cámara y un procesador. Es comúnmente utilizado en dispositivos móviles.}
}
     
\newglossaryentry{MMAL}
{
  name=MMAL,
  description={(Multi-Media Abstraction Layer / Capa de abstracción multimedia): Es una librería que brinda una interfaz de bajo nivel para controlar dispositivos que se ejecutan en el núcleo de video de la Raspberry Pi, como el módulo de cámara.}
}
 
\newglossaryentry{V4L}
{
  name=V4L,
  description={(Video 4 Linux / video para linux): Es una interfaz de programación de video para Linux. Uno de los tipos de dispositivos soportados son las cámaras web USB.}
}

\newglossaryentry{RGB}
{
  name=RGB,
  description={(Red Green Blue / rojo, verde, azul): Es un modelo de color que se basa en la intensidad de los colores primarios de la luz (rojo, verde y azul).}
}
  
\newglossaryentry{HSV}
{
  name=HSV,
  description={(Hue, Saturation, Value / Matiz, Saturación, Valor): Es un modelo de color que se basa en las cualidades de matiz, saturación y valor del color.}
}

\newglossaryentry{HDMI}
{
  name=HDMI,
  description={(High-Definition Multimedia Interface/ interfaz multimedia de alta definición): Es una interfaz para transferir datos de audio y video digital entre un dispositivo y un monitor, proyector, televisor digital o dispositivo de audio digital.}
}

\newglossaryentry{SD}
{
  name=SD,
  description={(Secure Digital / Digital Seguro) : Es un formato de tarjetas de memoria de almacenamiento digital. Existen tarjetas SD con diferentes características en cuanto su clase, capacidad de almacenamiento y tamaño físico.}
}  

\newglossaryentry{LEGO}
{
  name=LEGO,
  description={ Es una serie de juguetes de construcci\'on, que ofrece um kit de materiales para rob\'otica llamada Mindstorms. La serie posee una trajeta programable, sensores y actuadores}
}  

\newglossaryentry{DC}
{
  name=Motor DC,
  description={ Es un motor de corriente directa que transforma la energ\'ia el\'ectrica en mec\'anica generando un movimiento rotatorio}
}  

\newglossaryentry{USB}
{
  name=USB,
  description={(Universal Serial Bus): Es un bus est\'andar de conecci\'on, comunicaci\'on y fuente de poder entre dispositivos electr\'onicos}
}  


\newglossaryentry{UART}
{
  name=UART,
  description={(Universal Asynchronous Receiver/Transmitter): Es dispositivo para la comunicaci\'on est\'andar as\'incrona. }
}  

\newglossaryentry{VGA}
{
  name=VGA,
  description={(Video Graphics Array): Es un conector de video est\'andar, una interfaz de video usada para proyectar video en alta defini\'on }
}  

\newglossaryentry{raspivid}
{
  name=Raspivid,
  description={ Es una herramienta de captura de video por medio de comandos en la Raspeberry Pi}
} 

\newglossaryentry{raspistill}
{
  name=Raspistill,
  description={ Es una herramienta de captura de fotos por l\'inea de comandos en la Raspberry Pi }
} 
