\chapter{Especificación formal de la sintaxis}\label{gramaticas_lenguaje}

Para reconocer el lenguaje propuesto en la sección \ref{chapter:def_lenguaje},
se crearon las gramáticas que están en la sección \ref{gramaticas_lenguaje_detalle}
y los tokens correspondientes a estas últimas se encuentran en la sección 
\ref{tokens_lenguaje_detalle}.

\section{Gramáticas} \label{gramaticas_lenguaje_detalle}

\begin{lstlisting}[mathescape]
principal:
  definicion TKEXIT TKID TKLBRACE descripcion restriccion TKRBRACE definicion

definicion:
  definicion TKAUXILIAR TKID TKLBRACE descripcion restriccion TKRBRACE
  |definicion TKFUNCTIONS TKLBRACE funciones TKRBRACE
  |

descripcion:
  TKDESCRIPTION TKLBRACE declaracion_variable TKRBRACE 

restriccion:
  TKRESTRICTION lista_bloque_restricciones porcentaje 
  |

lista_bloque_restricciones:
  bloque_restricciones
  |lista_bloque_restricciones operador_logico_simple bloque_restricciones 

bloque_restricciones:
  operador_unario bloque_restricciones 
  |TKLBRACE lista_sub_bloque_restricciones TKRBRACE 
  |TKLBRACKET lista_sub_bloque_restricciones TKRBRACKET 

lista_sub_bloque_restricciones:
  lista_sub_bloque_restricciones_operados porcentaje TKSEMICOLON 
  |lista_sub_bloque_restricciones lista_sub_bloque_restricciones_operados porcentaje TKSEMICOLON 

lista_sub_bloque_restricciones_operados:
  sub_bloque_restricciones 
  |lista_sub_bloque_restricciones_operados operador_logico_simple sub_bloque_restricciones 

sub_bloque_restricciones:
  bloque_restricciones 
  |expresion
  |distribucion 
  |cuantificador 
  |operador_unario cuantificador 

declaracion_variable:
  anulable tipo_variable rango_random TKID asignacion TKSEMICOLON 
  |declaracion_variable anulable tipo_variable rango_random TKID asignacion TKSEMICOLON 

rango_random:
  TKLPARENTHESES negativo TKINTVALUE TKCOMMA negativo TKINTVALUE TKRPARENTHESES
  |TKLPARENTHESES negativo TKFLOATVALUE TKCOMMA negativo TKINTVALUE TKRPARENTHESES
  |TKLPARENTHESES negativo TKINTVALUE TKCOMMA negativo TKFLOATVALUE TKRPARENTHESES
  |TKLPARENTHESES negativo TKFLOATVALUE TKCOMMA negativo TKFLOATVALUE TKRPARENTHESES
  |

negativo:
  TKHYPHEN
  |

anulable:
  TKIGNORE  
  |

tipo_variable:
  TKBOOL      
  |TKINT      
  |TKCHAR     
  |TKFLOAT    
  |TKSTRING   
  |TKDOUBLE   
  |TKVECTOR2  
  |TKVECTOR3  
  |TKVECTOR4
  |TKLIST TKLESSTHAN tipo_variable TKMORETHAN
  |TKID
  
asignacion:
  TKEQUAL expresion
  |TKEQUAL TKTILDE TKLBRACKET opciones TKRBRACKET 
  |TKTILDE lista_bloque_restricciones porcentaje  
  |
  
expresion:
  tipos_basicos 
  |variable_mixta 
  |operador_unario expresion
  |expresion operador_binario expresion
  |TKLPARENTHESES expresion TKRPARENTHESES 
  |llamada_funcion 
  |TKLBRACKET elemento_lista TKRBRACKET 
  |TKLBRACKET TKRBRACKET 
  |TKLPARENTHESES expresion TKCOMMA expresion TKRPARENTHESES
  |TKLPARENTHESES expresion TKCOMMA expresion TKCOMMA expresion TKRPARENTHESES
  |TKLPARENTHESES expresion TKCOMMA expresion TKCOMMA expresion TKCOMMA expresion TKRPARENTHESES 
  |TKNULL 
  |expresion operador_logico expresion
  |expresion operador_binario_matematico_logico expresion

variable_mixta:
  variable_acceso 
  |variable_acceso TKLBRACKET expresion TKRBRACKET accesos 
  |TKNUMBERSIGN 
  |TKNUMBERSIGN TKDOT variable_acceso 
  |TKNUMBERSIGN TKLBRACKET expresion TKRBRACKET accesos 
  |TKNUMBERSIGN TKDOT variable_acceso TKLBRACKET expresion TKRBRACKET accesos 

variable_acceso:
  TKID                        
  |variable_acceso TKDOT TKID 

accesos:
  accesos TKDOT variable_acceso TKLBRACKET expresion TKRBRACKET 
  | 

operador_unario:
  TKHYPHEN  
  |TKNEGATE 

operador_binario:
  TKHYPHEN    
  |TKPLUS     
  |TKASTERISK 
  |TKSLASH    
  |TKPERCENT  
  |TKCARET    

opciones:
  tipos_basicos porcentaje 
  |opciones TKBAR tipos_basicos porcentaje 

tipos_basicos:
  TKSTRINGVALUE 
  |TKTRUE       
  |TKFALSE      
  |TKINTVALUE   
  |TKFLOATVALUE 
  |TKDOUBLEVALUE 
  |TKCHARVALUE  

porcentaje:
  TKINTVALUE TKPERCENT 
  |TKFLOATVALUE TKPERCENT 
  | 

llamada_funcion:
  TKID TKLPARENTHESES parametro_funcion TKRPARENTHESES 

parametro_funcion:
  expresion  
  |parametro_funcion TKCOMMA expresion 

operador_logico:
  operador_logico_simple  
  |TKAND                  
  |TKOR                   

operador_logico_simple:
  TKEQUIVALENT            
  |TKIMPLICATION          
  |TKCONSECUENCE          
  |TKDISTINCT             

operador_binario_matematico_logico:
  TKLESSTHAN              
  |TKMORETHAN             
  |TKLESSEQUALTHAN        
  |TKMOREEQUALTHAN        
  |TKEQUAL                

elemento_lista:
  expresion 
  |elemento_lista TKCOMMA expresion 

distribucion:
  variable_mixta TKTILDE llamada_funcion 

cuantificador:
  TKFOR operador_cuantificador TKLPARENTHESES lista_variables TKRPARENTHESES TKTILDE lista_bloque_restricciones

operador_cuantificador:
  TKANY                   
  |TKALL                  
  |TKATMOST TKINTVALUE    
  |TKATLEAST TKINTVALUE   
  |TKEXACTLY TKINTVALUE   

lista_variables:
  TKID TKFROM variable_mixta 
  |lista_variables TKCOMMA TKID TKFROM variable_mixta 

funciones:
  funcion_firma 
  |funciones funcion_firma

funcion_firma:
  tipo_variable TKID TKLPARENTHESES firma_parametros TKRPARENTHESES TKEQUAL variables_aleatorias expresion_funciones

firma_parametros:
  par_tipo_nombre 
  |

par_tipo_nombre:
  tipo_variable TKID 
  |par_tipo_nombre TKCOMMA tipo_variable TKID 

variables_aleatorias:
  TKVARIABLES TKLBRACE declaracion_variable TKRBRACE TKIN 
  | 

expresion_funciones:
  expresion 
  |TKIF TKLPARENTHESES expresion TKRPARENTHESES TKTHEN expresion_funciones else_if TKELSE expresion_funciones 

else_if:
  else_if TKELSEIF TKLPARENTHESES expresion TKRPARENTHESES TKTHEN expresion_funciones 
  | 
\end{lstlisting}

\section{Tokens}\label{tokens_lenguaje_detalle}

Los tokens que se definieron para el lenguaje se clasificaron en 3 grupos.

\begin{table}[h]
\centering
\begin{tabular}{|c|c|c|c|}
\hline
Símbolo & Token & Símbolo & Token \\
\hline
\{     & TKLBRACE &  !     & TKNEGATE  \\
\hline
\}     & TKRBRACE & ;     & TKSEMICOLON \\
\hline
[     & TKLBRACKET & *     & TKASTERISK \\
\hline
]     & TKRBRACKET & $<$     & TKLESSTHAN \\
\hline
(     & TKLPARENTHESES & $>$     & TKMORETHAN \\
\hline
)     & TKRPARENTHESES & =     & TKEQUAL \\
\hline
$\sim$     & TKTILDE & !=   & TKDISTINCT \\
\hline
\end{tabular}
\caption{Tokens de símbolos 1}\label{tab:tok_simb_1}
\end{table}

\begin{table}[h]
\centering
\begin{tabular}{|c|c|c|c|}
\hline
Símbolo & Token & Símbolo & Token \\
\hline
\#     & TKNUMBERSIGN & ,     & TKCOMMA\\
\hline
-     & TKHYPHEN & +     & TKPLUS\\
\hline
/     & TKSLASH  & \%     & TKPERCENT\\
\hline
$\wedge$  & TKCARET & $\mid$ & TKBAR\\
\hline
.     & TKDOT & 	\textdollar  & TKIGNORE \\
\hline
$==$   & TKEQUALEQUAL & === & TKEQUIVALENT \\
\hline
$==>$ & TKIMPLICATION & $<==$ & TKCONSECUENCE\\
\hline
\&\&   & TKAND & $\mid\mid$   & TKOR \\
\hline
$<=$   & TKLESSEQUALTHAN & $>=$   & TKMOREEQUALTHAN \\
\hline
$<-$   & TKFROM & ~ & ~ \\
\hline
\end{tabular}
\caption{Tokens de símbolos 2}\label{tab:tok_simb_2}
\end{table}

\begin{table}[h]
\centering
\begin{tabular}{|c|c|}
\hline
Expresión regular & Token \\
\hline
$0|[1-9][0-9]*$ & TKINTVALUE \\
\hline
$0.[0-9]+|[1-9][0-9]*.[0-9]+$ & TKFLOATVALUE\\
\hline
$0.[0-9]+|[1-9][0-9]*.[0-9]+$ & TKDOUBLEVALUE\\
\hline
[a-zñá-úä-üA-ZÑÁ-ÚÄ-Ü\_]+ & TKID \\
\hline
``.*'' & TKSTRINGVALUE \\
\hline
'.'  & TKCHARVALUE \\
\hline
\end{tabular}
\caption{Tokens de valores}\label{tab:tok_valores}
\end{table}

\begin{table}[h]
\centering
\begin{tabular}{|c|c|c|}
\hline
Expresión regular inglés & Expresión regular español & Token \\
\hline
exit & salida & TKEXIT \\
\hline
functions & funciones & TKFUNCTIONS \\
\hline
aux & auxiliar & TKAUXILIAR \\
\hline
description & descripcion & TKDESCRIPTION \\
\hline
restriction & restriccion & TKRESTRICTION \\
\hline
entero & int    & TKINT \\
\hline
bool  & bool & TKBOOL \\
\hline
char   & caracter & TKCHAR \\
\hline
float  & flotante &  TKFLOAT \\
\hline
Double & Doble  & TKDOUBLE \\
\hline
Str    & Caracteres & TKSTRING \\
\hline
vector2 & vector2 &TKVECTOR2 \\
\hline
vector3 & vector3 &TKVECTOR3 \\
\hline
vector4 & vector4 &TKVECTOR4 \\
\hline
list   & lista  & TKLIST \\
\hline
for    & para   & TKFOR \\
\hline
any    & algun  & TKANY \\
\hline
all    & todos  & TKALL \\
\hline
at\ most & a\ lo\ sumo & TKATMOST \\
\hline
at\ least & al\ menos & TKATLEAST \\
\hline
exactly & exactamente & TKEXACTLY \\
\hline
variables & variables & TKVARIABLES \\
\hline
in     & en     & TKIN \\
\hline
if     & si     & TKIF \\
\hline
then   & entonces & TKTHEN \\
\hline
elseif & si\_en\_vez &  TKELSEIF \\
\hline
else   & si\_no & TKELSE \\
\hline
null   & nulo   & TKNULL \\
\hline
true   & verdadero & TKTRUE \\
\hline
false  & falso  & TKFALSE \\
\hline
\end{tabular}
\caption{Tokens de palabras reservadas}\label{tab:tok_palabras}
\end{table}
