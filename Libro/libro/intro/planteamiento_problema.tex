\chapter{Introducción}

\label{sect:planteamiento}
Pero, ¿de qué forma obtener resultados no determinísticos podría traer una ventaja?.
Ciertamente si un resultado es completamente aleatorio su utilidad es muy pobre y 
el cálculo sería considerado como un desperdicio. Sin embargo, si pudieramos definir
como queremos que sea el resultado y limitarlo para que cumpla con ciertas propiedades
tendríamos entonces un objeto aleatorio cuya utilidad sería definida por su definición y
su constitución estaría bajo las limitantes que se establezcan.

Basado en esto, hemos planteado el diseño y desarrollo de una herramienta que basada en
estos principios no deterministas pero igualmente rigurosos permitan dar soluciones variadas
a este tipo de problemas pero que cuyas respuestas se mantengan al margen de nuestra
especificación

Los beneficios de esta herramienta son muchos, entre ellos podríamos destacar:

Generación de casos de pruebas, creación y llenado de bases de datos con ejemplos consistentes, 
resolución de problemas son los ejemplos de circunstancias más básicas que podrían atacarse con
esta herramienta. De la misma forma hay problemas un poco más complejos y que resultan de más
interés personal para nosotros:

La computación gráfica, la computación especializada en creación de simuladores y videojuegos
son uno de los negocios más prósperos de la actualidad, con una herramienta para generar
instancias de objetos aleatorios podrían crearse videos o juegos que son diferentes cada vez
que se ven o juegan, podrían generarse desde edificios, hasta modelos de personas, todas
diferentes pero respetando las necesidades de el diseñador. Haciendo que la experiencia sea
única para cada persona que haga uso del medio o incluso, para cada vez que se utilice la misma
persona.

Dicho esto queda más claro que nuestro objetivo general sea: diseñar e implementar esta 
herramienta. De forma que podamos generar instancias aleatorias de objetos basadas en una 
definición y cuyos valores satisfagan ciertas condiciones y restricciones definidas como entrada.

\label{sect:objetivos}
Mientras que como objetivos parciales o específicos postulamos los siguientes:
\begin{itemize}
\item {Diseñar y elaborar nuestro propio lenguaje de programación que facilite especificar un 
modelo a crear incluyendo su estructuración así como sus restricciones internas.}
\item Implementar un parser escrito en c++ utilizando las herramientas Flex y Bison
\item Elaborar un compilador para el lenguaje que resuelva los problemas asociados a restricciones
y asocie los valores posibles a las variables.
\item {Desarrollar un interpretador que sirva como auxiliar al compilador y permita dividir
el proceso en partes para facilitar el calculo cuando se requiere para generar múltiples
instancias de un mismo objeto.}
\item {Estudiar las posibles formas de optimizar la herramienta para aumentar su velocidad y
mejorar su uso de memoria.}
\item {Investigar y proponer formas de ampliar las funcionalidades de la herramienta para mejorar las
ya existentes de forma que trabajos futuros tengan sólidas referencias y guías.} 
\end{itemize}