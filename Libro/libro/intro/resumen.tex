\setcounter{page}{4}
\begin{center}
	{\bf Resumen} \pdfbookmark[0]{Resumen}{resumen} % Sets a PDF bookmark for the dedication
\end{center}	

RobotCup \cite{robotcup} es una competencia de fútbol iniciada desde 1997 donde contribuyen las áreas de robótica, investigación e inteligencia artificial. Entre sus categorías se encuentra RobotCup Soccer \cite{robotcupsoccer}, la cual consiste en la participación de pequeños robots humanoides que se enfrentan a otro equipo para jugar fútbol. El objetivo de esta competencia es lograr que en el año 2050 el equipo campeón logre vencer al ganador del año en la copa mundial de la FIFA (International Federation of Association Football).

La finalidad de este proyecto es la construcion de robot un humanoide autonomo e inteligente de tamaño pequeño (38 cm de alto) nombrado Junny , capaz de detectar la ubicación de una pelota y acercarse a ella para patearla con la utilizacion de aprendizaje por reforzamiento. Ha sido construido con las piezas del kit Bioloid Premium \cite{robotics} del fabricante ROBOTIS \cite{robotics1}. Del kit se ha excluido la tarjeta CM-510 para sustituirla por la tarjeta controladora Arbotix, que será la que controle los 16 motores Dynamixel Ax-12+ (para mover al robot) y 2 servomotores analógicos (para mover la cámara). Además se ha agregado un mini computador Raspberry Pi, con su cámara \cite{raspberrycam}, para que el robot pueda detectar y seguir la pelota de forma autónoma. 

Todos estos componentes deben ser coordinados para que se logre cumplir la tarea de detectar, seguir y patear la pelota. Por ello se hace necesaria la comunicación entre la Arbotix y la Raspberry Pi. La herramienta empleada para ello es el framework ROS (Ros Operating System) \cite{ros}.

En la Raspberry Pi se usa el lenguaje C++ y se ejecuta un solo programa encargado de captar la imagen de la cámara, filtrar y procesar para encontrar la pelota, tomar la decisión de la acción a tomar y hacer la petición a la Arbotix para que de la orden a los motores de ejecutar el movimiento. Para captar la imagen de la cámara se ha utilizado la librería raspicam\_cv \cite{camara}. Para filtrar y procesar la imagen se ha usado las librerías OpenCv \cite{opencv}. 

La Arbotix, además de controlar los motores, se encarga de monitorizar que el robot se encuentre balanceado, para ello usa el sensor Gyro de Robotis \cite{gyro}. Si detecta un desbalance de un cierto tamaño puede saber si se ha caído y levantarse. 