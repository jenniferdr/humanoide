\setcounter{page}{4}
\begin{center}
	{\bf Resumen} \pdfbookmark[0]{Resumen}{resumen} % Sets a PDF bookmark for the dedication
\end{center}	

RoboCup \cite{robotcup} es una competencia de fútbol, iniciada en 1997, donde contribuyen las áreas de robótica, investigación e inteligencia artificial. Entre sus categorías se encuentra RoboCup Soccer \cite{robotcupsoccer}, la cual consiste en la participación de pequeños robots humanoides que se enfrentan a otro equipo para jugar fútbol.

En este proyecto se describe la construción de Junny, un robot humanoide autónomo e inteligente de tamaño pequeño (38 cm de alto), capaz de detectar la ubicación de una pelota y acercarse a ella para patearla. Sus metas se enmarcan parcialmente dentro de las reglas de la categoría RoboCup Soccer.

Junny ha sido construido con las piezas del kit Bioloid Premium \cite{robotics} del fabricante ROBOTIS \cite{robotics1}. Del kit se ha excluido la tarjeta CM-510 para sustituirla por la tarjeta controladora Arbotix, que será la que controle los 16 motores Dynamixel Ax-12+ (para mover al robot) y 2 servomotores analógicos (para mover la cámara). Además se ha agregado un mini computador Raspberry Pi, con su cámara \cite{raspberrycam}, para que el robot pueda detectar la posición de la pelota de forma autónoma. 

En la Raspberry Pi se usa el lenguaje C++ y se ejecuta un solo programa encargado de detectar la posición de la pelota y decidir qué movimientos son necesarios para llegar a ella. La manera de elegir las acciones se ha realizado con aprendizaje por reforzamiento. Para captar la imagen de la cámara se ha utilizado la librería raspicam\_cv \cite{camara}. Para filtrar y procesar la imagen se ha usado la segmentaci\'on por regiones para la detecci\'on de la pelota, con ayuda de las librerías OpenCv \cite{opencv}. 

La Arbotix, además de controlar los motores para ejecutar los movimientos deseados, se encarga de monitorizar que el robot se encuentre balanceado, para ello usa el sensor Gyro de Robotis \cite{gyro}. Si detecta un desbalance de un cierto tamaño puede saber si se ha caído y levantarse.

Todos estos componentes deben ser coordinados para que se logre cumplir la tarea de seguir y patear la pelota. Por ello se hizo necesaria la comunicación entre la Arbotix y la Raspberry Pi. La herramienta 
empleada para ello ha sido el framework ROS (Ros Operating System) \cite{ros}. 

Finalmente se obtivieron los siguientes resultados, un robot aut\'onomo e inteligente que es capaz de reconocer la pelota, desplazarse a ella y patearla, con resultados obtenidos, luego del aprendizaje por reforzamiento, con un 100\% de acertividad y con una tasa de eficiencia de $0.72$.

%Palabras claves: Rob\'otica, Apredizaje por reforzamiento, Humanoide, RoboCup, ROS,Arbotix.