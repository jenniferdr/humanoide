
\chapter{Introducción}\label{intro}

\pdfbookmark[0]{Introducción}{introduccion} % Sets a PDF bookmark for the dedication

\label{sect:justificacion}

La inteligencia artificial (IA) y la rob\'otica podr\'ian llegar a dominar el mundo o quiz\'as s\'olo har\'an todo por nosotros, el futuro de estas \'areas es muy incierto, pero lo que si es cierto es que los avances se est\'an dando, los estudios se est\'an haciendo y que cada d\'ia m\'as personas en todo el mundo se familiarizan con IA y rob\'otica.

Se encuentran robots de distintas formas y tama\~nos, que realizan diversas tareas, algunos famosos como NAO \cite{nao}, Darwin \cite{darwin} o  ASIMO \cite{asimo}, el denominador com\'un de estos robots es su forma a semenzanja de un humano por ello son llamados humanoides. En el \'area de robots humanoides hay varios enfoques de estudios, por ejemplo, la interacc\'ion entre humanos y humanoides, en el estudio de (Erhan Oztop 2005) \cite{interacion} donde se realizan experimentos para describir si los factores de contagio y sincronicidad de movimiento en grupos Humano-Humano se puede dar en un grupo Humano-Humanoide, concluyen que la combinaci\'ion de forma y movimiento son factores importantes para las competencias sociales de la interacci\'on Humano - Humanoide. 
Las investigaciones con humanoides tambi\'en pueden ser muy espec\'ificas como el estudio (Makoto Tomimoto 2014)\cite{sudor} que investiga el efecto de la fricci\'on del sudor humano y compara la efectividad del sudor artificial en sensaciones t\'actiles de fricci\'on para dedos de robots humanides. Entre sus conclusiones se encuentra que efectivamente el sudor artificial realiza aumentos favorables en las se\~nales t\'actiles al contacto de distintos materiales.

Para los humanoides el equilibrio y balance al realizar una tarea son de vital importancia, por ello existen numerosas investigaciones que se enfocan en estas habilidades, uno de ellas es la investigaci\'on de (Andrej Gams 2014) \cite{pedal} que dice que para realizar alguna tarea como pedalear o patinar existe una serie de movimientos que poseen un ritmo, se plantean adaptar estas trayectorias r\'itmicas o peri\'odicas en un comportamiento con la fuerza y habilidades motoras correctas en distintas condiciones. Su nuevo enfoque es la utilizaci\'on de \textit{feedback} y \textit{feed-forward}. Sus experimentos son simulados y con un robot real, en ambos casos el mejor resultado fue la utilizaci\'on de \textit{feedback} y \textit{feed-forward} simult\'aneamente. en particular la simulaci\'on obtuvo un mejor desempe\~no que los experimentos en la vida real.

Algunas destrezas de robots con forma de humanos son caminar, percibir el mundo y tomar alguna acción sobre él. Una de las más avanzadas muestras en el área es el robot ASIMO \cite{asimo}, creado por la compañía Honda, cuyos últimos avances incluyen la predicción de trayectoria de objetos para poder esquivarlos.  %asimo y otros como COG del MIT hay muchos (hay trabajos q muestran que la gente interactúa mejor con robots humanoides

RoboCup \cite{robotcup} es una competencia de robótica que ha iniciado en el año 1997 y se ha realizado de forma anual, siendo la más reciente celebrada en Brasil (2014). En ella se busca promover avances en las áreas de robótica, inteligencia artificial, mecatrónica, entre otras. El enfoque principal de la competencia se centra en las cinco categorías agrupadas dentro de la liga RoboCup Soccer \cite{robotcupsoccer}, la cual consiste en la participación de robots humanoides que se enfrentan a otro equipo para jugar fútbol. Aunque existen otras ligas dentro de la RoboCup que no se involucran con el juego de fútbol, la meta final de la competencia es lograr que en el año 2050 el equipo campeón logre vencer al ganador del año en la copa mundial de la FIFA (International Federation of Association Football).

Existen equipos que han participado durante varios años consecutivos en la competencia Robocup, logrando mejoras en sus diseños y técnicas; tal es el caso del equipo MRL que ha participado en los años 2011, 2012, 2013 y 2014 en la categoría “Humanoid League”, han iniciado con el hardware del robot DARwIn-OP y con el tiempo han modificado los componentes electrónicos para agregar eficiencia y estabilidad. Para el balance han utilizado un giroscopio y sensores de aceleración, y para la visión, una cámara conectada por USB al CPU principal \cite{paper1}.

 
%En el desarrollo de habilidades más específicas con respecto a la competencia RobotCup, se han propuesto dos posibles estrategias para el pateo de la pelota donde los factores fundamentales para un buen desempeño es la fuerza y la rapidez con que se patea \cite{paper2}. Las dos estrategias de pateo se ponen en práctica en distintas circunstancias del juego basado en la cinemática y dinámica de equilibrar el cuerpo al momento de patear.

En el desarrollo de esta investigaci\'on se presenta a Junny, un robot humanoide aut\'onomo e inteligente de 38 cm de altura, cuyos objetivos están inspirados en la competencia RoboCup. Se planteó como objetivo principal construir un prototipo que sea capaz de detectar la cercanía de una pelota, acercarse a ella y patearla, reincorpor\'andose a la posici\'on de pie en caso de perder el equilibrio y caer. Se ha tomado en cuenta un conjunto de objetivos específicos para guiar el desarollo de este trabajo. %Para cumplir con este objetivo se ha desglosado un conjunto de objetivos específicos que se describen a continuación: 

Un objetivo espec\'ifico es el diseño y construcción de un humanoide con piezas del kit de robótica Bioloid Premium, sustituyendo su tarjeta controladora CM-510 por la tarjeta de software libre ArbotiX para controlar los motores Dynamixel y otros sensores. Esta tarea a su vez se subdivide en las siguientes asignaciones: Instalación y configuración de la tarjeta ArbotiX, instalación y configuración de la tarjeta Raspberry Pi, instalación y configuración de la cámara Raspberry Pi, instalación de servomotores  para el movimiento de la cámara, instalación del giroscopio Gyro. 
Luego el robot deber\'a detectar la pelota, realizando la captura de la imagen, con la cámara Raspberry Pi y procesar la imagen para extraer información de la posición de la pelota con las librerías de OpenCV. Cuando Junny logre detectar la pelota tiene que buscarla y acercarse a ella, para completar esta tarea deber\'a tambi\'en hacer los movimientos necesarios para caminar, girar, levantarse y patear usando el software pypose, controlar servomotores para el movimiento de la cámara, establecer mecanismo de comunicación entre la tarjetas ArbotiX y Raspberry Pi, detectar ca\'idas. 
Para llegar a la pelota el robot debe hacerlo por medio de aprendizaje de m\'aquinas especificamente por reforzamiento con ello se debe hacer la implementación de algoritmo de planificación de acciones que lleve al humanoide a acercarse a la pelota con el aprendizaje y la utilizaci\'on de Aprendizaje Q, su entrenamiento y realizar pruebas de desempe\~no, finalmente cuando llegue a la pelota deber\'a ubicar el arco, posicionarse y patear para meter gol. 
    
 %\begin{enumerate}
%\item  Diseño y construcción de un humanoide con piezas del kit de robótica Bioloid Premium, sustituyendo su tarjeta controladora
%CM-510 por la tarjeta de software libre ArbotiX para controlar los motores Dynamixel y otros sensores.
%\begin{enumerate}
%\item Instalación y configuración de la tarjeta ArbotiX.
%\item Instalación y configuración de la tarjeta Raspberry Pi.
%\item Instalación y configuración de la cámara Raspberry Pi.
%\item Instalación de servomotores  para el movimiento de la cámara
%\item Instalación del giroscopio Gyro.
 %\end{enumerate}

%\item Detección de la pelota
%\begin{enumerate}
%\item Captura de imagen con la cámara Raspberry Pi a través de la librería raspicam cv.
%\item Procesamiento de la imagen para extraer información de la posición de la pelota con las librerías de OpenCV.
%\end{enumerate}


%\item Búsqueda de la pelota y pateo de la misma. 
%\begin{enumerate}
%\item Creación de las poses necesarias para caminar, girar, levantarse y patear usando el software pypose.
%\item Programación de transiciones de movimientos.
%\item Control de servomotores para el movimiento de la cámara.
%\item Establecer mecanismo de comunicación entre la tarjetas ArbotiX y Raspberry Pi.  
%\item Programación de algoritmo de planificación de acciones que lleve al humanoide a acercarse a la pelota.
%\item Detección de movimientos angulares bruscos que sugieran una caída, a través de la lectura del giroscopio
%\item Identificación del momento en que la pelota se encuentre en una zona adecuada para patear.
%\end{enumerate}

%\item Aprendizaje
%\begin{enumerate}
%\item Implementaci\'on de aprendizaje por reforzamiento para determinar las acciones correctas en trayecto a la pelota
%\item Utilizaci\'on de Aprendizaje Q como t\'ecnica para el aprendizaje por reforzamiento
%\item Entrenamiento del robot 
%\item Pruebas de desempeño del aprendizaje

%\end{enumerate}

%\item Orientaci\'on al arco
%\begin{enumerate}
%\item Detecci\'on del arco
%\item Programación de algoritmo de planificación de acciones que lleve al humanoide a acercarse a la pelota
%\item Detecci\'ion de posici\'ion adecuada 

%\end{enumerate}


%\end{enumerate}

%Para construir al robot se ha debido considerar las opciones disponibles en cuanto a los componentes y la forma de ensamblarlos, al mismo tiempo se han realizado las instalaciones y configuraciones necesarias en algunos componentes para verificar su funcionamiento. Tal es el caso de las tarjetas Arbotix y Raspberry Pi, la cámara, micro servomotores y giroscopio. La documentación de estas herramientas en algunos casos no era muy amplia, por lo que estas tareas han requerido dedicación especial en las primeras etapas del proyecto. 
% Por ejemplo no existe documentacion oficial para extraer la imagen de la camara en openCV y c++, los avances eran muy recientes
% De la tarjeta Arbotix no se contaba con la documentacion oficial del problema en el que no se quemaban los programas en la tarjeta y no se contaba con el puerto ISP que era necesario.
%Una vez construido el robot, la primera tarea a cumplir ha sido detectar la cercanía de la pelota a través de la cámara. Se decidió usar la técnica de segmentación por regiones con apoyo del conjunto de librerías de OpenCv. Entonces, Junny ha podido observar de forma parcial su medio ambiente y en caso de que detectara la pelota dentro de su rango de visión ha podido identificar su posición con respecto a ella. 
%Luego de identificar la posición del robot con respecto a la pelota, se debió crear el conjunto de movimientos necesarios para que Junny pudiera aproximarse a ella. Ha sido necesaria la creación de poses del cuerpo del robot y la programación de las transiciones entre ellas para formar cada movimiento. Los movimientos son controlados desde la tarjeta Arbotix. De forma general, los movimientos creados han sido caminar, girar, patear y levantarse. De esta forma Junny ha logrado acercarse a la pelota y levantarse en caso de caer. Para detectar de manera autónoma alguna caída se usó un giroscopio, el cual brinda la velocidad angular. Se debió estudiar el comportamiento de los datos recibidos por el giroscopio para establecer los límites dentro de los cuales se podía considerar que el robot seguía en pie y cuándo no.
%Para coordinar la detección de la pelota con los movimientos del robot, se debió establecer la comunicación entre la tarjata Arbotix y la Raspberry Pi. Esto se ha logrado con la instalación y configuración del \gls{framework} ROS. 
%Una mejora que se pudo agregar, ha sido el movimiento físico de la cámara. De esta manera se logró ampliar el rango de visión de Junny, mientras se mantenía sin desplazarse en el ambiente.

%Falta decir que se debió formar una representacion del mundo y elegir la accion más adecuada en cada caso.       


Esta investigaci\'on se divide en 5 cap\'itulos. El cap\'itulo \ref{chap:marco_teorico} se refiere a una base te\'orica de conceptos e informaci\'on necesarias para el soporte del desarrollo de este proyecto que se encuentra en el cap\'itulo  \ref{chapter:introAdesarrollo} donde se explica todo el procedimiento realizado para llevar a cabo el producto. En el cap\'itulo \ref{chapter:resultados} se encuentran  los experimentos y sus resultados. Por \'ultimo estan las conclusiones y recomendaciones en el capítulo \ref{chapter:conclusiones}. 