\chapter*{Introducción}

\pdfbookmark[0]{Introducción}{introduccion} % Sets a PDF bookmark for the dedication

\label{sect:justificacion}

Ya existen aproximaciones a este enfoque, la mayoría relacionados con: el diseño gráfico como 
la herramienta \emph{Brain Storm} del programa \emph{After Effects} de \emph{Adobe} \cite{HR10}, que muestra variaciones de los
elementos en pantalla para sugerir posibles modificaciones; la inteligencia artificial, como 
la herramienta Gecode \cite{Gecode}, que permite resolver problemas algebraicos y lógicos para obtener
rangos de soluciones posibles; y el desarrollo de videojuegos, como los juegos de la saga
\emph{Diablo, Minecraft} \cite{B12}, o \emph{Dwarven Fortress} \cite{D08}, en los que los mapas son generados de forma aleatoria
pero mantienen coherencia.

Incluso hay otro tipo de herramientas o que guardan bastante similitud a esta idea.
Entre ellas destaca \emph{QuickCheck} \cite{TQP}, que es una herramienta en \emph{Haskell} que 
permite generar casos de prueba de estructuras que se han descrito. Sin embargo, tiene
como limitación que sólo está hecha para realizar pruebas y no generar soluciones de
salida para aplicaciones en produción.

En el presente trabajo de investigación se explica el proceso y diseño de la herramienta de generación
de estas instancias aleatorias. Partiendo en el desarrollo de un lenguaje de
programación propio (sección \ref{chapter:def_lenguaje}), para luego ahondar en el proceso de resolución de
problemas y asignación de valores (sección \ref{chapter:dise_solver_compilador}). Por último se evaluarán los resultados de
las pruebas hechas sobre la herramienta y se concluirá acerca de sus posibles mejoras y
enfoques para trabajos futuros (sección \ref{chapter:imp_y_res}).
